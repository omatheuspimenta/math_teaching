\documentclass[oneside,a4paper,12pt]{article}
\usepackage[english,brazilian]{babel}
\usepackage{multicol}
\usepackage{textcomp}
\usepackage[alf]{abntex2cite}
\usepackage[utf8]{inputenc}
\usepackage[T1]{fontenc}
\usepackage{amsmath,amssymb,exscale}
\usepackage[top=20mm, bottom=20mm, left=20mm, right=20mm]{geometry}%margens cima, baixo, esquerda direita
\usepackage{framed}
\usepackage{booktabs} %Pacote para deixar tabelas mais bonitas.
\usepackage{color} %Pacote de Cores
\usepackage{hyperref} %Pacotes para Hiperlinks
\usepackage{graphicx} %Pacote de imagens
\graphicspath{{./Figuras/}}%Direciona as imagens para uma pasta chamada "Figuras" (uso isso para organizar. Uma vez que todas as imagens vao ficar em uma pasta isolada)    
\definecolor{shadecolor}{rgb}{0.8,0.8,0.8}

%FAZ EDICOES AQUI (somente no conteudo que esta entre entre as ultimas  chaves de cada linha!!!)
\newcommand{\universidade}{Universidade Tecnológica Federal do Paraná}
\newcommand{\centro}{Câmpus Cornélio Procópio}
\newcommand{\departamento}{Departamento Acadêmico de Matemática}
\newcommand{\curso}{Engenharia da Computação}
\newcommand{\professores}{Matheus Pimenta}
\newcommand{\disciplina}{Geometria Analítica e Álgebra Linear - EC31G}
%\newcommand{\tema}{Lista 01}
%\newcommand{\turma}{MA31G}
%\newcommand{\data}{Março de 2019}%{\today}
%\newcommand{\tempodeaula}{30 minutos}
%\newcommand{\prerequisitos}{Matrizes, Transformações Lineares e Bases}
%ATE AQUI !!!	

\begin{document}
	\pagestyle{empty}
	
	\begin{center}
		\includegraphics[width=\linewidth/8]{logo.jpg}%LOGOTIPO DA INSTITUICAO
	 	\vspace{2pt} 	
		
		\universidade
		\par
		\centro
		\par
		\departamento
		\par
	%	Curso de \curso
		\par
		\vspace{12pt}
		\LARGE \textbf{Lista 01}
		
	\end{center}
	
	\vspace{12pt}
	
	\begin{tabular}{ |l|p{12cm}| }
		
		\hline
		\multicolumn{2}{|c|}{\textbf{Dados de Identificação}} \\
		\hline
		Professor:         &    \professores           \\
		\hline
		Disciplina:        &    \disciplina          \\
		\hline
	%	Tema:              &    \tema                \\
	%	\hline
	%	Pré-requisito	:  &    \prerequisitos         \\
	%	\hline
		Aluno:             &                   \\
	%	\hline
	%	Data:              &    \data                \\
	%	\hline
	%	Duração da aula:   &    \tempodeaula         \\
		\hline
		
	\end{tabular}
	\vspace{6pt}
	
	
	\begin{snugshade}
	\end{snugshade}

\begin{enumerate}

	\item Determinar se as matrizes são inversíveis e no caso positivo, determinar sua inversa:
	\begin{multicols}{2}
	\begin{enumerate}
		\item $A=\left[
				\begin{array}{ccc}
				1	&	1	&	0	\\
				0	&	1	&	1	\\
				1	&	0	&	2
				\end{array}
				\right]
				$
		\item $B=\left[
		\begin{array}{ccc}
		1	&	2	&	6	\\
		0	&	1	&	5	\\
		2	&	3	&	7
		\end{array}
		\right]
		$
		\item $C=\left[
		\begin{array}{ccc}
		1	&	2	&	1	\\
		0	&	1	&	2	\\
		1	&	1	&	1
		\end{array}
		\right]
		$
		\item $D=\left[
		\begin{array}{ccc}
		1	&	2	&	2	\\
		0	&	1	&	2	\\
		1	&	3	&	4
		\end{array}
		\right]
		$
	\end{enumerate}
	\end{multicols}



	\item Calcule o determinante das matrizes usando o desenvolvimento de Laplace.
	\begin{multicols}{2}
	\begin{enumerate}
			\item $A=\left[
			\begin{array}{ccc}
			2	&	0	&	-1	\\
			3	&	0	&	2	\\
			4	&	-3	&	7
			\end{array}
			\right]
			$
			\item $B=\left[
			\begin{array}{cccc}
			2	&	3	&	-1	&	2	\\
			0	&	4	&	-3	&	5	\\
			1	&	2	&	1	&	3	\\
			0	&	4	&	0	&	0
			\end{array}
			\right]
			$
			\item $C=\left[
			\begin{array}{cccc}
			3	&	4	&	-1	&	3	\\
			0	&	2	&	0	&	3	\\
			0	&	12	&	-9	&	2	\\
			0	&	18	&	-9	&	2
			\end{array}
			\right]
			$
			\item $D=\left[
			\begin{array}{ccc}
			1	&	4	&	3	\\
			-9	&	6	&	6	\\
			0	&	1	&	-1
			\end{array}
			\right]
			$
	\end{enumerate}
	\end{multicols}

	\item Determine as matrizes inversas (se existirem) do exercício anterior.
	
	\item Mostre que, sendo $A$ uma matriz quadrada de ordem $n$ e $r$ um número real, temos a seguinte relação:
	$$\det(rA) = r^{n}\det(A)$$
	
	\item Dadas as matrizes:
	$$
	A=\left[
	\begin{array}{cc}
	1	&	2	\\
	1	&	0	\\
	\end{array}
	\right]
	\text{ e }
	B=\left[
	\begin{array}{cc}
	3	&	-1	\\
	0	&	1	\\
	\end{array}
	\right]
	$$
	Calcule:
	\begin{enumerate}
		\item $\det(A) + \det(B)$
		\item $\det(A+B)$
	\end{enumerate}

	\item Dada a matriz:
	
	$$A=\left[
	\begin{array}{ccc}
		2	&	0	&	-1	\\
		3	&	0	&	2	\\
		4	&	-3	&	7
	\end{array}
	\right]
	$$
	Calcule:
	\begin{enumerate}
		\item $adj(A)$
		\item $\det(A)$
		\item $A^{-1}$
	\end{enumerate}
	
	\item Escreva a matriz $A=(a_{ij})_{2 \times 3}$, onde $a_{ij} = 2i + 3j$
	
	\item Escreva a matriz $B=(b_{ij})_{3 \times 3}$, onde $b_{ij} = \frac{i}{j}$
	
	\item Escreva a matriz $C=(c_{ij})_{4 \times 1}$, onde $c_{ij} = i^2 + j$
	
	\item Determine a matriz $A=(a_{ij})_{4 \times 3}$, onde $a_{ij} = \begin{cases}
	2, \text{ se } i \geq j;\\ -1, \text{ se } i<j
	\end{cases}$
	
	\item Chama-se traço de uma matriz quadrada a soma dos elementos da diagonal principal. Determine o traço de cada uma das matrizes.
	\begin{multicols}{2}
	$A=\left[
	\begin{array}{ccc}
	2	&	0	&	-3	\\
	3	&	5	&	2	\\
	22	&	\sqrt{2}	&	-9
	\end{array}
	\right]
	$
	
	$
	B=\left[
	\begin{array}{cc}
	3	&	-1	\\
	0	&	1	\\
	\end{array}
	\right]$
	\end{multicols}
	
	\item Calcule os seguintes produtos, se possível.
	\begin{multicols}{2}
	\begin{enumerate}
		\item 	$
		5 \left[
		\begin{array}{cc}
		1	&	-1	\\
		4	&	7	\\
		\end{array}
		\right]$
		
		\item 
		$ \left[
		\begin{array}{cc}
		5	&	0	\\
		1	&	4	\\
		\end{array}
		\right]$
		$ \left[
		\begin{array}{cc}
		1	&	4	\\
		2	&	3	\\
		\end{array}
		\right]$
		
		\item 
		$ \left[
		\begin{array}{ccc}
		1	&	2	&	3	\\
		\end{array}
		\right]$
		$ \left[
		\begin{array}{c}
		4	\\
		5	\\
		6	\\
		\end{array}
		\right]$
		
		\item 
		$ \left[
		\begin{array}{c}
		4	\\
		5	\\
		6	\\
		\end{array}
		\right]$
		$ \left[
		\begin{array}{ccc}
		1	&	2	&	3	\\
		\end{array}
		\right]$

		\item 		$ \left[
		\begin{array}{ccc}
		5	&	2	&	12	\\
		1	&	4	&	0	\\
		2	&	4	&	1
		\end{array}
		\right]$
		$ \left[
		\begin{array}{ccc}
		4	&	2	&	2	\\
		7	&	9	&	0	\\
		1	&	5	&	1
		\end{array}
		\right]$
	\end{enumerate}
	\end{multicols}
	
	\item Se $A$, $B \in \mathbb{M}_{n}(\mathbb{K})$, vale $AB = BA$? Demonstre ou dê um contra-exemplo.
	
	\item Sejam $A$, $B$ matrizes quadradas de ordem $n$. É verdadeiro que $(A + B)^2 = A^2 + 2AB + B^2$? Demonstre ou dê um contra-exemplo.
	
	\item Dadas as matrizes 
	$
	A=\left[
	\begin{array}{cc}
	1	&	2	\\
	a	&	3	\\
	\end{array}
	\right]$
	e
	$
	B=\left[
	\begin{array}{cc}
	x	&	3	\\
	b	&	3	\\
	\end{array}
	\right]$
	determinar $a$, $b$ e $x$ para que $A = B^t$.
	
	\item Determine $x$ e $y$ na igualdade:
	$ \left[
	\begin{array}{c}
	\log_{3}x	\\
	y^2	\\
	5	\\
	\end{array}
	\right]$
	=
	$ \left[
	\begin{array}{c}
	4	\\
	9	\\
	5	\\
	\end{array}
	\right]$
	
	\item Determine a matriz adjunta de
	$ A = \left[
	\begin{array}{ccc}
	1	&	3	&	5	\\
	2	&	8	&	1	\\
	0	&	3	&	4
	\end{array}
	\right]$
	
	
	
\end{enumerate}

	
\end{document}
