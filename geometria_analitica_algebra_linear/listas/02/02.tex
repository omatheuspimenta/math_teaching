\documentclass[oneside,a4paper,12pt]{article}
\usepackage[english,brazilian]{babel}
\usepackage{multicol}
\usepackage{textcomp}
\usepackage[alf]{abntex2cite}
\usepackage[utf8]{inputenc}
\usepackage[T1]{fontenc}
\usepackage{amsmath,amssymb,exscale}
\usepackage[top=20mm, bottom=20mm, left=20mm, right=20mm]{geometry}%margens cima, baixo, esquerda direita
\usepackage{framed}
\usepackage{booktabs} %Pacote para deixar tabelas mais bonitas.
\usepackage{color} %Pacote de Cores
\usepackage{hyperref} %Pacotes para Hiperlinks
\usepackage{graphicx} %Pacote de imagens
\graphicspath{{./Figuras/}}%Direciona as imagens para uma pasta chamada "Figuras" (uso isso para organizar. Uma vez que todas as imagens vao ficar em uma pasta isolada)    
\definecolor{shadecolor}{rgb}{0.8,0.8,0.8}

%FAZ EDICOES AQUI (somente no conteudo que esta entre entre as ultimas  chaves de cada linha!!!)
\newcommand{\universidade}{Universidade Tecnológica Federal do Paraná}
\newcommand{\centro}{Câmpus Cornélio Procópio}
\newcommand{\departamento}{Departamento Acadêmico de Matemática}
\newcommand{\curso}{Engenharia da Computação}
\newcommand{\professores}{Matheus Pimenta}
\newcommand{\disciplina}{Geometria Analítica e Álgebra Linear - EC31G}
%\newcommand{\tema}{Lista 01}
%\newcommand{\turma}{MA31G}
%\newcommand{\data}{Março de 2019}%{\today}
%\newcommand{\tempodeaula}{30 minutos}
%\newcommand{\prerequisitos}{Matrizes, Transformações Lineares e Bases}
%ATE AQUI !!!	

\begin{document}
	\pagestyle{empty}
	
	\begin{center}
		\includegraphics[width=\linewidth/8]{logo.jpg}%LOGOTIPO DA INSTITUICAO
	 	\vspace{2pt} 	
		
		\universidade
		\par
		\centro
		\par
		\departamento
		\par
	%	Curso de \curso
		\par
		\vspace{12pt}
		\LARGE \textbf{Lista 02}
		
	\end{center}
	
	\vspace{12pt}
	
	\begin{tabular}{ |l|p{12cm}| }
		
		\hline
		\multicolumn{2}{|c|}{\textbf{Dados de Identificação}} \\
		\hline
		Professor:         &    \professores           \\
		\hline
		Disciplina:        &    \disciplina          \\
		\hline
	%	Tema:              &    \tema                \\
	%	\hline
	%	Pré-requisito	:  &    \prerequisitos         \\
	%	\hline
		Aluno:             &                   \\
	%	\hline
	%	Data:              &    \data                \\
	%	\hline
	%	Duração da aula:   &    \tempodeaula         \\
		\hline
		
	\end{tabular}
	\vspace{6pt}
	
	
	\begin{snugshade}
	\end{snugshade}

\begin{enumerate}

	\setcounter{enumi}{17}

	\item Determinar a forma escalonada de cada matriz:
	\begin{multicols}{2}
	\begin{enumerate}
		\item $A=\left[
		\begin{array}{cccc}
		1	&	-3	&	0	&	2 \\
		1	&	-3	&	1	&	1 \\
		0	&	-3	&	1	&	1 \\
		\end{array}
		\right]
		$
		\item $B=\left[
		\begin{array}{ccc}
		1	&	2	&	3	\\
		2	&	4	&	6	\\
		-3	&	-6	&	-9
		\end{array}
		\right]
		$
		\item $C=\left[
		\begin{array}{ccc}
		1	&	0	&	1	\\
		1	&	1	&	0	\\
		-1	&	1	&	1
		\end{array}
		\right]
		$
		\item $D = \left[
		\begin{array}{ccc}
		-1	&	1	&	-1	\\
		1	&	-1	&	1	\\
		-1	&	1	&	1
		\end{array}
		\right]$
	\end{enumerate}
	\end{multicols}



	\item Classifique os sistemas quanto a serem impossíveis, possíveis determinados ou indeterminados.
	\begin{multicols}{2}
	\begin{enumerate}
			\item $\begin{cases}
				x + y = 0 \\
				2x - y = 0
			\end{cases}$
			\item $\begin{cases}
			x + y = 0 \\
			2x + y = 0
			\end{cases}$
			\item $\begin{cases}
			x + y = 0 \\
			2x + 2y = 0
			\end{cases}$
			\item $\begin{cases}
			x + y = 0 \\
			2x + 2y = 2
			\end{cases}$
	\end{enumerate}
	\end{multicols}

	\item Encontre as soluções do sistema $Ax=0$, onde $x = (x_1, \dots, x_n)^{T}$ (para o valor conveniente de $n$) onde $A$ é cada uma das matrizes do exercício $18$.
	
	\item Resolva os seguintes sistemas de equações lineares:
	\begin{multicols}{2}
	\begin{enumerate}
		\item $\begin{cases}
		3x + y = 5 \\
		2x - 3y = -4
		\end{cases}$
		\item $\begin{cases}
		x + 2y - 3z = 9 \\
		3x - y + 4z = -5 \\
		2x + y + z = 0
		\end{cases}$
	\end{enumerate}
	\end{multicols}	
	
	\item Determine para quais valores de $k$ o sistema $\begin{cases}
	x + 2y = 3 \\
	2x + ky = 2
	\end{cases}
	$ é:
	\begin{enumerate}
		\item possível e determinado;
		\item possível e indeterminado;
		\item impossível
	\end{enumerate}

	\item Analise as soluções do seguinte sistema de equações lineares: $\begin{cases}
	-x + 2y + z - 2 = 0 \\
	x + 2y - 2z = 0 \\
	x - 4y + 10z - 6 = 0 \\
	2x + 7y - 5z - 2 = 0
	\end{cases}
	$
	
	\item Analise as soluções dos seguintes sistemas de soluções lineares (resolva via escalonamento):
	\begin{multicols}{2}
	\begin{enumerate}
		\item $\begin{cases}
		x + 2y - 3z = 4 \\
		2x + 3y + 4z = 5 \\
		4x + 7y - 2z = 12
		\end{cases}$
		\item $\begin{cases}
		x + 2y - 3z = 4 \\
		3y + 2x + 4z = 5 \\
		7y - 2z + 4x = 13
		\end{cases}$
		\item $\begin{cases}
		x - 2y + 3z = 4 \\
		2x - 4y + 6z = 5 \\
		2x - 6y + 9z = 12
		\end{cases}$
		\item $\begin{cases}
		5732x + 2134y + 2134z = 7866 \\
		2134x + 5732y + 2134z = 670 \\
		2134x + 2134y + 5732z = 11464 
		\end{cases}$
		\item $\begin{cases}
		x + 3y + 5z + 7w = 12 \\
		3x + 5y + 7z + w = 0 \\
		5x + 7y + z + 3w = 4 \\
		7x + y + 3z + 5w = 16
		\end{cases}$
		\item $\begin{cases}
		x + z = 2 \\
		y + z = 4 \\
		x + y = 5 \\
		x + y + z = 0
		\end{cases}$
		\item $\begin{cases}
		x - 2y + z + t = 1 \\
		2x + y - 2z + 2t = 0 \\
		x + 6y = -2
		\end{cases}$
	\end{enumerate}
	\end{multicols}		
	
	\item Agora resolva o exercício anterior via método de Cramer (se possível).
	
	\item Determine para quais valores de $m$ e $n$ o sistema $\begin{cases}
	2x - y + 3z = 1 \\
	x + 2y - z = 4  \\
	3x + y + mz = n
	\end{cases}
	$ é:
	\begin{enumerate}
		\item possível e indeterminado;
		\item impossível
	\end{enumerate}
	
	\item Analise as soluções dos seguintes sistemas de soluções lineares (resolva via escalonamento):
	\begin{multicols}{2}
	\begin{enumerate}
		\item $\begin{cases}
		2x - y + 3z = 11 \\
		4x - 3y + 2z = 0 \\
		x + y + z = 6 \\
		3x + y + z = 4
		\end{cases}$
		\item $x_1 + 2x_2 - x_3 + 3x_4 = 1$
		\item $\begin{cases}
		x+y+z = 4 \\
		2x +5y - 2z = 3
		\end{cases}$
		\item $\begin{cases}
		x + y+z = 4 \\
		2x + 5y - 2z = 3 \\
		x + 7y - 7z = 5
		\end{cases}$
	\end{enumerate}	
	\end{multicols}
	
	\item Agora resolva o exercício anterior via método de Cramer (se possível).
	
	\item Analise as soluções dos seguintes sistemas de soluções lineares (resolva via escalonamento):
	\begin{multicols}{2}
		\begin{enumerate}
			\item $\begin{cases}
			x - 2y + 3z = 0 \\
			2x + 5y + 6z = 0
			\end{cases}$
			\item $\begin{cases}
			x_1 + x_2 + x_3 + x_4 = 0 \\
			x_1 + x_2 + x_3 - x_4 = 4 \\
			x_1 + x_2 - x_3 + x_4 = -4 \\
			x_1 - x_2 + x_3 + x_4 = 2 
			\end{cases}$
			\item $\begin{cases}
			x+2y+3z = 0 \\
			2x +y + 3z = 0 \\
			3x + 2y + z = 0
			\end{cases}$
			\item $\begin{cases}
			3x + 2y - 4z = 1 \\
			x - y + z = 3 \\
			x - y - 3z = -3 \\
			3x + 3y - 5z = 0 \\
			-x + y +z = 1
			\end{cases}$
		\end{enumerate}	
	\end{multicols}		
		
	\item Determine o posto e a nulidade das seguintes matrizes:
	\begin{multicols}{3}
		\begin{enumerate}
			\item 	$
			\left[
			\begin{array}{cccc}
			1	&	-2 & 3 & -1 \\
			2 	&	-1 & 2 & 3 \\
			3	&	 1 & 2 & 3 
			\end{array}
			\right]$
			\\ {\bf R: (forma escada) } \\
			$\left[
			\begin{array}{cccc}
			1	&	0 & 0 & -4 \\
			0 	&	1 & 0 & -3 \\
			0	&	0 & 1 & -1 
			\end{array}
			\right]$
			\item 				
			$\left[
			\begin{array}{cccc}
				0	&	1 & 3 & -2 \\
				2 	&	1 & -4 & 3 \\
				2	&	 3 & 2 & -1 
			\end{array}
			\right]$
			\\ {\bf R: (forma escada) } \\
			$\left[
			\begin{array}{cccc}
			1	&	0 & -7/2 & 5/2 \\
			0 	&	1 & 3 & -2 \\
			0	&	0 & 0 & 0 
			\end{array}
			\right]$
			\item 	$
			\left[
			\begin{array}{ccc}
			0	&	2 & 2 \\
			1	&	1 & 3 \\
			3	&	-4 & 2 \\
			2 	& 	-3 &1
			\end{array}
			\right]$
			\\ {\bf R: (forma escada) } \\
			$\left[
			\begin{array}{ccc}
			1	&	0 & 2  \\
			0 	&	1 & 1  \\
			0	&0& 0 \\
			0	&0&0 \\
			\end{array}
			\right]$
		\end{enumerate}
	\end{multicols}
	
	\item Determine $k$ para que o sistema admita solução. $\begin{cases}
	-4x + 3y = 2 \\
	5x - 4y = 0 \\
	2x - y = k
	\end{cases}$
	
	\item Estude a solução do sistema de equação linear:
	$\begin{cases}
	x_1 + 3x_2 + 2x_3 + 3x_4 - 7x_5 = 14 \\
	2x_1 + 6x_2 + x_3 - 2x_4 + 5x_5 = -2 \\
	x_1 + 3x_2 - x_3 + 2x_5 = -1 \\		
	\end{cases}$
	
	\item Sabe-se que uma alimentação diária equilibrada em vitaminas deve constar de $170$ unidades de vitamina $A$, $180$ unidades de vitamina $B$, $140$ unidades de vitamina $C$, $180$ unidades de vitamina $D$ e $350$ unidades de vitamina $E$. Com o objetivo de descobrir como deverá ser uma refeição equilibrada, foram estudados $5$ alimentos. Fixada a mesma quantidade (1g) de cada alimento, determinou-se que:
	\begin{enumerate}
		\item o alimento 1 tem $1$ unidade de vitamina $A$, $10$ unidades de vitamina $B$, $1$ unidade de vitamina $C$, $2$ unidades de vitamina $D$ e $2$ unidades de vitamina $E$.
		\item o alimento 2 tem $9$ unidades de vitamina $A$, $1$ unidade de vitamina $B$, não contém vitamina $C$, $1$ unidade de vitamina $D$ e $1$ unidade de vitamina $E$.
		\item o alimento 3 tem $2$ unidades de vitamina $A$, $2$ unidades de vitamina $B$, $5$ unidades de vitamina $C$, $1$ unidade de vitamina $D$ e 2 unidades de vitamina $E$.
		\item o alimento 4 tem $1$ unidade de vitamina $A$, $1$ unidade de vitamina $B$, $1$ unidade de vitamina $C$, $2$ unidades de vitamina $D$ e $13$ unidades de vitamina $E$.
		\item o alimento 5 tem $1$ unidade de vitamina $A$, $1$ unidade de vitamina $B$, $1$ unidade de vitamina $C$, $9$ unidades de vitamina $D$ e $2$ unidades de vitamina $E$.
	\end{enumerate}
	Quantos gramas de cada um dos alimentos $1,2,3,4$ e $5$ devemos ingerir diariamente para que nossa alimentação seja equilibrada?
	
\end{enumerate}

	
\end{document}
