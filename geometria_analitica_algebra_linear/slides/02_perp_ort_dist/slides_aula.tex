\documentclass[hyperref={pdfpagelabels=false}]{beamer}
\usepackage{lmodern}
\usetheme{CambridgeUS}

\usepackage[english,brazilian]{babel}
\usepackage{multicol}
\usepackage{textcomp}
\usepackage[alf]{abntex2cite}
\usepackage[utf8]{inputenc}
\usepackage[T1]{fontenc}

\title{Perpendicularidade, Ortogonalidade, Medida Angular e Distância}  
\author[Matheus Pimenta]{Matheus Pimenta} 
\institute[UTFPR-CP]{\normalsize Universidade Tecnológica Federal do Paraná \\
	Câmpus Cornélio Procópio
} 
\date{Abril de 2019} 
\begin{document}
	
\begin{frame}
\titlepage
\end{frame} 


%\begin{frame}
%\frametitle{Table of contents}
%\tableofcontents
%\end{frame} 


\section{Perpendicularidade e Ortogonalidade} 


\begin{frame}
\frametitle{Retas} 
\begin{itemize}

	\item Duas retas ortogonais podem ser concorrentes ou reversas e duas retas perpendiculares são obrigatoriamente concorrentes.
	\item Duas retas são ortogonais se, e somente se, cada vetor diretor de uma é ortogonal a qualquer vetor diretor da outra.
\end{itemize}
\end{frame}

\begin{frame}
\frametitle{Exemplo}
\begin{itemize}
	\item Verifique se as retas $r: X = (1,1,1) + \lambda(2,1,-3)$ e $s: X = (0,1,0) + \lambda(-1,2,0)$ são ortogonais. Caso sejam, verifique se são perpendiculares.
\end{itemize}

\end{frame}

\begin{frame}
\frametitle{Vetor Normal a um Plano}
{\bf Definição:} Dado um plano $\pi$, qualquer vetor não-nulo ortogonal a $\pi$ é um {\bf vetor normal} a $\pi$.

\pause

Um vetor $\overrightarrow{n}$ não-nulo é perpendicular ao plano $\pi$ se, e somente se, $\overrightarrow{n}$ é ortogonal a qualquer vetor diretor de $\pi$.

\pause
Assim, se $(\overrightarrow{u}, \overrightarrow{v})$ é um par de vetores diretores de $\pi$, então $\overrightarrow{u} \land \overrightarrow{v}$ (ou qualquer um de seus múltiplos escalares) é vetor normal a $\pi$.

\end{frame}

\begin{frame}
\frametitle{Proposição}

{\bf Proposição: } Se o sistema de coordenadas é ortogonal, então $\overrightarrow{n} = (a,b,c)$ é um vetor normal ao plano $\pi$ se, e somente se, $\pi$ tem uma equação geral da forma $ax + by + cz + d = 0$.

\end{frame}

\begin{frame}
\frametitle{Exemplo}

Obtenha a equação geral do plano $\pi$ que contém o ponto $A = (1,0,2)$ sabendo que $\overrightarrow{n} = (1,1,4)$ é um vetor normal a $\pi$.

\end{frame}


\begin{frame}
\frametitle{Plano e Reta}

Se $\overrightarrow{n}$ é um vetor normal ao plano $\pi$ e $\overrightarrow{r}$ é um vetor diretor da reta $r$, então $r$ e $\pi$ são perpendiculares se, e somente se, $\overrightarrow{r}$ e $\overrightarrow{n}$ são paralelos.
\end{frame}

\begin{frame}
\frametitle{Planos}

Se $\overrightarrow{n}_1$ e $\overrightarrow{n}_2$ são vetores normais aos planos $\pi_1$ e $\pi_2$, então os planos são perpendiculares se, e somente se, $\overrightarrow{n}_1$ e $\overrightarrow{n}_2$ são ortogonais, isto é, $\overrightarrow{n}_1 \cdot \overrightarrow{n}_2 = 0$

\end{frame}

\begin{frame}
\frametitle{Exemplo} 

Verifique se $\pi_1:X = (0,0,1) + \lambda(1,0,1) + \mu(-1,1,1)$ e $\pi_2: 2x - 7y + 16z -40=0$ são perpendiculares.

\end{frame}

\section{Medida Angular}

\begin{frame}
\frametitle{Definição}

{\bf Definição:} Sejam $r$ e $s$ duas retas, $\overrightarrow{r}$ e $\overrightarrow{s}$ vetores diretores. A {\bf medida angular entre $r$ e $s$} é a medida angular entre os vetores $\overrightarrow{r}$ e $\overrightarrow{s}$, se esta pertence ao intervalo $[0,\pi/2]$ $rad$ ou $[0,90]$ (graus) e é a medida angular entre $\overrightarrow{r}$ e $ - \overrightarrow{s}$, se pertence a $[\pi/2,\pi]$ ou a $[90,180]$. Notação: ang(r,s)

{\bf OBS:} Se $\theta = 0$, então $r$ e $s$ são paralelas e se $\theta = 90$, então $r$ e $s$ são ortogonais.
\pause

$$\cos(\theta) = \frac{|\overrightarrow{r} \cdot \overrightarrow{s}|}{||\overrightarrow{r}||||\overrightarrow{s}||}$$
\end{frame}

\begin{frame}
\frametitle{Exemplo}

Calcule a medida angular $\theta$ entre as retas $r:X = (1,1,9) + \lambda(0,-1,1)$ e $s:(1,3,2) + \mu(1,1,0)$
\end{frame}

\begin{frame}
\frametitle{Plano e Reta}
{\bf Definição: } Sejam $r$ uma reta e $\pi$ um plano. A {\bf medida angular} entre $r$ e $\pi$ é $90 - ang(r,s)$, sendo $s$ uma reta qualquer perpendicular a $\pi$. Notação: $ang(r,\pi)$

\pause

$$\sin(\theta) = \frac{|\overrightarrow{n} \cdot \overrightarrow{r}|}{||\overrightarrow{n}||||\overrightarrow{r}||}$$

\end{frame}

\begin{frame}
\frametitle{Exemplo}
Obtenha a medida angular em radianos entre a reta $r: X = (0,1,0) + \lambda(-1,-1,0)$ e o plano $\pi:y+z-10=0$


\end{frame}


\begin{frame}
\frametitle{Planos}

{\bf Exemplo: } Sendo $\pi_1: x -y + z = 20$ e $\pi_2: X= (1,1-2) + \lambda(0,-1,1) + \mu(1,-3,2)$


\end{frame}

\section{Distância}

\begin{frame}
\frametitle{Entre Pontos}

Sejam $A = (x_1, y_1, z_1)$ e $B = (x_2, y_2, z_2)$. A distância $d(A,B)$ entre $A$ e $B$ é $||\overrightarrow{BA}||$, ou seja, 
$$d(A,B) = \sqrt{(x_1 - x_2)^2 + (y_1 - y_2)^2 + (z_1 - z_2)^2}$$

\end{frame}

\begin{frame}
\frametitle{Entre Ponto e Reta}

Uma forma é através da projeção ortogonal do ponto a reta.

\pause

Outra forma é: sejam $A$ e $B$ dois pontos quaisquer de $r$, distintos. A área do triângulo $ABP$ é $||\overrightarrow{AP} \land \overrightarrow{AB}||/2$, logo se $h$ é a altura relativa ao vértice $P$, segue que:
$$d(P,r) = \frac{||\overrightarrow{AP} \land \overrightarrow{AB}||}{||\overrightarrow{AB}||}$$

\pause
Indicando por $\overrightarrow{r}$ o vetor $\overrightarrow{AB}$, que é vetor diretor de $r$, obtemos

$$d(P,r) = \frac{||\overrightarrow{AP} \land \overrightarrow{r}||}{||\overrightarrow{r}||}$$

onde $\overrightarrow{r}$ é um vetor diretor de $r$ e $A$ é um ponto qualquer de $r$.

\end{frame}

\begin{frame}
\frametitle{Exemplo}

Calcule a distância de $P = (1,1,-1)$ à interseção de $\pi_1: x-y=1$ e $\pi_2:x+y-z=0$

\end{frame}

\begin{frame}
\frametitle{Entre Ponto e Plano}

Através da projeção do ponto no plano, ou através da projeção ortogonal de $\overrightarrow{AP}$ sobre $\overrightarrow{n}$, onde $\overrightarrow{n}$ é um vetor normal de $\pi$.

\pause

$$d(P,\pi) = \frac{|\overrightarrow{AP}\cdot\overrightarrow{n}|}{||\overrightarrow{n}||}$$

A versão com coordenadas será:

$$d(P,\pi) = \frac{|ax_0 + by_0 + cz_0 + d|}{\sqrt{a^2 + b^2 + c^2}}$$

Onde $P = (x_0, y_0, z_0)$, $A = (x_1, y_1, z_1)$ e $\pi = ax + by + cz + d = 0$

\end{frame}

\begin{frame}
\frametitle{Exemplo}

Calcule a distância do ponto $P = (1,2,-1)$ ao plano $\pi : 3x - 4y -5z + 1 = 0$

\end{frame}


\begin{frame}
\frametitle{Distância entre Retas}

Outra forma de se calcular a distância entre retas é utilizando:

$$d(r,s) = \frac{|\overrightarrow{AB}\cdot \overrightarrow{r}\land \overrightarrow{s}|}{||\overrightarrow{r}\land \overrightarrow{s}||}$$

Onde $B$ é ponto qualquer de $s$ e $A$ é ponto qualquer de $r$.

\pause

{\bf OBS:} Se $r$ e $s$ são concorrentes a distância é zero
\\ 
Se $r$ e $s$ são paralelas não podemos utilizar a fórmula acima, temos que selecionar pontos quaisquer entre elas e calcular a distância entre eles.

\end{frame}


\begin{frame}
\frametitle{Distância entre Reta e Plano}

Utilizamos um vetor diretor da reta $r$ e um vetor normal $\overrightarrow{n}$ ao plano $\pi$ e calculamos $\overrightarrow{r}\cdot \overrightarrow{n}$, se:
\begin{itemize}
	\item $\overrightarrow{r}\cdot \overrightarrow{n} \neq 0$, $r$ é transversal a $\pi$, logo, a distância é zero;
	\item $\overrightarrow{r}\cdot \overrightarrow{n} = 0$, $r$ está contida em $\pi$ e neste caso a distância é zero, {\bf pode também} a reta ser paralela ao plano $\pi$, neste caso a distância é a distância de um ponto qualquer de $r$ ao plano $\pi$.
\end{itemize} 

\end{frame}

\begin{frame}
\frametitle{Distância entre Planos}

Analisaremos os vetores normais, assim se:
\begin{itemize}
	\item $(\overrightarrow{n_1},\overrightarrow{n_2})$ é LI, então $\pi_1$ e $\pi_2$ são transversais e a distância é zero;
	\item $(\overrightarrow{n_1},\overrightarrow{n_2})$ é LD, então $\pi_1$ e $\pi_2$ são paralelos e a distância entre eles é dada através da distância entre dois pontos quaisquer deles.
\end{itemize}

\end{frame}








\end{document}

