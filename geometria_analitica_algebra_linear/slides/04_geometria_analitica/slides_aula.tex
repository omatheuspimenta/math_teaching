\documentclass[hyperref={pdfpagelabels=false}]{beamer}
\usepackage{lmodern}
\usetheme{CambridgeUS}

\usepackage[english,brazilian]{babel}
\usepackage{multicol}
\usepackage{textcomp}
\usepackage[alf]{abntex2cite}
\usepackage[utf8]{inputenc}
\usepackage[T1]{fontenc}

\title{Geometria Analítica}  
\author[Matheus Pimenta]{Matheus Pimenta} 
\institute[UTFPR-CP]{\normalsize Universidade Tecnológica Federal do Paraná \\
	Câmpus Cornélio Procópio
} 
\date{Setembro de 2019} 
\begin{document}
	
\begin{frame}
\titlepage
\end{frame} 


%\begin{frame}
%\frametitle{Table of contents}
%\tableofcontents
%\end{frame} 


\section{Soma de Ponto com Vetor} 


\begin{frame}
\frametitle{Definição} 
Dados um ponto $P$ e um vetor $\overrightarrow{u}$, o ponto $Q$ tal que o segmento orientado $(P,Q)$ é representante de $\overrightarrow{u}$ é chamado de \emph{soma de $P$ com $\overrightarrow{u}$} e indicado por $P + \overrightarrow{u}$. 

Simbolicamente: $$P + \overrightarrow{u} = Q \Leftrightarrow \overrightarrow{PQ} = \overrightarrow{u}$$

Decorre da definição que, quaisquer que sejam os pontos $P$ e $Q$, $$P + \overrightarrow{PQ} = Q$$
\end{frame}

\begin{frame}
\frametitle{Definição}
Pode-se entender $P + \overrightarrow{u}$ como o resultado do deslocamento de um ponto material, inicialmente situado na origem do segmento até a sua extremidade.

{\bf OBS: } A notação $P - \overrightarrow{u}$ indica a soma de $P$ com o vetor oposto de $\overrightarrow{u}$.

{\bf OBS: } A operação que ao par ordenado $(P,\overrightarrow{u})$ associa o ponto $P + \overrightarrow{u}$ é chamada \emph{adição de ponto com vetor.} 

\end{frame}

\begin{frame}
\frametitle{Propriedades}
Quaisquer que sejam os pontos $A$ e $B$ e os vetores $\overrightarrow{u}$ e $\overrightarrow{v}$, valem as propriedades:
\begin{enumerate}
	\item [P1] $(A + \overrightarrow{u}) + \overrightarrow{v} = A + (\overrightarrow{u} + \overrightarrow{v})$
	\pause
	\item [P2] $A + \overrightarrow{u} = A + \overrightarrow{v} \Rightarrow \overrightarrow{u} = \overrightarrow{v} $
	\pause
	\item [P3] $A + \overrightarrow{u} = B + \overrightarrow{u} \Rightarrow A = B $
	\pause
	\item [P4] $(A - \overrightarrow{u}) + \overrightarrow{u} = A$
\end{enumerate}
\end{frame}

\section{Dependência Linear}

\begin{frame}
\frametitle{Introdução}

Diversas aplicações práticas no estudo de paralelismo são facilmente solucionadas através da utilização dos conceitos de dependência linear. 

\pause

Seja $n \in \mathbb{N}$, o símbolo $(\overrightarrow{v}_1,\overrightarrow{v}_2, \dots, \overrightarrow{v}_n)$ indica a sequência, ou $n-upla$ ordenada.
$$(\overrightarrow{v}_1,\overrightarrow{v}_2, \dots, \overrightarrow{v}_n) = (\overrightarrow{u}_1,\overrightarrow{u}_2, \dots, \overrightarrow{u}_n)$$ \\ $$\Leftrightarrow$$ \\ $$\overrightarrow{v}_1 = \overrightarrow{u}_1, \overrightarrow{v}_2 = \overrightarrow{u}_2, \dots, \overrightarrow{v}_n = \overrightarrow{u}_n$$

\end{frame}

\begin{frame}
\frametitle{Definição}

Em nossos estudos, o conceito de \emph{dependência linear} de uma sequência $(\overrightarrow{v}_1,\overrightarrow{v}_2, \dots, \overrightarrow{v}_n)$ será definido caso a caso, dependendo do valor de $n$.

\begin{enumerate}
	\item Uma sequência $(\overrightarrow{v})$ é {\bf linearmente dependente} se $\overrightarrow{v} = \overrightarrow{0}$ e {\bf linearmente independente} se $\overrightarrow{v} \neq \overrightarrow{0}$
	\pause
	\item Uma sequência $(\overrightarrow{u},\overrightarrow{v})$ é {\bf linearmente dependente} se $\overrightarrow{u}$ e $\overrightarrow{v}$ são paralelos. Caso contrário, $(\overrightarrow{u},\overrightarrow{v})$ é {\bf linearmente independente}.
	\pause
	\item Uma tripla ordenada $(\overrightarrow{u}, \overrightarrow{v}, \overrightarrow{w})$ é {\bf linearmente dependente} se $\overrightarrow{u}, \overrightarrow{v}$ e $\overrightarrow{w}$ são paralelos a um mesmo plano. Caso contrário, $(\overrightarrow{u}, \overrightarrow{v}, \overrightarrow{w})$ é {\bf linearmente independente}.
	\pause
	\item Se $n \geq 4$, qualquer sequência de $n$ vetores é {\bf linearmente dependente}.
\end{enumerate}

\end{frame}


\begin{frame}
\frametitle{Observação}

Dependência e independência linear são qualidades referentes a uma sequência de vetores e não aos próprios vetores. Usualmente vamos tratar como ``Os vetores $\overrightarrow{u}$ e $\overrightarrow{v}$ são LI'' tratando que o par ordenado $(\overrightarrow{u},\overrightarrow{v})$ é LI.
\end{frame}

\begin{frame}
\frametitle{Combinação Linear}

Se $\overrightarrow{u} = \alpha_1\overrightarrow{v}_1 + \alpha_2\overrightarrow{v}_2+\dots+\alpha_n\overrightarrow{v}_n$, dizemos que $\overrightarrow{u}$ é {\bf combinação linear} de $\overrightarrow{v}_1, \overrightarrow{v}_2, \dots, \overrightarrow{v}_n$, ou que $\overrightarrow{u}$ é {\bf gerado} por $\overrightarrow{v}_1, \overrightarrow{v}_1, \overrightarrow{v}_3,\dots, \overrightarrow{v}_n$. Os escalares $\alpha_1,\alpha_2,\dots,\alpha_n$ são chamados {\bf coeficientes} da combinação linear.

\end{frame}

\begin{frame}
\frametitle{Exemplo} 

\begin{enumerate}
	\item Sabe-se que $\overrightarrow{u} = 3\overrightarrow{v}$. Escreva três expressões diferentes do vetor nulo como combinação linear de $\overrightarrow{u}, \overrightarrow{v}$.
	\item Refaça o item anterior, supondo que $(\overrightarrow{u},\overrightarrow{v})$ seja LI.
\end{enumerate}
\pause

A partir de agora, ao invés de analisar se a sequência de vetores são paralelos a um plano dado, basta verificar se algum deles é gerado pelos demais.
\end{frame}

\begin{frame}
\frametitle{Proposição}

{\bf Proposição: } Se $(\overrightarrow{u},\overrightarrow{v})$ é LI, então $(\overrightarrow{u},\overrightarrow{v},\overrightarrow{w})$ é LD se, e somente se, $\overrightarrow{w}$ é gerado por $\overrightarrow{u}, \overrightarrow{v}$

\pause

{\bf Proposição: } $(\overrightarrow{u},\overrightarrow{v},\overrightarrow{w})$ é LD se, e somente se, um dos vetores é gerado pelos outros dois.
\end{frame}

\begin{frame}
\frametitle{Proposição}

{\bf Proposição: } Se $(\overrightarrow{u},\overrightarrow{v},\overrightarrow{w})$ é LI, então qualquer vetor $\overrightarrow{x}$ é combinação linear de $\overrightarrow{u},\overrightarrow{v},\overrightarrow{w}$.

\pause

{\bf Proposição: } Uma sequência $(\overrightarrow{v}_1,\overrightarrow{v}_2,\dots,\overrightarrow{v}_n)$, com $n \geq 2$, é LD se, e somente se, algum vetor da sequência é gerado pelos demais.


\end{frame}

\begin{frame}
\frametitle{Proposição}
{\bf Proposição: } Uma sequência $(\overrightarrow{v}_1,\overrightarrow{v}_2,\dots,\overrightarrow{v}_n)$, com $1 \leq n \leq 3$, é LI se, e somente se, a equação $\alpha_1\overrightarrow{v}_1 + \alpha_2\overrightarrow{v}_2+\dots+\alpha_n\overrightarrow{v}_n = \overrightarrow{0}$ admite {\bf apenas} a solução nula $\alpha_1=\alpha_2=\dots=\alpha_n=0$

\pause

{\bf Corolário: } Se $(\overrightarrow{v}_1,\overrightarrow{v}_2,\dots,\overrightarrow{v}_n)$ é LI, então, para cada vetor gerador por $\overrightarrow{v}_1,\overrightarrow{v}_2,\dots,\overrightarrow{v}_n$, os coeficientes são univocamente determinados, isto é,
$$\alpha_1\overrightarrow{v}_1 + \alpha_2\overrightarrow{v}_2+\dots+\alpha_n\overrightarrow{v}_n = \beta_1\overrightarrow{v}_1 + \beta_2\overrightarrow{v}_2+\dots+\beta_n\overrightarrow{v}_n$$ $$\Rightarrow$$ $$\alpha_1 = \beta_1, \alpha_2=\beta_2,\dots,\alpha_n=\beta_n$$

\end{frame}

\begin{frame}
\frametitle{Preposição}

{\bf Preposição: } Uma sequência $(\overrightarrow{v}_1,\overrightarrow{v}_2,\dots,\overrightarrow{v}_n)$, com $1 \leq n \leq 3$, é LD se, e somente se, a equação $\alpha_1\overrightarrow{v}_1 + \alpha_2\overrightarrow{v}_2+\dots+\alpha_n\overrightarrow{v}_n = \overrightarrow{0}$ admite solução não-nula, isto é, existem escalares $\alpha_1,\alpha_2,\dots,\alpha_n$, não todos nulos, tais que  $\alpha_1\overrightarrow{v}_1 + \alpha_2\overrightarrow{v}_2+\dots+\alpha_n\overrightarrow{v}_n = \overrightarrow{0}$

\end{frame}

\section{Base}

\begin{frame}
\frametitle{Definição}

Uma tripla ordenada linearmente independente $E = (\overrightarrow{e}_1,\overrightarrow{e}_2,\overrightarrow{e}_3)$ chama-se {\bf base} de $\mathbb{V}^3$.

\pause

Sendo $E = (\overrightarrow{e}_1,\overrightarrow{e}_2,\overrightarrow{e}_3)$ uma base, podemos dizer que todo vetor $\overrightarrow{u}$ é gerado por $\overrightarrow{e}_1,\overrightarrow{e}_2,\overrightarrow{e}_3$, isto é, existem escalares $\alpha_1,\alpha_2,\alpha_3$ tais que
$$\overrightarrow{u} = \alpha_1\overrightarrow{e}_1 + \alpha_2\overrightarrow{e}_2 + \alpha_3\overrightarrow{e}_3$$
Essa tripla ordenada de escalares é única, devido a unicidade da combinação linear, e é chamada como {\bf coordenada de $\overrightarrow{u}$ em relação à base $E$}

Notação: $\overrightarrow{u} = (\alpha_1,\alpha_2,\alpha_3)_E$ ou $\overrightarrow{u} = (\alpha_1,\alpha_2,\alpha_3)$

\end{frame}

\begin{frame}
\frametitle{Propriedades}

{\bf Proposição: } \begin{enumerate}
	\item $ (\alpha_1,\alpha_2,\alpha_3)_E + (\beta_1,\beta_2,\beta_3)_E = (\alpha_1 + \beta_1, \alpha_2 + \beta_2, \alpha_3 + \beta_3)_E$
	\item $\rho(\alpha_1,\alpha_2,\alpha_3)_E = (\rho\alpha_1,\rho\alpha_2,\rho\alpha_3)_E$
\end{enumerate}

\end{frame}

\begin{frame}
\frametitle{Proposição}

Os vetores $\overrightarrow{u} = (a_1,b_1,c_1)_E$ e $\overrightarrow{v} = (a_2,b_2,c_2)_E$ são LD se, e somente se, $a_1,b_1,c_1$ e $a_2,b_2,c_2$ são proporcionais ou, equivalentemente, os três determinantes
\begin{center}
	$\left|
	\begin{array}{cc}
	a_1	&	b_1	\\
	a_2	&	b_2
	\end{array}
	\right|
	$
	,
	$\left|
	\begin{array}{cc}
	a_1	&	c_1	\\
	a_2	&	c_2
	\end{array}
	\right|
	$
	,
	$\left|
	\begin{array}{cc}
	b_1	&	c_1	\\
	b_2	&	c_2
	\end{array}
	\right|
	$
\end{center}
são nulos.
\end{frame}

\begin{frame}
\frametitle{Proposição}

Os vetores $\overrightarrow{u} = (a_1,b_1,c_1)_E$, $\overrightarrow{v} = (a_2,b_2,c_2)_E$ e $\overrightarrow{w} = (a_3,b_3,c_3)_E$ são LD se, e somente se,
\begin{center}
	$\left|
	\begin{array}{ccc}
	a_1	&	b_1	&	c_1	\\
	a_2	&	b_2	&	c_2	\\
	a_3	&	b_3	&	c_3
	\end{array}
	\right| = 0
	$
\end{center}


\end{frame}

\begin{frame}
\frametitle{Exemplo}

Verifique se são LI ou LD os vetores $\overrightarrow{u} = (1,-1,2)_E$, $\overrightarrow{v} = (0,1,3)_E$, $\overrightarrow{w} = (4,-3,11)_E$.

\end{frame}

\begin{frame}
\frametitle{Definição}

\begin{enumerate}
	\item Os vetores não-nulos $\overrightarrow{u}$ e $\overrightarrow{v}$ são {\bf ortogonais} se existe um representante $(A,B)$ de um deles e um representante $(C,D)$ do outro tais que $AB$ e $CD$ sejam ortogonais.
	
	Notação: $\overrightarrow{u} \perp \overrightarrow{v}$
	\item O vetor nulo é ortogonal a qualquer vetor.
\end{enumerate}

\end{frame}


\begin{frame}
\frametitle{Proposição}

Os vetores $\overrightarrow{u}$ e $\overrightarrow{v}$ são ortogonais se, e somente se, $$||\overrightarrow{u} + \overrightarrow{v}||^2 = ||\overrightarrow{u}||^2 + ||\overrightarrow{v}||^2$$

\pause

{\bf Definição: } Uma base $(\overrightarrow{e}_1,\overrightarrow{e}_2,\overrightarrow{e}_3)$ é {\bf ortonormal} se $e_1,e_2$ e $e_3$ são unitários e dois a dois ortogonais.

\pause

{\bf Proposição: } Seja $(\overrightarrow{e}_1,\overrightarrow{e}_2,\overrightarrow{e}_3)$ uma base ortonormal. Se $\overrightarrow{u} = \alpha\overrightarrow{e}_1 + \beta\overrightarrow{e}_2+\gamma\overrightarrow{e}_3$, então
$$||\overrightarrow{u}|| = \sqrt{\alpha^2 + \beta^2 + \gamma^2}$$

\end{frame}


\begin{frame}
\frametitle{Exemplo}

Seja $E$ uma base ortonormal e $\overrightarrow{u} = (2,-1,3)_E$. Calcule $||\overrightarrow{u}||$.

\end{frame}

\section{Mudança de Base}

\begin{frame}
\frametitle{Mudança de Base}

A mudança de base tem como objetivo facilitar as operações.
Nosso interesse é na seguinte situação:

Sejam $B = \{ u_1, \dots, u_n \}$ e $B' = \{ w_1, \dots, w_n \}$ duas bases ordenadas de um mesmo espaço vetorial $V$. Dado um vetor $v \in V$, podemos escrevê-lo como:
\begin{eqnarray}
v = x_1 u_1 + \dots + x_n u_n \\\label{eq00}
v = y_1 w_1 + \dots + y_n w_n 
\end{eqnarray}

\end{frame}

\begin{frame}
	\frametitle{Mudança de Base}
	
	Podemos relacionar as coordenadas de $v$ em relação à base $B$,
	$$[v]_B = \left[
	\begin{array}{c}
	x_1	\\
	\vdots	\\
	x_n \\
	\end{array}
	\right]$$
	e também podemos relacionar as coordenadas de $v$ em relação à base $B'$,
	$$[v]_{B'} = \left[
	\begin{array}{c}
	y_1	\\
	\vdots	\\
	y_n \\
	\end{array}
	\right]$$
\end{frame}

\begin{frame}
	\frametitle{Mudança de Base}
	
	Como $\{ u_1, \dots, u_n\}$ é base de $V$, podemos escrever os vetores $w_1$ como combinação linear dos $u_j$, isto é,
	\begin{eqnarray}
	\label{eq01}
	\begin{cases}
	w_1 = a_{11}u_1 + a_{21}u_2 + \dots + a_{n1}u_n \\
	w_2 = a_{12}u_1 + a_{22}u_2 + \dots + a_{n2}u_n \\
	\vdots \hspace{40pt} \vdots \hspace{40pt} \vdots \hspace{60pt} \vdots \\
	w_n = a_{1n}u_1 + a_{2n}u_2 + \dots + a_{nn}u_n \\
	\end{cases}
	\end{eqnarray}
	
\end{frame}

\begin{frame}
	\frametitle{Mudança de Base}
	
	Substituindo os valores de (\ref{eq01}) em (\ref{eq00}) tem-se
	\begin{eqnarray*}
		v & = & y_1 w_1 + \dots + y_n w_n \\
		& = & y_1 (a_{11}u_1 + \dots + a_{n1}u_n) + \dots + y_n (a_{1n}u_1 + \dots + a_{nn}u_n) \\
		& = & (a_{11}y_1 + \dots + a_{1n}y_n)u_1 + \dots + (a_{n1}y_1 + \dots + a_{nn}y_n)u_n
	\end{eqnarray*}

	Como $v = x_1 u_1 + \dots + x_n u_n$ e as coordenadas em relação a uma base são únicas, temos:
	
	\begin{eqnarray*}
		x_1 = a_{11}y_1 + a_{21}y_2 + \dots + a_{n1}y_n \\
		x_2 = a_{12}y_1 + a_{22}y_2 + \dots + a_{n2}y_n \\
		\vdots \hspace{40pt} \vdots \hspace{40pt} \vdots \hspace{60pt} \vdots \\
		x_n = a_{1n}y_1 + a_{2n}y_2 + \dots + a_{nn}y_n \\
	\end{eqnarray*}
	
\end{frame}

\begin{frame}
	\frametitle{Mudança de Base}
	
	Matricialmente, temos a seguinte equação:
	
	$$\left[\begin{array}{c}
	x_{1} \\
	\vdots \\
	x_{n}
	\end{array}\right]
	=
	\left[ \begin{array}{ccc}
	a_{11}  &  \ldots  &  a_{1n} \\
	\vdots  &  \ddots  &  \vdots \\
	a_{m1}  &  \cdots  &  a_{mn} \\
	\end{array}\right]
	\left[\begin{array}{c}
	y_{1} \\
	\vdots \\
	y_{n}
	\end{array}\right]$$
	
	Definindo,
	$$[I]_{B}^{B'} = \left[ \begin{array}{ccc}
	a_{11}  &  \ldots  &  a_{1n} \\
	\vdots  &  \ddots  &  \vdots \\
	a_{m1}  &  \cdots  &  a_{mn} \\
	\end{array}\right]$$
	
	Temos a seguinte relação:
	
	$$[v]_{B} = [I]_{B}^{B'}[v]_{B'}$$
	
	A matriz $[I]_{B}^{B'}$ é chamada \emph{matriz de mudança de base $B'$ para a base $B$}.
	
\end{frame}

\begin{frame}
	\frametitle{Mudança de Base}
	
	Compare $[I]_{B}^{B'}$ com (\ref{eq01}) e observe que esta matriz é obtida, colocando as coordenadas em relação a $B$ de $w_i$ na i-ésima coluna. Uma vez obtida $[I]_{B}^{B'}$ podemos encontrar as coordenadas de $v$ na base $B'$ (supostamente conhecidas).
	
\end{frame}

\begin{frame}
	\frametitle{Exemplo}
	
	{\bf Exemplo:} Sejam $B = \{ (2,-1), (3,4) \}$ e $B' = \{ (1,0), (0,1) \}$ bases de $\mathbb{R}^2$. Determine $[I]_{B}^{B'}$.
	
\end{frame}

\begin{frame}
	\frametitle{Exemplo - Resolução}
	
	Inicialmente, determinaremos os vetores de $B'$ como combinação linear dos vetores de $B$, assim:
	\begin{eqnarray*}
		w_1 = (1,0) & = & a_{11}(2,-1) + a_{21}(3,4) \\
		(1,0) & = & (2a_{11} + 3a_{21}, -a_{11} + 4a_{21})
	\end{eqnarray*}
	Determinando os coeficientes, segue que:
	$$a_{11} = \frac{4}{11} \text{ e } a_{21} = \frac{1}{11}$$
	\begin{eqnarray*}
		w_2 = (0,1) & = & a_{12}(2,-1) + a_{22}(3,4) \\
		(0,1) & = & (2a_{12} + 3a_{22}, -a_{12} + 4a_{22})
	\end{eqnarray*}
	Determinando os coeficientes, segue que:
	$$a_{12} = \frac{-3}{11} \text{ e } a_{22} = \frac{2}{11}$$
	
\end{frame}

\begin{frame}
	\frametitle{Exemplo - Resolução}
	
	Portanto,
	$$[I]_{B}^{B'} = \left[ \begin{array}{cc}
	a_{11}  &  a_{12}  \\
	a_{21}  &  a_{22} \\
	\end{array}\right]
	=
	\left[ \begin{array}{cc}
	\frac{4}{11}  &  \frac{-3}{11}  \\
	\frac{1}{11}  &  \frac{2}{11} \\
	\end{array}\right]$$
	
\end{frame}

\begin{frame}
	\frametitle{Exemplo - Resolução}
	
	Podemos utilizar o mesmo exemplo para determinar as coordenadas do vetor $v = (5,-8)$ na base $B$.
	\begin{eqnarray*}
		[(5,-8)]_{B} & = & [I]_{B}^{B'}[(5,-8)]_{B'} \\
		& = & \left[ \begin{array}{cc}
			\frac{4}{11}  &  \frac{-3}{11}  \\
			\frac{1}{11}  &  \frac{2}{11} \\
		\end{array}\right] 
		\left[\begin{array}{c}
			5 \\
			-8 \\
		\end{array}\right]\\
		& = & 
		\left[\begin{array}{c}
			4 \\
			-1 \\
		\end{array}\right]
	\end{eqnarray*}
	Ou seja,
	$$(5,-8) = 4(2,-1) -1(3,4)$$
	
\end{frame}

\begin{frame}
	\frametitle{Mudança de Base}
	
	O cálculo utilizando a matriz de mudança de base é operacionalmente vantajoso quando trabalhamos com mais vetores, pois não há necessidade de resolução de mais de um sistema de equação para cada vetor.
	
\end{frame}

\begin{frame}
	\frametitle{Mudança de Base}
	
	{\bf Proposição: } Toda matriz de mudança de base possui matriz inversa.

	\pause

	{\bf Proposição: } Se $E = (\overrightarrow{e}_1,\overrightarrow{e}_2,\overrightarrow{e}_3)$, $F = (\overrightarrow{f}_1,\overrightarrow{f}_2,\overrightarrow{f}_3)$ e $G = (\overrightarrow{g}_1,\overrightarrow{g}_2,\overrightarrow{g}_3)$ são bases, então $$M_{EF}M_{FG} = M_{EG}$$
	
	\pause
	
	{\bf Proposição: } A matriz de mudança de $F$ para $E$ é a matriz inversa da matriz de mudança de $E$ para $F$, isto é, $M_{FE} = M_{EF}^{-1}$
	
\end{frame}

\section{Produto Escalar}

\begin{frame}
	\frametitle{Medida Angular}
	
	Sejam $\overrightarrow{u}$ e $\overrightarrow{v}$ vetores não-nulos. Chama-se {\bf medida angular entre $\overrightarrow{u}$ e $\overrightarrow{v}$} a medida $\theta$, onde $0 \leq \theta \leq \pi$, do ângulo $P\hat{O}Q$, sendo $(O,P)$ e $(O,Q)$, respectivamente, representantes quaisquer de $\overrightarrow{u}$ e $\overrightarrow{v}$ \emph{com mesma origem}
	
	Notação: $\theta = ang(\overrightarrow{u},\overrightarrow{v})$, se necessário, especificando a unidade adotada (grau ou radiano).
	
	\pause
	
	Se $ang(\overrightarrow{u},\overrightarrow{v}) < 90^{\circ}$ então dizemos que é um \emph{ângulo agudo} formado por $\overrightarrow{u}$ e $\overrightarrow{v}$, analogamente, se $ang(\overrightarrow{u},\overrightarrow{v}) = 90^{\circ}$ então dizemos que é um \emph{ângulo reto} formado por $\overrightarrow{u}$ e $\overrightarrow{v}$.
\end{frame}


\begin{frame}
	\frametitle{Determinando $\cos(\theta)$ através de $\overrightarrow{u},\overrightarrow{v}$}
	Lousa
\end{frame}

\begin{frame}
	\frametitle{Definição}
	
	O {\bf produto escalar} entre dois vetores $\overrightarrow{u},\overrightarrow{v}$, indicado por $\overrightarrow{u}\cdot\overrightarrow{v}$, é o número real tal que:
	\begin{enumerate}
		\item se $\overrightarrow{u}$ ou $\overrightarrow{v}$ é nulo, $\overrightarrow{u}\cdot\overrightarrow{v} = 0$;
		\item se $\overrightarrow{u}$ e $\overrightarrow{v}$ não são nulos e $\theta$ é a medida angular entre eles, $\overrightarrow{u}\cdot\overrightarrow{v} = ||\overrightarrow{u}||||\overrightarrow{v}||\cos(\theta)$
	\end{enumerate}
\end{frame}
	
\begin{frame}
	\frametitle{Proposição}
	\begin{enumerate}
		\item Se $\overrightarrow{u}$ e $\overrightarrow{v}$ não são nulos e $\theta = ang(\overrightarrow{u},\overrightarrow{v})$, então
		$$\cos(\theta) = \frac{\overrightarrow{u}\cdot\overrightarrow{v}}{||\overrightarrow{u}||||\overrightarrow{v}||}$$
		\item Qualquer que seja o vetor $\overrightarrow{u}$, 
		$$||\overrightarrow{u}|| = \sqrt{\overrightarrow{u}\cdot\overrightarrow{u}}$$
		\item Quaisquer que sejam os vetores $\overrightarrow{u}$ e $\overrightarrow{v}$,
		$$\overrightarrow{u}\perp\overrightarrow{v} \Leftrightarrow \overrightarrow{u}\cdot\overrightarrow{v} = 0$$
	\end{enumerate}
\end{frame}

\begin{frame}
	\frametitle{Proposição}
	Se em relação a uma base ortonormal, $\overrightarrow{u} = (a_1,b_1,c_1)$ e $\overrightarrow{v} = (a_2,b_2,c_2)$, então
	$$\overrightarrow{u}\cdot\overrightarrow{v} = a_1a_2 + b_1b_2 + c_1c_2$$
	
	\pause
	
	{\bf Exemplo:} Em relação a uma base ortonormal, são dados $\overrightarrow{u} = (2,0,-3)$ e $\overrightarrow{v} = (1,1,1)$. Calcule a medida angular entre $\overrightarrow{u}$ e $\overrightarrow{v}$.
\end{frame}

\begin{frame}
	\frametitle{Propriedades}
	
	Quaisquer que sejam os vetores $\overrightarrow{u}, \overrightarrow{v}$ e $\overrightarrow{w}$ e qualquer que seja o número real $\lambda$, são válidas:
	\begin{enumerate}
		\item $\overrightarrow{u}\cdot(\overrightarrow{v}+\overrightarrow{w}) = \overrightarrow{u}\cdot\overrightarrow{v} + \overrightarrow{u}\cdot\overrightarrow{w}$
		\item $\overrightarrow{u}\cdot(\lambda\overrightarrow{v}) = (\lambda\overrightarrow{u})\cdot\overrightarrow{v} = \lambda(\overrightarrow{u}\cdot\overrightarrow{v})$
		\item $\overrightarrow{u}\cdot\overrightarrow{v} = \overrightarrow{v}\cdot\overrightarrow{u}$
		\item Se $\overrightarrow{u}\neq\overrightarrow{0}$, então $\overrightarrow{u}\cdot\overrightarrow{u} > 0$
	\end{enumerate}
	
\end{frame}

\begin{frame}
	\frametitle{Exemplo}
	
	As medidas angulares entre os vetores $\overrightarrow{u}$ e $\overrightarrow{v}$, $\overrightarrow{u}$ e $\overrightarrow{w}$ e $\overrightarrow{v}$ e $\overrightarrow{w}$ são, respectivamente, $30$, $45$ e $90$ graus. Mostre que $(\overrightarrow{u}, \overrightarrow{v}, \overrightarrow{w})$ é base.
\end{frame}

\section{Projeção Ortogonal}

\begin{frame}
	\frametitle{Introdução}


Sejam $\overrightarrow{u} = \overrightarrow{OA}$ e $\overrightarrow{v} = \overrightarrow{OB}$ vetores não-nulos, que formam um \emph{ângulo agudo} de medida $\theta$ radianos e $C$ é o pé da perpendicular, por $B$, à reta $OA$. O vetor $\overrightarrow{p} = \overrightarrow{OC}$ é a \emph{projeção ortogonal de $\overrightarrow{v}$ sobre $\overrightarrow{u}$}.

Tem-se que $\overrightarrow{p}$ é paralelo a $\overrightarrow{u}$ e, dentre todos os vetores paralelos a $\overrightarrow{u}$, parece ser $\overrightarrow{p}$ o único para o qual $\overrightarrow{v} - \overrightarrow{p}$, ou seja, $\overrightarrow{CB}$ é ortogonal a $\overrightarrow{u}$.

\pause

Uma outra forma de se observar é através da decomposição de $\overrightarrow{v}$ como soma de duas parcelas, $\overrightarrow{p}$ e $\overrightarrow{q}$, onde $\overrightarrow{p}$ é paralelo a $\overrightarrow{u}$ e $\overrightarrow{q}$ ortogonal a $\overrightarrow{u}$ $(\overrightarrow{q} = \overrightarrow{CB} = \overrightarrow{v} - \overrightarrow{p})$:
$$\overrightarrow{v} = \overrightarrow{p} + \overrightarrow{q}$$
$$\overrightarrow{p} // \overrightarrow{u}$$
$$\overrightarrow{q} \perp \overrightarrow{u}$$

\end{frame}

\begin{frame}
	\frametitle{Definição}
	
	Seja $\overrightarrow{u} \neq \overrightarrow{0}$. Dado $\overrightarrow{v}$ qualquer, o vetor $\overrightarrow{p}$ é chamado de {\bf projeção ortogonal de $\overrightarrow{v}$ sobre $\overrightarrow{u}$}, e indicado por $proj_{\overrightarrow{u}}\overrightarrow{v}$, se satisfaz as condições:
	\begin{enumerate}
		\item $\overrightarrow{p}//\overrightarrow{u}$
		\item $(\overrightarrow{v} - \overrightarrow{p})\perp\overrightarrow{u}$
	\end{enumerate}
	
\end{frame}


\begin{frame}
	\frametitle{Preposição}
	
	Seja $\overrightarrow{u}$ um vetor não-nulo. Qualquer que seja $\overrightarrow{v}$, existe e é única a projeção ortogonal de $\overrightarrow{v}$ sobre $\overrightarrow{u}$. Sua expressão em termos de $\overrightarrow{u}$ e $\overrightarrow{v}$ é:
	$$proj_{\overrightarrow{u}}\overrightarrow{v} = \frac{\overrightarrow{v}\cdot \overrightarrow{u}}{||\overrightarrow{u}||^2}\overrightarrow{u}$$
	e a expressão de sua norma,
	$$||proj_{\overrightarrow{u}}\overrightarrow{v}|| = \frac{|\overrightarrow{v}\cdot \overrightarrow{u}|}{||\overrightarrow{u}||}$$
	
	
\end{frame}

\begin{frame}
	\frametitle{Exemplo}
	Dada a base ortonormal $B = (\overrightarrow{i},\overrightarrow{j},\overrightarrow{k})$, sejam $\overrightarrow{u} = 2\overrightarrow{i}-2\overrightarrow{j}+\overrightarrow{k}$ e $\overrightarrow{v} = 3\overrightarrow{i} - 6\overrightarrow{j}$
	
	\begin{enumerate}
		\item Obtenha a projeção ortogonal de $\overrightarrow{v}$ sobre $\overrightarrow{u}$
		\item Determine $\overrightarrow{p}$ e $\overrightarrow{q}$ tais que $\overrightarrow{v} = \overrightarrow{p} + \overrightarrow{q}$, sendo $\overrightarrow{p}$ paralelo e $\overrightarrow{q}$ ortogonal a $\overrightarrow{u}$
	\end{enumerate}
\end{frame}

\begin{frame}
	\frametitle{Exemplo}
	
	Seja $B = (\overrightarrow{i},\overrightarrow{j},\overrightarrow{k})$ e $E = (\overrightarrow{u}, \overrightarrow{v}, \overrightarrow{w})$ base qualquer. Prove que $E$ é ortonormal se, e somente se, a matriz $M = M_{BE}$ satisfaz a igualdade $M^tM=I$, ou seja, sua inversa é igual a sua transposta.

	\end{frame}




\end{document}

