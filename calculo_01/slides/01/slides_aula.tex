\documentclass[hyperref={pdfpagelabels=false}]{beamer}
\usepackage{lmodern}
\usetheme{CambridgeUS}

\usepackage[english,brazilian]{babel}
\usepackage{multicol}
\usepackage{textcomp}
\usepackage[alf]{abntex2cite}
\usepackage[utf8]{inputenc}
\usepackage[T1]{fontenc}
\usepackage{graphicx} %Pacote de imagens
\graphicspath{{./figures/}}

\title{REVISÃO \\ CÁLCULO I}  
\author[Matheus Pimenta]{Matheus Pimenta} 
\institute[UEL]{\normalsize Universidade Estadual de Londrina \\
	Londrina
} 
\date{Fev. 2022} 
\begin{document}
	
\begin{frame}
\titlepage
\end{frame} 


%\begin{frame}
%\frametitle{Table of contents}
%\tableofcontents
%\end{frame} 


\section{Apresentação} 


\begin{frame}
\frametitle{Apresentação} 

Matheus Pimenta \pause

e-mail: matheus.pimenta@outlook.com ou omatheuspimenta@outlook.com \pause

Informações sobre a disciplina \pause

Dúvidas gerais


\end{frame}

\section{REVISÃO - FUNÇÕES}

\begin{frame}
\frametitle{Conjuntos Numéricos}
\begin{itemize}
 \item $\mathbb{N}$: O conjunto $\mathbb{N}$	dos números naturais é caracterizado pelos axiomas de Peano.

    \begin{enumerate}
        \item Todo número natural $n$ tem um sucessor $s(n)$, que ainda é um número natural, números diferentes têm sucessores diferentes. \pause
        \item Existe um único número natural $1$ que não é sucessor de nenhum outro. \pause
        \item Se um conjunto de números naturais contém o $1$ e contém também o sucessor de cada um de seus elementos, então esse conjunto contém todos os números naturais. \pause
    \end{enumerate}

\end{itemize}

\end{frame}


\begin{frame}
\frametitle{Conjuntos Numéricos}
No conjunto $\mathbb{N}$ dos números naturais são definidas duas operações fundamentais: adição e multiplicação, caracterizadas por:
\begin{multicols}{2}
\begin{itemize}
	\item $n+1 = s(n)$
	\item $n+s(m) = s(n+m)$
	\item $m \cdot 1 = m$
	\item $m \cdot s(n) = m \cdot n + m$
\end{itemize}
\end{multicols}

\textbf{Propriedades:}
\begin{enumerate}
	\item Associativa: $(m+n)+p = m+(n+p)$ \\ $m(np)=(mn)p$
	\item Distributiva: $m(n+p)=mn + mp$
	\item Comutativa: $m+n = n+m$ \\ $mn=nm$
	\item Lei do Corte: $n+m = p+m \implies n=p$ \\ $nm=pm \implies n=p$
\end{enumerate}

\end{frame}


\begin{frame}
\frametitle{Conjuntos Numéricos}
Outra propriedade importante de $\mathbb{N}$ é conhecida como {\it o Princípio da Boa Ordenação} que é todo subconjunto não-vazio $A \subseteq \mathbb{N}$ possui um menor elemento, isto é, um elemento $n_0 \in A$ tal que $n_o \leq n$ $\forall n \in A$.

\end{frame}

\begin{frame}
\frametitle{Conjuntos Numéricos}

\begin{itemize}
 \item $\mathbb{Z}$: \pause dos números inteiros é definido por: $$\mathbb{Z} = \{ 0, \pm 1, \pm 2, \pm 3, \dots \}$$. \pause
 \item $\mathbb{Q}$: \pause dos números racionais é formado pelas razões de inteiros: $$ \mathbb{Q} = \left\{ \frac{p}{q} ; p,q \in \mathbb{Z} \text{ e } q \neq 0 \right\}$$
\end{itemize}

\end{frame}


\begin{frame}
\frametitle{Conjuntos Numéricos}

\begin{itemize}
 \item $\mathbb{R}$: \pause dos números reais pode ser definido como o conjunto de todas as possíveis expansões decimais. Assim, um número real $a,d_{1}d_{2}d_{3}\dots$ pode ser representado por uma ``soma infinita'' de números racionais: $$a + \frac{d_1}{10} + \frac{d_2}{10^2} + \dots $$
\end{itemize}

\end{frame}


\begin{frame}
\frametitle{Conjuntos Numéricos}
\textbf{$\mathbb{R}$ é um corpo}, isto é, estão definidas em $\mathbb{R}$ duas operações: adição e multiplicação, que satisfazem as seguintes propriedades:
\begin{multicols}{2}
\begin{itemize}
	\item Associativa: $(x+y)+z = x+(y+z)$ \\ $(xy)z=x(yz)$ \pause
	\item Comutativa: $x+y=y+x$ \\ $xy=yx$ \\ \pause
	\item Distributiva: $x(y+z) = xy + xz$ \pause
	\item Elemento Neutro: \\ existem em $\mathbb{R}$ dois elementos distintos, $0$ e $1$, tais que: $x+0=x$ e $x \cdot 1= x$ para qualquer $x \in \mathbb{R}$
\end{itemize}
\end{multicols}

\end{frame}


\begin{frame}
\frametitle{Conjuntos Numéricos}
{\bf Inversos:} todo $x \in \mathbb{R}$ possui um {\it inverso aditivo} $(-x) \in \mathbb{R}$ tal que $x + (-x) = 0$ e se $x \neq 0$, existe também um {\it inverso multiplicativo} $x^{-1}$ tal que $x \cdot x^{-1} = 1$ \pause

A soma $x + (-y)$ é indicada por $x-y$ e é denominada {\bf diferença}. \pause

O produto $x \cdot y^{-1} $ é indicado por $\frac{x}{y}$ e é denominado {\bf quociente}. \pause

A divisão de $x$ por $y$ só faz sentido se $y \neq 0$. Podemos demonstrar outras propriedades como:

\begin{itemize}
	\item $xy=0 \implies x =0 \text{ ou } y=0$ \pause
	\item $x(-y)=(-x)y = -(xy)$ \pause
	\item $(-x)(-y)=xy$ \pause
	\item $x^2=y^2 \implies x = \pm y$ \pause
\end{itemize}

\end{frame}


\begin{frame}
\frametitle{Conjuntos Numéricos}
{\bf $\mathbb{R}$ é um corpo ordenado}, isto é, existe um subconjunto $\mathbb{R}_{+}^{*} \subset \mathbb{R}$ chamado o conjunto dos números reais positivos, que cumpre duas condições: \pause
\begin{itemize}
	\item [p1:] $x,y \in \mathbb{R}_{+}^{*} \implies x+y \in \mathbb{R}_{+}^{*}$ e $xy \in \mathbb{R}_{+}^{*}$ \pause
	\item [p2:] Dado $x \in \mathbb{R}$, $x=0$ ou $x \in \mathbb{R}_{+}^{*}$ ou $(-x) \in \mathbb{R}_{+}^{*}$ \pause
\end{itemize}

Escreve-se $x<y$ e diz-se que $x$ é menor que $y$ se $y-x \in \mathbb{R}_{+}^{*}$. \pause

De {\bf p2}, segue que, dados $x,y \in \mathbb{R}$, $y-x=0$ ou $y-x \in \mathbb{R}_{+}^{*}$ ou $x-y \in \mathbb{R}_{+}^{*}$. Ou seja, $x=y$, $x>y$ ou $y>x$. \pause

{\bf $\mathbb{R}$ é um corpo ordenado completo}, isto é, não possui espaços ``vazios'' na reta.

\end{frame}

\begin{frame}{Intervalos}

\textbf{Intervalos:} Um intervalo é um conjunto de números reais com a seguinte propriedade: dados dois números pertencentes ao intervalo, todos os números entre eles também pertencem ao intervalo. \pause

\begin{table}[h!]
	\centering
	\begin{tabular}{|c|c|c|c|}
		\hline
		&	Notação			& Descrição do conjunto						 & Representação Geométrica \\
		\hline
		A	& $(a,b)$		& $\{ x \in \mathbb{R} ; a< x <b \} $		 & 	\\
		\hline
		B	& $[a,b]$		& $\{ x \in \mathbb{R} ; a\leq x \leq b \} $ & \\
		\hline
		C	& $[a,b)$		& $\{ x \in \mathbb{R} ; a\leq  x <b \} $	 & \\	
		\hline
		D	& $(a,b]$		& $\{ x \in \mathbb{R} ; a< x \leq b \} $	 & \\
		\hline
		E	& $(a, \infty)$	& $\{ x \in \mathbb{R} ; a< x < \infty \} $  & \\
		\hline
		F	& $[a,\infty)$	& $\{ x \in \mathbb{R} ; a \leq x <\infty \}$& \\
		\hline
		G	& $(- \infty,b)$& $\{ x \in \mathbb{R} ; - \infty < x <b \} $& \\
		\hline
		H	& $(- \infty, b]$& $\{ x \in \mathbb{R} ; -\infty< x \leq b \} $	 & \\
		\hline
		I	& $(- \infty, \infty)$ & $\mathbb{R}$ &	\\
		\hline
		J	& $[a,a]$		& $\{a\}$									 & \\
		\hline
	\end{tabular}
\end{table}
 
\end{frame}

\begin{frame}{Desigualdades}
 
 \textbf{Desigualdades:} Como um corpo ordenado, $\mathbb{R}$ possui as seguintes propriedades:
\begin{enumerate}
	\item $a<b$ e $b<c$ $\implies a<c$; \pause
	\item $a<b$ $\implies a+c<b+c$ \\ $a<b$ $\implies a-c<b-c$ \pause
	\item $a<b$ e $c<d$ $\implies a+c<b+d$ \pause
	\item $a<b$ e $c>0$ $\implies ac<bc$ \\ $a<b$ e $c<0$ $\implies ac>bc$ \pause
	\item $a>0$ $\implies \frac{1}{a}>0$ \\ $a<0$ $\implies \frac{1}{a}<0$ \pause
	\item Se $a$ e $b$ são ambos positivos ou ambos negativos, então $a<b$ implica $\displaystyle \frac{1}{a} > \displaystyle \frac{1}{b}$. Se $a$ é negativo e $b$ positivo, então $\displaystyle \frac{1}{a} < \displaystyle \frac{1}{b}$
\end{enumerate}
 
\end{frame}

\begin{frame}{Valor Absoluto}
 
\textbf{Valor Absoluto:} O valor absoluto de um número $x$ é a distância de $x$ a $0$ na reta real, dada por:

$$|x| = 
\begin{cases}
x, \text{  se } x \geq 0 \\
-x, \text{ se } x \leq 0
\end{cases}
$$

$$|x| = \sqrt{x^{2}} $$

 
\end{frame}

\begin{frame}{Valor Absoluto}
 
 {\bf Algumas propriedades:}
\begin{enumerate}
	\item $|x| = a \Leftrightarrow x = \pm a$ \pause
	\item $|x| < a \Leftrightarrow -a < x < a$ \pause
	\item $|x| > a \Leftrightarrow x>a$ ou $x<a$ \pause
	\item $|-a| = |a|$ \pause
	\item $|ab|=|a|.|b|$ \pause 
	\item $|\frac{a}{b}| = \frac{|a|}{|b|}$ \pause
	\item $|a^{n}| = |a|^{n}$ \pause
	\item $|a+b| \leq |a| + |b|$ Desigualdade Triangular
\end{enumerate}
 
\end{frame}



\begin{frame}
\frametitle{Equações}

\begin{itemize}
 \item Equações de 1º Grau: \pause são expressões do tipo $ax+b=0$, com $a \neq 0$, onde $a$ e $b$ são coeficientes da equação e $x$ é a incógnita. \pause
 
A equação do primeiro grau pode não ter solução em $\mathbb{N}$ ou $\mathbb{Z}$, mas sempre possui uma solução em $\mathbb{Q}$ ou $\mathbb{R}$.

\begin{eqnarray*}
	&  ax + b & = 0 \\ \pause
	\implies & ax + b - b & = -b \\ \pause
	\implies & ax & = -b \\ \pause
	\implies & \displaystyle \frac{ax}{a} & = \displaystyle - \frac{b}{a}\\ \pause
	\implies & x & = \displaystyle - \frac{b}{a}
\end{eqnarray*}
 
 \end{itemize}

\end{frame}
 
 \begin{frame}
\frametitle{Equações}

\begin{itemize}
 \item Equações de 2º Grau: \pause são expressões do tipo $ax^{2} + bx + c = 0$, com $a \neq 0$, onde $a,b$ e $c$ são coeficientes e $x$ é a incógnita. Nem sempre possui raízes reais, mas possui uma ou duas raízes em $\mathbb{C}$. \pause
 
{\bf Fórmula Geral de Resolução da Equação de 2$^\circ$ Grau:}

\begin{equation}
x = \displaystyle \frac{-b \pm \displaystyle \sqrt{b^2 - 4ac}}{2ac}
\end{equation}
\end{itemize}

\end{frame}

\begin{frame}{Equações}
 A existência de raízes reais dependerá do valor de $\Delta = b^2 - 4ac$. Suponha, então que $\Delta \geq 0 $, assim:
$$x_1 = \displaystyle \frac{-b + \displaystyle \sqrt{b^2 - 4ac}}{2ac}$$
$$x_2 = \displaystyle \frac{-b - \displaystyle \sqrt{b^2 - 4ac}}{2ac}$$

\pause
Temos as {\bf Relações de Girard}:
$$\displaystyle \frac{b}{a} = -x_1 - x_2 \implies - \displaystyle \frac{b}{a} = x_1 + x_2$$
$$\displaystyle \frac{c}{a} = x_1 x_2$$
\end{frame}


\begin{frame}{Inequações}
 \textbf{Exemplo 01:}Resolva a desigualdade $x^{2} - 5x + 6 \leq 0$ \pause
 
 Fatorando o lado esquerdo da inequação: $a(x-x_1)(x-x_2) \leq 0$ 
$$(x-2)(x-3) \leq 0$$

Avaliando o sinal nos intervalos:

Se $x \in (-\infty,2) \implies x < 2$, logo:
$$x-2<0 \land x-3 < 0 $$

Se $x \in (2,3) \implies 2<x<3$, logo:
$$x-2 > 0 \land x-3 < 3$$

Se $x \in (3, \infty) \implies x > 3$, logo:
$$x - 2 > \land x - 3>0$$

\end{frame}

\begin{frame}{Inequações}
 Assim temos a tabela:
\begin{table}[h]
	\centering
	\begin{tabular}{|c|c|c|c|}
		\hline
		Intervalo \ Fator	&	$x-2$	&	$x-3$	&	$(x-2)(x-3)$ \\
		\hline
		$x<2$				&	-		&	-		&	+	\\
		\hline
		$2<x<3$				&	+		&	-		&	-	\\
		\hline
		$x>3$				&	+		&	+		&	+	\\
		\hline
	\end{tabular}
\end{table}
\end{frame}



\begin{frame}
\frametitle{Divisão de Polinômios}

Seja $f(x) = a_0x^n+a_1x^{n-1}+\dots+a_{n-1}x+a_n$ e $g(x)=x-a$. 

Se $f(a)=0$, então existe $q(x)=q_0x^{n-1}+q_1x^{n-2}+\dots+q_{n-2}x+q_{n-1}$ tal que $q(x).g(x)=f(x)$.


\end{frame}

\begin{frame}
\frametitle{Funções Reais de uma Variável Real}

Sejam $D$ e $Y$ conjuntos não-degenerados. Uma função $f:D \rightarrow Y$ é uma regra que associa, a cada elemento $x \in D$, um único elemento $f(x) \in Y$. \pause

Função: \pause É uma relação em que todos os elementos $x$ do conjunto $A$, se relacionam uma {\bf única} vez com os elementos $y$ do conjunto $B$, ou seja, $f: A \rightarrow B$, ou ainda, $(f): \mathbb{R} \rightarrow \mathbb{R}$, ou simplesmente $(f): X\rightarrow Y = f(x)$, na prática escrevemos $y = f(x)$. \pause

O conjunto $D$ é denominado domínio da função. O conjunto $Y$ é denominado contra-domínio. A imagem da função $f$ é o conjunto formado por todos os pontos $y \in Y$ tais que existe um $x_0 \in D$ com $f(x_0) = y$. Note que $Im(f) \subset Y$, mas não necessariamente a imagem possui todos os elementos do contra-domínio.
\end{frame}

\begin{frame}{Funções definidas por partes}
 Uma função pode ser descrita por fórmulas diferentes a partes diferentes de seu domínio. Por exemplo, a função valor absoluto. \pause

$$|x| = 
\begin{cases}
x, \text{  se } x \geq 0 \\
-x, \text{ se } x < 0
\end{cases}
$$
\end{frame}

\begin{frame}{Simetrias}
 Uma função $y=f(x)$ é dita:
\begin{itemize}
	\item par, se $f(-x) = f(x)$, $\forall x \in D$; \\ Simetria em relação ao eixo $y$.
	\item ímpar, se $f(-x) = -f(x)$, $\forall x \in D$. \\ Simetria em relação à origem.
\end{itemize}
\end{frame}


\begin{frame}
\frametitle{Funções Monótonas}

Uma função $f:D \rightarrow \mathbb{R}$, com $D \subset \mathbb{R}$, chama-se:
\begin{itemize}
	\item não-decrescente, se $x<y \Rightarrow f(x) \leq f(y)$;
	\item não-crescente, se $x<y \Rightarrow f(x) \geq f(y)$;
	\item crescente, se $x<y \Rightarrow f(x) < f(y)$;
	\item decrescente, se $x<y \Rightarrow f(x) > f(y)$.
\end{itemize}

\end{frame}

\begin{frame}
\frametitle{Composição de Funções}
Sejam $A$, $B$ e $C$ conjuntos e sejam as funções $f:A \rightarrow B$ e $g: B \rightarrow C$. A função $ g \circ f : A \rightarrow C$ é definida por:
$$(g \circ f)(x) = g(f(x))$$

Note que $g \circ f$ costuma ser diferente de $ f \circ g$.

\end{frame}

\begin{frame}{Funções}
 Uma função $f:D \rightarrow Y$ é dita {\it injetora} se $f(x_1) \neq f(x_2)$ sempre que $x_1 \neq x_2$ em $D$. Em outras palavras, $f$ é injetora se $f(x_1) = f(x_2)$ implica em $x_1 = x_2$. Para cada $y = f(x)$ só há um $x$ que satisfaça.
 \pause
 
 Uma função $f:D \rightarrow Y$ é dita {\it sobrejetora} se, para cada $y \in Y$, existe $x \in D$ tal que $f(x) = y$. Ou seja, $f$ é sobrejetora se $Im(f) = Y$. \pause

Observe que se restringirmos o contra-domínio à imagem, ou seja, tomarmos $f: D \rightarrow Im(f)$ pela mesma lei de formação, teremos uma função sobrejetora.
\end{frame}


\begin{frame}{Funções}
Uma função injetora e sobrejetora é dita {\it bijetora}. \pause

Se $f: D \rightarrow Y$ é bijetora, então, para cada $y \in Y$, existe um único $x \in D$ tal que $f(x)=y$. \pause

Assim, podemos definir $f^{-1}:Y \rightarrow D$ por 
$$f^{-1}(y) = x \text{ se } f(x) = y$$ \pause

Temos que:

$f \circ f^{-1}(y) = f(f^{-1}(y)) = f(x) = y$ 

$f^{-1} \circ f(x) = f^{-1}(f(x)) = f^{-1}(y) = x $
\end{frame}


\begin{frame}
\frametitle{Principais Funções Elementares}

\begin{itemize}
 \item Função Polinomial: \pause $$f(x) = a_0+a_1x+a_2x^2+\dots+a_nx^n$$ \pause
 \item Função Racional: \pause $$f(x) = \frac{p(x)}{q(x)},$$ onde $p$ e $q$ são funções polinomiais. O domínio de $f$ é $\{x \in \mathbb{R}; q(x) \neq 0 \}$ \pause
 \item Funções Trigonométricas: \pause $\sin (\alpha) = \displaystyle \frac{b}{a}$ e $\cos(\alpha) = \displaystyle \frac{c}{a}$
\end{itemize}

\end{frame}


\begin{frame}
\frametitle{Principais Funções Elementares}

\begin{itemize}
 \item Função Exponencial: \pause A função exponencial de base $a$ é dada por $f(x) = a^{x}$. Se $x = \frac{p}{q}$ é um número racional, então:
$$a^{\frac{p}{q}} = \sqrt[q]{a^p} = (\sqrt[q]{a})^{p}$$ \pause

Consideramos sempre $a>0$ e $a \neq 1$. \pause

\textbf{Algumas propriedades:}
\begin{enumerate}
	\item $a^{x} \cdot a^{y} = a^{x+y}$; \pause
	\item $\displaystyle \frac{a^{x}}{a^{y}} = a^{x-y}$; \pause
	\item $(a^{x})^{y} = a^{xy}$ \pause
\end{enumerate}

\end{itemize}

\end{frame}

\begin{frame}{Principais Funções Elementares}
 \begin{itemize}
   \item Função Logarítmica: \pause A função logarítmica de base $a$ é a inversa da função exponencial de base $a$, ou seja:
$$\log_{a}x = y \Leftrightarrow a^{y} = x$$ \pause

Seu domínio é $(0, \infty)$ e sua imagem é $(- \infty, \infty)$ \pause

Algumas bases recebem notação especial:
$$\log_{a}x = \log(x)$$
$$\log_{e}x = \ln(x)$$
 \end{itemize}

\end{frame}

\end{document}

