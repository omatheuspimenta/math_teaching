\documentclass[oneside,a4paper,12pt]{article}
\usepackage[english,brazilian]{babel}
\usepackage{multicol}
\usepackage{textcomp}
\usepackage[alf]{abntex2cite}
\usepackage[utf8]{inputenc}
\usepackage[T1]{fontenc}
\usepackage{amsmath,amssymb,exscale}
\usepackage[top=20mm, bottom=20mm, left=20mm, right=20mm]{geometry}%margens cima, baixo, esquerda direita
\usepackage{framed}
\usepackage{booktabs} %Pacote para deixar tabelas mais bonitas.
\usepackage{color} %Pacote de Cores
\usepackage{hyperref} %Pacotes para Hiperlinks
\usepackage{graphicx} %Pacote de imagens
\graphicspath{{./Figuras/}}%Direciona as imagens para uma pasta chamada "Figuras" (uso isso para organizar. Uma vez que todas as imagens vao ficar em uma pasta isolada)    
\definecolor{shadecolor}{rgb}{0.8,0.8,0.8}

%FAZ EDICOES AQUI (somente no conteudo que esta entre entre as ultimas  chaves de cada linha!!!)
\newcommand{\universidade}{Universidade Estadual de Londrina}
\newcommand{\centro}{Centro de Ciências Exatas}
\newcommand{\departamento}{Departamento Matemática}
\newcommand{\curso}{Física}
\newcommand{\professores}{Matheus Pimenta}
\newcommand{\disciplina}{Cálculo I - 1MAT096}
%\newcommand{\tema}{Lista 01}
%\newcommand{\turma}{MA31G}
%\newcommand{\data}{Março de 2019}%{\today}
%\newcommand{\tempodeaula}{30 minutos}
%\newcommand{\prerequisitos}{Matrizes, Transformações Lineares e Bases}
%ATE AQUI !!!	

\begin{document}
	\pagestyle{empty}
	
	\begin{center}
		\includegraphics[width=\linewidth/2]{logo.jpg}%LOGOTIPO DA INSTITUICAO
	 	\vspace{2pt} 	
		
		\universidade
		\par
		\centro
		\par
		\departamento
		\par
	%	Curso de \curso
		\par
		\vspace{12pt}
		\LARGE \textbf{Lista 01 - REVISÃO}
		
	\end{center}
	
	\vspace{12pt}
	
	\begin{tabular}{ |l|p{12cm}| }
		
		\hline
		\multicolumn{2}{|c|}{\textbf{Dados de Identificação}} \\
		\hline
		Professor:         &    \professores           \\
		\hline
		Disciplina:        &    \disciplina          \\
		\hline
	%	Tema:              &    \tema                \\
	%	\hline
	%	Pré-requisito	:  &    \prerequisitos         \\
	%	\hline
		Aluno:             &                   \\
	%	\hline
	%	Data:              &    \data                \\
	%	\hline
	%	Duração da aula:   &    \tempodeaula         \\
		\hline
		
	\end{tabular}
	\vspace{6pt}
	
	
	\begin{snugshade}
	\end{snugshade}

\begin{enumerate}

	\item Resolva as equações:
	\begin{multicols}{2}
	\begin{enumerate}
		\item $\frac{x-1}{3} + \frac{x+1}{2} = \frac{x}{2} + 1$
		\item $ \frac{2x+5}{x^{2}+x}- \frac{3}{x} = \frac{2}{x+1}$
		\item $ x(x-1)(x^{2}-5x+6) = 0$
		\item $ \frac{x+2}{2}+\frac{2}{x-2}= - \frac{1}{2}$
	\end{enumerate}
	\end{multicols}

	\item Resolva as inequações:
	\begin{multicols}{2}
	\begin{enumerate}
		\item $4x - (x-4) > 20 - 3(x+2) $
		\item $ \frac{x+1}{2} - \frac{2x +1}{6} > 3$
		\item $ 1 + 6y \leq 2y + \frac{9-y}{3}$
		\item $ (4x - 8)(3x+1) > 0 $
		\item $ x^{2} - 8x + 12 < 0 $
		\item $ \frac{2x - 3}{x-1} \leq 0 $
	\end{enumerate}
	\end{multicols}

	\item Resolva as equações modulares:
	\begin{multicols}{2}
	\begin{enumerate}
		\item $|x-3| = 7$
		\item $|x-2| = 3x -8 $
		\item $|7x-1| = |2x+5| $
		\item $|x+2| + |x-3| = 13$
	\end{enumerate}
	\end{multicols}

	\item Resolva as inequações modulares:
	\begin{multicols}{2}
	\begin{enumerate}
		\item $|7x -2| < 4 $
		\item $ \frac{|7 - 2x|}{|4+x|} \leq 2$
		\item $|5-6x| \geq 9$
		\item $|-2x + 4| < x+1$
	\end{enumerate}
	\end{multicols}

	\item Determine as seguintes somas algébricas:
	\begin{multicols}{2}
	\begin{enumerate}
		\item $\frac{3x}{2y} + \frac{x}{4y} - \frac{7x}{10y} $
		\item $ \frac{3b}{a} + \frac{5b}{2a} + \frac{7b}{4a}$
		\item $ \frac{a-x}{a}+ \frac{a - x}{x}$
		\item $ \frac{1+x}{1-x} + \frac{1-x}{1+x} $
		\item $ \frac{x}{x+1} - \frac{1}{x-1} + \frac{2}{x^{2}+1}$
		\item $ \frac{x-5y}{x+y} + \frac{5y^{2}}{xy+y^{2}}$
	\end{enumerate}
	\end{multicols}

	\item Determine os seguintes produtos:
	\begin{multicols}{2}
	\begin{enumerate}
		\item $\frac{a+x}{10} \cdot \frac{5}{ax+x^{2}}$
		\item $\frac{a^{2} - b^{2}}{2ab} \cdot \frac{2b}{a-b}$
		\item $\frac{9x}{a^{2} -4} \cdot \frac{a+2}{3x}$
		\item $\frac{a^{2} + 2ax + x^{2}}{m^{2} - n^{2}} \cdot \frac{m-n}{a+x}$
		\item $\frac{5a + 5}{x^{4} + x^{2}} \cdot \frac{x^{3}}{a+1}$
		\item $\frac{x^{2} - 7x + 12}{x^{2} -9} \cdot \frac{x^{3} - 6x^{2} + 9x}{x^{3} - 4x^{2}} $
	\end{enumerate}
	\end{multicols}

	\item Determine o quociente e o resto da divisão de $f(x) = 2x^{3} + x^{2} - x + 2$ por $g(x) = x^{2} + 3x + 1$.
	
	\item Ache $Q(x)$ e $R(x)$ na divisão de $f(x) = x^{4} - 1$ por $g(x) = x + 1$.
	
	\item Determine os seguintes quocientes:
	\begin{multicols}{2}
	\begin{enumerate}
		\item $ \frac{ \frac{a}{x+1}}{ \frac{a^{2}}{x^{2} -1}}$
		\item $ \frac{ \frac{x^{2}-25}{xy}}{ \frac{2x + 10}{x} }$
		\item $ \frac{ \frac{a+x}{10}}{ \frac{5}{ax+x^{2}}}$
		\item $ \frac{ \frac{2}{x+2} - 3}{ \frac{4}{x} - x}$
	\end{enumerate}
	\end{multicols}

	\item Aplicando a definição de logaritmo, calcule o valor das expressões:
	\begin{multicols}{4}
	\begin{enumerate}
		\item $ \log_{8}64=x$
		\item $ \log_{\frac{1}{3}}27=x$
		\item $ \log_{x}32=5$
		\item $ \log_{3}x=4$
	\end{enumerate}
	\end{multicols}

	\item Resolva as equações:
	\begin{multicols}{2}
	\begin{enumerate}
		\item $27^{x} = 243$
		\item $2^{3x+1} = 8^{x-3}$
		\item $\log_{3}(x-5) = 3$
		\item $\log_{2}(x+3) + \log_{2}(x-4) = 3$
	\end{enumerate}
	\end{multicols}

	\item Informe em qual quadrante esta localizado os seguintes valores e determine os valores perpendiculares a eles no círculo trigonométrico.
	\begin{multicols}{2}
	\begin{enumerate}
		\item $\frac{4 \pi }{3}$
		\item $\frac{\pi}{4}$
	\end{enumerate}
	\end{multicols}

	\item Construa retângulos no círculo trigonométrico a partir da medida inicial dada:
	\begin{multicols}{2}
	\begin{enumerate}
		\item $115^{\circ}$
		\item $300^{\circ}$
	\end{enumerate}
	\end{multicols}

	\item Sendo $f(x) = \frac{3 - 2x}{x+1}$, prove que $[f^{-1}]^{-1}=f(x)$
	
	\item Sendo $f(x) = 3^{x}$ e $g(x) = x+4$, determine:
	\begin{multicols}{4}
	\begin{enumerate}
		\item $f \circ g$
		\item $g \circ f$
		\item $f \circ f$
		\item $g \circ g$
	\end{enumerate}
	\end{multicols}

	\item Prove que $\frac{f(x+y) + f(x-y)}{\frac{1}{2}[f(xy) + f(-xy)]}=f(x)$ onde $f(x) = x+2$.

	\item Sendo $f(x) = x^{2} + x -6$, calcule:
	\begin{multicols}{5}
	\begin{enumerate}
		\item $f(1)$
		\item $f(-1)$
		\item $f(0)$
		\item $f(-\frac{4}{5})$
		\item $\frac{f(2)+f(-2)}{3f(4)}$
	\end{enumerate}
	\end{multicols}

\end{enumerate}


	
\end{document}
