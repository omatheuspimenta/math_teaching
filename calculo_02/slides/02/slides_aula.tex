\documentclass[hyperref={pdfpagelabels=false}]{beamer}
\usepackage{lmodern}
\usetheme{CambridgeUS}

\usepackage[brazilian]{babel}
\usepackage{multicol}
\usepackage{textcomp}
\usepackage[alf]{abntex2cite}
\usepackage[utf8]{inputenc}
\usepackage[T1]{fontenc}
\usepackage{graphicx} %Pacote de imagens
\graphicspath{{./figures/}}
\usepackage{amsthm}

\usepackage[normalem]{ulem}


\title{Sequências e Séries}  
\author[Matheus Pimenta]{Matheus Pimenta} 
\institute[UEL]{\normalsize Universidade Estadual de Londrina \\
	Londrina
} 
\date{Fev. 2022} 
\begin{document}

	
\begin{frame}
\titlepage
\end{frame} 


%\begin{frame}
%\frametitle{Table of contents}
%\tableofcontents
%\end{frame} 


\section{Apresentação} 


\begin{frame}
\frametitle{Apresentação} 

Matheus Pimenta \pause

e-mail: matheus.pimenta@outlook.com ou omatheuspimenta@outlook.com \pause

Informações sobre a disciplina \pause

Dúvidas gerais


\end{frame}

\section{Sequências}

\begin{frame}
\frametitle{Sequências}

\begin{definition}[Sequências]
    Uma sequência é uma função $$a: \mathbb{N} \rightarrow \mathbb{R},$$ onde $\mathbb{N} = \{1,2,3,\dots\}$
\end{definition} \pause

\begin{example}
 Considere $a(n) = n + 1$, $n \in \mathbb{N}$ \pause
 
 \begin{align*}
  a(1) &= 2 \\ \pause
  a(2) &= 3 \\ \pause
  \vdots \\ 
  a(5) &= 6
 \end{align*}

\end{example}


\end{frame}

\begin{frame}{Sequências}
 Os valores de $a(n)$ são chamados de \emph{termos da sequência}. \\
 O termo $a(n)$ é chamado de $n-$ésimo termo da sequência. \\ \pause
 
 {\bf Notação:} Denotaremos uma sequência $a: \mathbb{N} \rightarrow \mathbb{R}$ por: $$(a_n) = (a(n))$$ \\
 
 Ou seja, a sequência $(a_n)$ tem como $n-$ésimo termo $a_n$. \pause
 
 \begin{example}[Considere a sequência:]
  \begin{equation*}
   a_n = \frac{n}{n+1}\text{,} n \in \mathbb{N} 
  \end{equation*} \pause
  Os primeiros $4$ elementos da sequência $a_n$ são: \pause
  
  {\bf Preencher!}
 \end{example}

\end{frame}

\begin{frame}{Sequências}
 {\bf Questões:} \pause
 \begin{enumerate}
  \item Qual é o domínio de uma sequência? \pause
  \item Podemos restringir uma sequência? \pause Quais exemplos? \pause Conseguimos representar uma sequência de outras maneiras? \pause
  \item Quais exemplos de outras sequências podemos ter neste momento? \pause
  \item Como podemos representar graficamente uma sequência?
 \end{enumerate}

\end{frame}

\begin{frame}{Sequências - Exemplos}
 \begin{example}
  \begin{equation*}
   a_n = \frac{1}{n}
  \end{equation*} \pause
  $a_1 = 1$, \pause$a_2 = \displaystyle \frac{1}{2}$, \pause$a_3 = \displaystyle \frac{1}{3}$, \pause$\dots$, \pause$a_n = \displaystyle \frac{1}{n}$
 \end{example}
 
   \begin{example}
  \begin{equation*}
   a_n = \frac{(-1)^n}{n}
  \end{equation*} \pause
  $a_1 = -1$, \pause $a_2 = \displaystyle \frac{1}{2}$, \pause $a_3 = -\displaystyle \frac{1}{3}$
  
 \end{example}
 
\end{frame}

\begin{frame}{Sequências}
 Podemos listar os termos da sequência através do uso de $\{ \}$. \pause 
 
 \begin{example}
  \begin{align*}
   \{a_n\} &= \{ \sqrt{1},\sqrt{2}, \sqrt{3}, \dots, \sqrt{n}, \dots \} \\ 
   \{c_n\} &= \{1, -1, 1, -1, \dots, \dots, (-1)^{n+1}, \dots \} \\
   \{b_n\} &= \left\{ 1, -\frac{1}{2}, \frac{1}{3}, -\frac{1}{4}, \dots, (-1)^{n+1}\frac{1}{n}, \dots, \right\}
  \end{align*}
 \end{example}

\end{frame}


\begin{frame}{Sequências}
 \begin{definition}[Igualdade de sequências]
  Dizemos que duas sequências $(a_n)$ e $(b_n)$ são iguais se e somente se, $a_n = b_n$, $\forall ~n \in \mathbb{N}$
 \end{definition} \pause
 
 \begin{example}
  Analise as sequências: $a_n = (-1)^n$ e $b_n = (-1)^{n+1}$
 \end{example} \pause
 
 Note que:
 \begin{align*}
  (a_n) &= (-1,1,-1,1,-1,\dots,(-1)^n) 
 \end{align*}\pause
  \begin{align*}
  (b_n) &= (1,-1,1,-1,1,\dots, (-1)^{n+1})
 \end{align*}\pause
Logo, as sequências \pause {\bf não são iguais}. Assim: $$(a_n) \neq (b_n)$$

\end{frame}

\section{Limite de uma Sequência}
\begin{frame}{Limite de uma Sequência}
 
 {\bf Motivação:} Tome $a_n = \displaystyle \frac{1}{n}$ \pause
 {\bf Graficamente:} \pause
 
 \begin{definition}{Limite de uma Sequência}
  Dizemos que uma sequência $a_n$ converge para $L$ se para dado $\epsilon > 0$, $\exists N > 0$ tal que $n > N$, $|a_n - L| < \epsilon$
 \end{definition} \pause
 {\bf Em outras palavras:} \\ \pause
 $\forall \epsilon >0, ~\exists ~N > 0 ~;~ n > N \implies L-\epsilon < a_n < L + \epsilon$ \\ \pause
 
 {\bf Notação:} \\ \pause Se $(a_n)$ converge para $L$ diremos que $L$ é o \emph{limite da sequência} $(a_n)$. \pause
 Escrevemos: $\displaystyle\lim_{n\to\infty}a_n = L$, ou $\lim a_n = L$, ou simplismente, $a_n \to L$
 
\end{frame}

\begin{frame}{Limite de uma Sequência}
 {\bf Observações:} \pause
 \begin{itemize}
  \item A noção de ``sequência convergente'' não depende apenas da sequência $(a_n)$, mas também do espaço $X$. \pause
  \begin{example}
   A sequência $\left(\displaystyle \frac{1}{n} \right)$ converge a $0$ em $\mathbb{R}$ ou $\mathbb{R}_+$, mas não em $\mathbb{R}_{+}^{*}$. \pause Pois $0 \notin \mathbb{R}_{+}^{*}$. \pause Dessa forma é conveniente dizer que ``$a_n$ converge (ou não) em $X$''.
  \end{example} \pause
  \item Podemos reescrever a definição de limite de sequência de outras maneiras: \pause
  \begin{itemize}
   \item Dizemos que $(p_n)$ converge se existe um ponto $p \in X$ tal que:
   $$\forall ~ \epsilon > 0~ \exists ~ n_0; d(p_n,p)< \epsilon ~ \text{ sempre que } n \geq n_0$$
  \end{itemize}

 \end{itemize}

\end{frame}

\begin{frame}{Limite de uma Sequência}
 Em particular, considerando $X = \mathbb{R}$. \pause Uma sequência $(x_n)$ em $\mathbb{R}$ converge se existe um número real $x$ tal que: \pause
 
 $$ \forall ~ \epsilon > 0~ \exists ~ n_0; |x_n - x| < \epsilon ~\text{ sempre que } n \geq n_0$$ \pause 
 
Uma outra forma de obter a definição de limite de sequência é a seguinte: \pause

\begin{definition}
 A sequência $\{a_n\}$ converge para o número $L$ se para todo númro positivo $\epsilon$ corresponder um número inteiro $N$, de forma que para todo $n$,
 $$n > N ~\implies ~|a_n - L| < \epsilon$$ \pause
 
 Se nenhum número $L$ existir, dizemos que $\{a_n\}$ \emph{diverge}.
\end{definition}


\end{frame}

\begin{frame}{Limite de uma Sequência}
 {\bf Exemplo 01:} Utilizando a definição mostre que $\displaystyle \lim_{n \to \infty}\frac{1}{n}$ \\\pause
 {\bf Solução:} \pause Queremos mostrar que $\forall \epsilon > 0, \exists ~N>0;$ \\ \pause
 \begin{align*}
  \text{se } n > N &\implies |a_n - L| < \epsilon &\iff \\ 
  \text{se } n > N &\implies \left| \frac{1}{n} - 0 \right| < \epsilon &\iff (x \in \mathbb{N} \implies |x| \in \mathbb{N}) \\
  \text{se } n > N &\implies \frac{1}{n} < \epsilon &\iff \\
  \text{se } n > N &\implies n > \frac{1}{\epsilon}
 \end{align*} \pause
 Tome $N \geq \displaystyle \frac{1}{\epsilon}$, então: \pause se $n > N \geq \displaystyle \frac{1}{\epsilon}$ implica em $n > \displaystyle \frac{1}{\epsilon}$ \pause
 
 \end{frame}

 \begin{frame}{Limite de uma Sequência}
 
 Logo, para dado $\epsilon > 0$, tome $N \geq \displaystyle \frac{1}{\epsilon}$, assim se $n > \displaystyle \frac{1}{n}$, então $\left| \displaystyle \frac{1}{n} - 0 \right| < \epsilon. \qed$
 
 \end{frame}

\begin{frame}{Limite de uma Sequência}
 {\bf Exemplo 02:} Utilizando a definição mostre que $\displaystyle \lim_{n \to \infty} \frac{n}{2n + 1} = \frac{1}{2}$ \\ \pause
 {\bf Solução:} \pause Queremos mostrar que para $\forall \epsilon > 0, \exists ~N = (N(\epsilon))>0;$ \\ \pause
 \begin{align*}
  \text{se } n > N &\implies \left| \frac{n}{2n + 1} - \frac{1}{2} \right| < \epsilon &\iff \\
  \text{se } n > N &\implies \left| \frac{2n - 2n - 1}{2(2n+1)} \right| < \epsilon &\iff \\
  \text{se } n > N &\implies \left| \frac{-1}{2(2n+1)}\right| < \epsilon &\iff \\
  \text{se } n > N &\implies \frac{-1}{2(2n+1)} < \epsilon &\iff \\
  \text{se } n > N &\implies 2(2n + 1) > \frac{1}{\epsilon} &\iff \\
 \end{align*}

\end{frame}

\begin{frame}{Limite de uma Sequência}
 {\bf Exemplo 02:}
 
 \begin{align*}
  \text{se } n > N &\implies 2n + 1 > \frac{1}{2\epsilon} &\iff \\
  \text{se } n > N &\implies n > \frac{1}{4\epsilon} - \frac{1}{2} = \frac{2-4\epsilon}{8\epsilon} &\iff \\
  \text{se } n > N &\implies n > \frac{1-2\epsilon}{4\epsilon}
 \end{align*} \pause
 
 Logo, tomando $N > \displaystyle \frac{1 - 2 \epsilon}{4\epsilon}$, obtemos que se $n > N$, então $$\left|\frac{n}{2n+1} - \frac{1}{2}\right| < \epsilon$$ $\qed$

\end{frame}

\begin{frame}{Limite de uma Sequência}
 {\bf Exemplo 03:} Mostre que a sequência $a_n = (-1)^{n+1}$ diverge. \\ \pause
 {\bf Solução:} \pause Suponha que a sequência convirja para algum número $L$. Escolha $\epsilon = \displaystyle \frac{1}{2}$, da definição de limite, temos que todos os termos $a_n$ da sequência com índice $n$ maior que $N$ deve se localizar a menos de $\epsilon = \displaystyle \frac{1}{2}$ de $L$. \pause Uma vez que o número $1$ aparece repetidamente como termo sim, termo não (sequência alternada), devemos ter o número $1$ localizado a uma distância a menos de $\epsilon = \displaystyle \frac{1}{2}$ de $L$. 
  
\end{frame}

\begin{frame}{Limite de uma Sequência}
{\bf Exemplo 03:} 
 
 Segue que $| L - 1| < \displaystyle \frac{1}{2}$, de forma equivalente,  $\displaystyle \frac{1}{2} < L < \displaystyle \frac{3}{2}$. De forma análoga o número $(-1)$ aparece repetidamente na sequência com índice arbitrariamente alto. E de forma análoga, segue que: $| L - (-1)| < \displaystyle \frac{1}{2}$, de forma equivalente,  $\displaystyle -\frac{1}{2} < L < \displaystyle -\frac{3}{2}$. \pause Contudo o número $L$ não pode estar contido em dois intervalos disjuntos, isto é, os intervalos $\left( \displaystyle \frac{1}{2}, \displaystyle \frac{3}{2} \right)$ e $\left( \displaystyle -\frac{1}{2}, \displaystyle -\frac{3}{2} \right)$ não possuem sobreposição.\pause Dessa forma, não existe tal limite $L$ e portanto a sequência diverge. $\qed$
 
\end{frame}

\begin{frame}{Sequências}
 \begin{definition}
  A sequência $a_n$ \emph{diverge ao infinito} se para cada número $M$ houver um número inteiro $N$, tal que para todo $n$ maior que $N$, $a_n > M$. Se essa condição for verdadeira, então: 
  \begin{equation*}
   \lim_{n \to \infty} a_n = \infty ~\text{ ou } ~ a_n \to \infty
  \end{equation*} \pause
  
  De maneira análoga a definição ocorre quando a sequência possui como limite $-\infty$.
 \end{definition}\pause
 
 \begin{definition}[Imagem de Sequência]
  A \emph{imagem} de uma sequência $(p_n)_{n \in \mathbb{N}}$ é o conjunto de seus pontos, isto é, $\{ x_n ; n \in \mathbb{N}\}$. 
 \end{definition} \pause
 
\end{frame}

\begin{frame}{Sequências}
 \begin{definition}[Sequência limitada]
  Dizemos que uma sequência é \emph{limitada superiormente} (ou inferiormente) se existir $M \in X$ $(m \in X)$ tal que:
  $$p_n < M,~\forall n \in \mathbb{N}$$
  $$m < p_n ,~\forall n \in \mathbb{N}$$
  
  Dizemos que uma sequência é limitada se existem $q \in X$ e $M > 0$ tal que $d(p_n,q)<M$ para todo $n$. Em outras palavras, se existirem $m,M \in X$ tal que $m<p_n<M,~\forall n \in \mathbb{N}$.
 \end{definition} \pause
 
 \begin{theorem}
  Se $\displaystyle \lim_{x \to \infty}f(x) = L$ e $f$ estiver definida em $X$, então $\displaystyle \lim_{n \to \infty}f(n) = L$. Em particular, $X = \mathbb{N}$.
 \end{theorem} \pause
 
 
  {\bf Demonstração:} Utilizar a definição de limite. $\qed$
 

\end{frame}

\begin{frame}{Limite de Sequências}
 Utilizando o Teorema anterior é possível utilizar as propriedades de limites que já foram estudadas. \pause
 
 \begin{theorem}
  Sejam $\{a_n\}$ e $\{b_n\}$ sequências de números reais, e sejam $A$ e $B$ números reais. As seguintes regras de aplicam se $\displaystyle \lim_{n \to \infty} a_n = A$ e $\displaystyle \lim_{n \to \infty} b_n = B$
  \begin{itemize}
  \item $\displaystyle \lim_{n \to \infty} (a_n \pm b_n) = A \pm B$ \pause
  \item $\displaystyle \lim_{n \to \infty} (k \cdot b_n) = k\cdot B~$ \text{, para todo $k$ constante} \pause
  \item $\displaystyle \lim_{n \to \infty} (a_n \cdot b_n) = A \cdot B$ \pause
  \item $\displaystyle \lim_{n \to \infty} \frac{a_n}{b_n} = \frac{A}{B} ~$ \text{, se } $B \neq 0$
 \end{itemize}
 \end{theorem}

\end{frame}

\begin{frame}{Sequências}
\begin{theorem}
 Seja $(a_n)$ uma sequência em um espaço métrico $X$.
 \begin{itemize}
  \item Se $a_n \to A$ e $a_n \to A'$, então $A = A'$ \pause {\bf Unicidade do Limite} \pause
  \item Se $(a_n)$ converge então $(a_n)$ é limitada. 
 \end{itemize}

\end{theorem}
 
\end{frame}

\begin{frame}{Sequências}
 {\bf OBSERVAÇÃO:} Cuidado ao aplicar as regras de limite em sequências, pois em alguns casos mesmo sequências divergentes podem, dentro de hipóteses serem convergentes quando aplicamos alguma operação. \pause
 
 Explico: Suponha as sequências $\{a_n\} = \{1,2,3,4,\dots,...\}$ e $\{b_n\} = \{-1,-2,-3,\dots,...\}$. Ambas as sequências sozinhas são sequências divergentes, contudo quando realizamos $\displaystyle \lim_{n \to \infty}\{a_n + b_n\}$ esta soma converge para $0$.  \pause
 
 Uma consequência é que toda sequência divergente quando multiplicada por algum escalar qualquer não nulo, também será uma sequência divergente. {\bf PENSAR SOBRE!}
\end{frame}

\begin{frame}{Sequências}
  \begin{theorem}[do confronto para sequências]
   Sejam $\{a_n\}$, $\{b_n\}$ e $\{c_n\}$ sequências de números reais. Se $a_n \leq b_n \leq c_n$ for verdade para todo $n$ além de algum índice $N$, e se $\displaystyle \lim_{n \to \infty} a_n = \displaystyle \lim_{n \to \infty} c_n = L$, então $\displaystyle \lim_{n \to \infty} b_n = L$.
  \end{theorem}\pause
  \begin{corollary}
   Se $|b_n| \leq c_n$ e $c_n \to 0$, então $b_n \to 0$.
  \end{corollary}\pause
  
  {\bf Ideia:} Já que $-c_n \leq b_n \leq c_n$, então segue o resultado.


\end{frame}

\begin{frame}{Sequências}
 \begin{theorem}[da função contínua para sequências]
  Seja $\{a_n\}$ uma sequência de números reais. Se $a_n \to L$ e se $f$ for uma função que é contínua em $L$ e definida em todo $a_n$, então $f(a_n) \to f(L)$.
 \end{theorem} \pause
 
 \begin{example}
  Mostre que $\displaystyle \lim_{n \to \infty} a_n = 1$, onde $a_n = \sqrt{\displaystyle \frac{n+1}{n}}$.
 \end{example} \pause
 
 {\bf Solução:} \pause Sabemos que $\displaystyle \frac{n+1}{n} \to 1$. Tomando $f(x) = \sqrt{x}$ e $L=1$. Aplicando o Teorema anterior tem-se que: $\sqrt{\displaystyle \frac{n+1}{n}} \to \sqrt{1} = 1$ $\qed$.


\end{frame}

\begin{frame}{Sequências}
 \begin{theorem}
  \begin{itemize}
   \item $a_n \to a$ se e somente se $a_{n_k} \to a$ para toda subsequência $(a_{n_k})$. \pause
   \item $a_n \to a$ se e somente se toda subsequência $(a_{n_k})$ possui uma subsubsequência $(a_{n_{k_i}})$ que converge a $a$, isto é, $\displaystyle \lim_{i \to \infty} a_{n_{k_i}} = a$
  \end{itemize} \pause
  
  \begin{definition}
   Uma sequência $(a_n)$ num espaço métrico $X$ é dita \emph{de Cauchy} se:
   \begin{equation*}
    \forall \epsilon >0,~\exists~ n_0; d(a_n,a_m)< \epsilon~\text{sempre que }n,m\geq n_0
   \end{equation*}

  \end{definition} \pause
  
  \begin{theorem}[Cauchy]
   Em um espaço métrico, toda sequência convergente é de Cauchy. Em outras palavras, se $(a_n)$ é convergente então para dado $\epsilon>0$ existe $N>0$ tal que $n,m>N$, então $|a_n - a_m| < \epsilon$, no caso do conjunto dos números reais.
  \end{theorem}

 \end{theorem}

\end{frame}

\begin{frame}{Definição recursiva}
 Uma sequência pode ser definida de maneira \emph{recursiva}, dessa forma a sequência fornece:
 \begin{itemize}
  \item O(s) valor(es) do termo inicial ou termos iniciais, e \pause
  \item Uma regra, chamada \emph{fórmula de recursão}, para o cálculo de qualquer termo posterior a partir dos termos que o precederem.
 \end{itemize} \pause
 
 \begin{example}
  \begin{itemize}
   \item As sentenças $a_1 = 1$ e $a_n = a_{n-1} + 1$ para $n > 1$ definem a sequência $1,2,3,\dots,n,\dots..$ de números inteiros positivos. \pause Para $a_1 = 1, a_2 = a_1 + 1 = 2, \dots$ \pause
   \item As sentenças $a_1 = 1$ e $a_n = n\cdot a_{n-1}$ para $n>1$ definem a sequência $1,2,6,24,\dots,n!,\dots$ de fatoriais. \pause Com $a_1 = 1$, temos $a_2 = 2 \cdot a_1 = 2$, $a_3 = 3 \cdot a_2 = 6$ e assim por diante. \pause
   \item Outros examplos são \emph{números de Fibonacci} e o método de Newton.
  \end{itemize}
 \end{example}
\end{frame}

\begin{frame}{Sequências monotônicas}
 \begin{definition}
  Dizemos que uma sequência $\{a_n\}$ é: \pause
  \begin{itemize}
   \item {\bf Decrescente:} se $a_{n+1} \leq a_n,~\forall n \in \mathbb{N}$, ou ainda $\displaystyle \frac{a_{n+1}}{a_n} \leq 1,~\forall n \in \mathbb{N}$ \pause
   \item {\bf Estritamente Decrescente:} se $a_{n+1} < a_n,~\forall n \in \mathbb{N}$, ou ainda $\displaystyle \frac{a_{n+1}}{a_n} < 1,~\forall n \in \mathbb{N}$ \pause
   \item {\bf Crescente:} se $a_{n+1} \geq a_n,~\forall n \in \mathbb{N}$, ou ainda $\displaystyle \frac{a_{n+1}}{a_n} \geq 1,~\forall n \in \mathbb{N}$ \pause
   \item {\bf Estritamente Crescente:} se $a_{n+1} > a_n,~\forall n \in \mathbb{N}$, ou ainda $\displaystyle \frac{a_{n+1}}{a_n} > 1,~\forall n \in \mathbb{N}$
  \end{itemize} \pause
  
  Se $\{a_n\}$ é uma sequência satisfazendo qualquer um dos itens anteriores dizemos que $\{a_n\}$ é monótona.

 \end{definition}

\end{frame}

\begin{frame}{Sequências monotônicas}
 Voltaremos a discutir sequências limitadas.
 \begin{definition}
  Um número $c \in \mathbb{R}$ é chamado de \emph{limitante inferior} para a sequência $\{a_n\}$ se $$c \leq a_n, \forall n \in \mathbb{N}$$
  
  Um número $D \in \mathbb{R}$ é chamado de \emph{limitante superior} para $\{a_n\}$ se $$a_n \leq D, \forall n \in \mathbb{N}$$
 \end{definition} \pause

  \begin{example}
   Para a sequência $\left\{\displaystyle\frac{1}{n}\right\}$ qualquer $c \leq 0$ é limitante inferior de $\left\{\displaystyle\frac{1}{n}\right\}$ e qualquer $D \geq 1$ é limitante superior de $\left\{\displaystyle\frac{1}{n}\right\}$.
 \end{example}
\end{frame}

\begin{frame}{Sequências monotônicas}
\begin{theorem}
 Se uma sequência $\{a_n\}$ é limitada e monótona, então a sequência converge.
\end{theorem} \pause

{\bf OBSERVAÇÃO:} O Teorema não afirma que sequências convergentes são monótonas. \pause 

{\bf EXEMPLO:} A sequência $\left\{\displaystyle\frac{(-1)^{n+1}}{n}\right\}$ converge e é limitada, mas não é monótona, uma vez que ela altera entre valores positivos e negativos à medida que tende a $0$.

\end{frame}

\begin{frame}{Exemplos}
 {\bf Exemplo 01:} Mostre que a sequência $\left\{\frac{2^n}{n!}\right\}$ converge. \pause
 
 {\bf Solução:} Inicialmente vamos determinar a monotocidade da sequência:
 \begin{equation*}
  \frac{2^n}{n!} > \frac{2^{n+1}}{(n+1)!} \iff 1 > \frac{2}{n+1}~,n > 1
 \end{equation*}
Isto ocorre pois: $2^{n+1} = 2^n \cdot 2$ e $(n+1)! = (n+1)\cdot n!$ \\{\it VERIFICAR AS CONTAS!}
\\$\therefore a_n > a_{n+1},~n>1$. Logo a sequência é estritamente decrescente a partir de $n =2$. \\ \pause
Agora note que, $\forall c \leq 0$ e então $$\frac{2^n}{n!} > c,~\forall n \geq 1$$
Dessa forma, como a sequência é estritamente decrescente e limitada inferiormente, segue que $\left\{\frac{2^n}{n!}\right\}$ converge. $\qed$
\end{frame}


\end{document}

