\documentclass[hyperref={pdfpagelabels=false}]{beamer}
\usepackage{lmodern}
\usetheme{CambridgeUS}

\usepackage[brazilian]{babel}
\usepackage{multicol}
\usepackage{textcomp}
\usepackage[alf]{abntex2cite}
\usepackage[utf8]{inputenc}
\usepackage[T1]{fontenc}
\usepackage{graphicx} %Pacote de imagens
\graphicspath{{./figures/}}
\usepackage{amsthm}

\usepackage[normalem]{ulem}


\title{Sequências e Séries}  
\author[Matheus Pimenta]{Matheus Pimenta} 
\institute[UEL]{\normalsize Universidade Estadual de Londrina \\
	Londrina
} 
\date{Fev. 2022} 
\begin{document}

	
\begin{frame}
\titlepage
\end{frame} 


%\begin{frame}
%\frametitle{Table of contents}
%\tableofcontents
%\end{frame} 

\section{Séries Infinitas}

\begin{frame}{Séries Infinitas}

{\bf MOTIVAÇÃO:} \pause
\begin{align*}
 2 &= 1 + 1 \\
 2 &= 1 + \frac{1}{2} + \frac{1}{2} \\
 2 &= 1 + \frac{1}{2} + \frac{1}{4} + \frac{1}{4} \\
 \vdots \\
 2 &= 1 + \frac{1}{2} + \frac{1}{4} + \frac{1}{4} + \dots + \frac{1}{2^{n-1}} + \dots
\end{align*}

\end{frame}

\begin{frame}{Séries Infinitas}
 \begin{definition}[Séries Infinitas]
  Seja $\{U_n\}$ uma sequência. Defina $S_1 = u_1$ e $S_n = u_1 + u_2 + \dots + u_n$. Diremos que $\{S_n\}$ é uma \emph{série infinita} (ou simplismente, série). \\
  {\bf NOTAÇÃO:} $\displaystyle \sum_{n=1}^{\infty}U_n$ para representar  $\{S_n\}$. \pause
  \begin{itemize}
   \item Os números $u_1,u_2,\dots,u_n,\dots$ são chamados de termos da série $\{S_n\}$. \pause
   \item Os números $S_1, S_2, S_3, \dots$ são chamados de somas parciais da série.
  \end{itemize}
 \end{definition}
 \pause

 {\bf OBS.:} 
 \begin{align*}
  S_n &= u_1 + u_2 + \dots + u_{n-1} + u_n \\
  S_{n-1} &= u_1 + u_2 + \dots + u_{n-1} \\
  S_n - S_{n-1} &= u_n, n \in \mathbb{N}
 \end{align*}
\end{frame}

\begin{frame}{Séries Infinitas}
 {\bf EXEMPLO 01:} Considere $\displaystyle \sum_{n=1}^{\infty} \frac{1}{2^{n-1}}$. A sequência $\{S_n\}$ é determinada por: \pause
 
 \begin{align*}
  S_1 &= 1 \\
  S_2 &= u_1 + u_2 = 1 + \frac{1}{2} = \frac{3}{2} \\
  S_3 &= u_1 + u_2 + u_3 = S_2 + u_3 = \frac{3}{2} + \frac{1}{4} = \frac{7}{4} \\
  \vdots \\
  S_n &= 1 + \frac{1}{2} + \dots + \frac{1}{2^{n-1}}
 \end{align*}

\end{frame}

\begin{frame}{Séries Infinitas}
 {\bf EXEMPLO 02:} Considere a série telescópica $\displaystyle \sum_{n=1}^{\infty} \frac{1}{n(n+1)}$. Determine $S_n$. \pause
 
 Note que: $\displaystyle \frac{1}{n(n+1)} = \displaystyle \frac{1}{n} - \displaystyle \frac{1}{n+1}, \forall n \geq 1$. Dessa forma: \pause
 \begin{align*}
  S_1 &= 1 - \frac{1}{2} = \frac{1}{2}\\
  S_2 &= \frac{1}{2} - \frac{1}{3} = \frac{3-2}{6} = \frac{1}{6}\\
  S_3 &= \frac{1}{3} - \frac{1}{4} = \frac{4-3}{12} = \frac{1}{12}\\
  S_4 &= \frac{1}{4} - \frac{1}{5} = \frac{5-4}{20} = \frac{1}{20}\\
  \vdots   
 \end{align*}

\end{frame}

\begin{frame}{Séries Infinitas}
 {\bf EXEMPLO 02:}
 Ou ainda: \pause
 \begin{align*}
  S_n &= u_1 + u_2 + u_3 + \dots + u_n \\
  &= \left(1 - \frac{1}{2} \right) + \left(\frac{1}{2} - \frac{1}{3} \right) + \left(\frac{1}{3} - \frac{1}{4} \right) + .. + \left(\frac{1}{n-1} - \frac{1}{n} \right) + \left(\frac{1}{n} - \frac{1}{n+1} \right) \\
  &\text{Simplificando} \\
  &= 1 - \frac{1}{n+1}
 \end{align*} \pause

 $\therefore S_n = 1 - \displaystyle \frac{1}{n+1}, \forall n \geq 1$
\end{frame}

\begin{frame}{Séries Infinitas}
 {\bf EXEMPLO 03:}
 Seja $|r|<1$. Calcule $S_n$ para $\displaystyle \sum_{n=1}^{\infty}r^{n-1}$ (série geométrica)
 \pause
 
 Temos que $S_n = u_1 + u_2 + u_3 + \dots + u_n$, logo: 
 
 \begin{equation}
 \label{serie_geo1}
  S_n = 1 + r + r^2 + r^3 + \dots + r^{n-1}
 \end{equation}

 Pode-se multiplicar \ref{serie_geo1} por $r$, e dessa forma:
 
 \begin{equation}
  \label{serie_geo2}
  rS_n = r + r^2 + r^3 + r^4 + \dots + r^n
 \end{equation}

 Realizando a subtração de \ref{serie_geo1} - \ref{serie_geo2} segue:
 
 \begin{align*}
  S_n - rS_n = 1 - r^n \iff S_n(1-r) = 1 - r^n \iff S_n = \frac{1-r^n}{1-r},\forall n \geq 1
 \end{align*}

\end{frame}

\begin{frame}{Séries Infinitas}
 {\bf EXEMPLO 03:}

 Como $|r|<1$, então:
 \begin{align*}
  \lim_{n\to \infty} S_n &= \lim_{n \to \infty} \frac{1-r^n}{1-r}\\
  &= \frac{1}{1-r}
 \end{align*}
Já que $\displaystyle \lim_{n\to \infty} r^n = 0$

$\therefore \displaystyle \lim_{n\to \infty} S_n = \frac{1}{1-r}$
 
\end{frame}

\begin{frame}{Séries Infinitas}
 \begin{definition}
  Seja $\displaystyle \sum_{n=1}^{\infty}u_n$ uma série e $\{s_n\}$ a sequência de suas somas parciais. Se $\displaystyle \lim_{n\to \infty}S_n = S$ ($S$ é definido como soma da série), então dizemos que a \emph{série converge}. Caso contrário, ou seja, se não existe $\displaystyle \lim_{n\to\infty}S_n$ então dizemos que a \emph{série diverge}.
 \end{definition} \pause
 
 {\bf OBS. 01:} A série telescópica $\displaystyle \sum_{n=1}^{\infty}\frac{1}{n(n+1)} = 1$, pois $\displaystyle \lim_{n\to \infty}S_n = \lim_{n \to \infty} 1-\frac{1}{1+n} = 1$
 
 {\bf OBS. 02:} A série geométrica (como apresentada anteriormente) é convergente, pois $\displaystyle \lim_{n\to \infty}S_n = \lim_{n \to \infty} \frac{1-r^n}{1-r} = \frac{1}{1-r}$ 

\end{frame}

\begin{frame}{Séries Infinitas}
    \begin{theorem}
        Se $\displaystyle \sum_{n=1}^{\infty}u_n$ é convergente, então $\displaystyle \lim_{n\to \infty}u_n = 0$
    \end{theorem} \pause
    
{\bf OBSERVAÇÃO:} O teorema anterior não diz que $\displaystyle \sum_{n=1}^{\infty}u_n$ converge se $a_n \to 0$. É possível para uma série divergir quando $u_n \to 0$. \pause

{\bf OBSERVAÇÃO:} NEM TODA SEQUÊNCIA $\{u_n\}$ TAL QUE $u_n \to 0$ É DO TIPO $\displaystyle \sum u_n$ CONVERGENTE. \pause

{\bf EXEMPLO???????????????????????????????????????????????????????????????????????????????????????????????????????????????????????} \pause

TEMOS SIM! E é bem famoso.

\end{frame}

\begin{frame}{Séries Infinitas}
    {\bf EXEMPLO:} A série $\displaystyle \sum_{n=1}^{\infty} \frac{1}{n}$ diverge. Esta série é chamada de \emph{Série Harmônica}. \pause
    
    Fazendo $(s_n)$, a sequência de somas parciais de $\displaystyle \frac{1}{n}$, isto é:
    \begin{align*}
        s_1 &= 1 \\
        s_2 &= a_1 + a_2 = 1 + \frac{1}{2} \\
        s_3 &= a_1 + a_2 + a_3 = 1 + \frac{1}{2} + \frac{1}{3} \\
        s_4 &= a_1 + a_2 + a_3 + a_4 = 1 + \frac{1}{2} + \frac{1}{3} + \frac{1}{4} \\
        \vdots \\
        s_8 &= a_1 + \dots + a_8 = 1 + \frac{1}{2} + \frac{1}{3} + \frac{1}{4} + \frac{1}{5} + \frac{1}{6} + \frac{1}{7} + \frac{1}{8} \\
    \end{align*}

\end{frame}


\begin{frame}{Séries Infinitas}
    Podemos analisar a sequência auxiliar como $$s_k^{aux} = 1 + \frac{(k-1)}{2}$$ e mais que isso, $s_k^{aux}$ é monótona crescente. \pause
    
    Note que:
    \begin{align*}
        \lim_{k\to \infty}s_k^{aux} &= \lim_{k\to \infty}\left[1 + \frac{(k-1)}{2}\right] \\ &= \infty
    \end{align*}

    Como $s_n > s_k^{aux}$ então $s_n \to \infty$, por também ser uma sequência monótona crescente. \pause
    
    $$\therefore \displaystyle \sum_{n=1}^{\infty} \frac{1}{n} \text{ é divergente}.$$
    
\end{frame}

\begin{frame}{Séries Infinitas}
    Do Teorema anterior é possível determinar o primeiro teste. \pause
    
    {\bf TESTE DO n-ÉSIMO TERMO PARA DIVERGÊNCIA:} 
    
    $\displaystyle \sum_{n=1}^{\infty}u_n$ é divergente se $\displaystyle \lim_{n\to \infty}u_n$ não existe ou é diferente de zero.
\end{frame}

\begin{frame}{Propriedades de Séries Convergentes}
    \begin{theorem}
        Se $\displaystyle \sum_{n=1}^{\infty}a_n = A$ e $\displaystyle \sum_{n=1}^{\infty}b_n =B$ são séries convergentes, então:
        \begin{itemize}
            \item {\bf Regra da Soma:} $\displaystyle \sum_{n=1}^{\infty}(a_n + b_n) = \displaystyle \sum_{n=1}^{\infty}a_n + \displaystyle \sum_{n=1}^{\infty}b_n = A + B$ \pause
            \item {\bf Regra da Diferença:} $\displaystyle \sum_{n=1}^{\infty}(a_n - b_n) = \displaystyle \sum_{n=1}^{\infty}a_n - \displaystyle \sum_{n=1}^{\infty}b_n = A - B$ \pause
            \item {\bf Regra da Multiplicação por constante:} $\displaystyle \sum_{n=1}^{\infty}ka_n = \displaystyle k \sum_{n=1}^{\infty}a_n = kA$ (qualquer  número $k$)
        \end{itemize} 

    \end{theorem}

\end{frame}

\begin{frame}{Propriedades de Séries Convergentes}
    \begin{corollary}
        \begin{itemize}
            \item Todo múltiplo constante diferente de zero de uma série divergente diverge. \pause
            \item Se $\displaystyle \sum_{n=1}^{\infty}a_n$ converge e $\displaystyle \sum_{n=1}^{\infty}b_n$ diverge, então tanto $\displaystyle \sum_{n=1}^{\infty}(a_n + b_n)$ quanto $\displaystyle \sum_{n=1}^{\infty}(a_n - b_n)$ divergem.
        \end{itemize}
    \end{corollary} \pause
    
{\bf OBSERVAÇÃO:} $\displaystyle \sum_{n=1}^{\infty}(a_n + b_n)$ pode convergir quando tanto $\displaystyle \sum_{n=1}^{\infty}a_n$ quanto $\displaystyle \sum_{n=1}^{\infty}b_n$ divergem. \pause

{\bf EXEMPLO?????????:}
    
\end{frame}

\begin{frame}{Propriedades de Séries Convergentes}
    {\bf Exemplo:}
    
    $\displaystyle \sum_{n=1}^{\infty}a_n = 1 + 1 + 1 + \dots $ e $\displaystyle \sum_{n=1}^{\infty}b_n = (-1) + (-1) + (-1) + \dots$ divergem, enquanto $\displaystyle \sum_{n=1}^{\infty}(a_n + b_n) = 0 + 0 + 0 + \dots $ converge para 0.
\end{frame}

\begin{frame}{Propriedades de Séries Convergentes}
    \begin{theorem}
        Se $\displaystyle \sum_{n=1}^{\infty}a_n$ converge e $\{s_n\}$ é a sequência das somas parciais de $\displaystyle \sum_{n=1}^{\infty}a_n$, então, para dado $\epsilon > 0$, existe $N>0$ tal que se $m,n > N$ então $|s_n - s_m| < \epsilon$
    \end{theorem} \pause
    
    \begin{theorem}
        Sejam $\displaystyle \sum_{n=1}^{\infty}a_n$ e $\displaystyle \sum_{n=1}^{\infty}b_n$ séries que diferem somente de uma quantidade ffinita de termos, isto é, existe $N>0$ tal que $a_n = b_n,~ \forall n>N$. Então ambas convergem ou ambas divergem.
    \end{theorem}

\end{frame}


\begin{frame}{Séries Infinitas}
    \begin{itemize}
        \item Adicionando ou retirando termos; \pause
        \item Reindexação; \pause
    \end{itemize}
    Vimos um exemplo quando apresentada a Série Geométrica.
\end{frame}

\section{Séries Infinitas de Termos Positivos}

\begin{frame}{Séries Infinitas}
    Se todos os termos de uma série for positivos temos que $\{s_n\}$, sequência das somas parciais, é monótina crescente. \pause
    
    Logo, para que uma série $\displaystyle \sum_{n=1}^{\infty}a_n$ de termos positivos seja convergente basta que $\{ s_n\}$ seja limitada superiormente.
\end{frame}

\begin{frame}{Séries Infinitas}
    \begin{corollary}
        Uma série $\displaystyle \sum_{n=1}^{\infty}a_n$ de termos não negativos converge se, e somente se, suas somas parciais são limitadas superiormente.
    \end{corollary}

\end{frame}

\section{Teste da Integral}

\begin{frame}{Séries Infinitas}
    {\bf MOTIVAÇÃO:}
    
    Estudaremos o comportamento da série $\displaystyle \sum_{n=1}^{\infty}\frac{1}{n^2}$. \pause
    
    Para determinarmos a convergência de $\displaystyle \sum_{n=1}^{\infty}\frac{1}{n^2}$ utilizamos a comparação com a seguinte integral $\displaystyle \int_1^{\infty}\left(\frac{1}{x^2}\right)dx$. Para a comparação, pensaremos nos termos da série como valores da função $f(x) = \displaystyle \frac{1}{x^2}$ e interpretamos esses valores cmo as áreas de retângulos sob a curva $y = \displaystyle \frac{1}{x^2}$. \pause
    
    {\bf FIGURA}
\end{frame}

\begin{frame}{Séries Infinitas}
    Da figura temos que:
    \begin{align*}
        s_n &= \frac{1}{1^2} + \frac{1}{2^2} + \frac{1}{3^2} + \frac{1}{4^2} + \dots + \frac{1}{n^2} \\
        &= f(1) + f(2) + f(3) + \dots + f(n) \\
        &< f(1) + \int_1^{n}\frac{1}{x^2}dx \\
        &< 1 + \int_1^{\infty}\frac{1}{x^2}dx \\
        &< 1 + 1 = 2
    \end{align*}

    {\bf OBS.:} $\displaystyle \int_1^n\frac{1}{x^2}dx < \int^{\infty} \frac{1}{x^2}dx$
    
    {\bf OBS.:} $\displaystyle \int_1^{\infty}\frac{1}{x^2}dx = 1$
\end{frame}

\begin{frame}{Séries Infinitas}
    Então, as somas parciais de $\displaystyle \sum_{n=1}^{\infty}\frac{1}{n^2}$ são limitadas superiormente (por 2) e a série converge. \pause
    
    A soma da série $\displaystyle \sum_{n=1}^{\infty}\frac{1}{n^2}$ é conhecida por ser $\displaystyle \frac{\pi^2}{6} \approx 1,64493$ \pause
    
    {\bf OBS.:} A série e a integral não precisam ter o mesmo valor no caso convergente. Conforme o exemplo anterior, a série $\displaystyle \sum_{n=1}^{\infty}\frac{1}{n^2} = \displaystyle \frac{\pi^2}{6}$ e a integral $\displaystyle \int_1^{\infty}\frac{1}{x^2}dx = 1$
\end{frame}

\begin{frame}{Teste da Integral}
    \begin{theorem}[Teste da Integral]
        Seja $\{a_n\}$ uma sequência de termos positivos. Suponha que $a_n = f(n)$, onde $f$ é uma função continua, positiva e decrescente de $x$ para todo $x \geq N$ (sendo $N$ um inteiro positivo). Então, tanto a série $\displaystyle \sum_{n=N}^{\infty}a_n$ quanto a integral $\displaystyle \int_N^{\infty}f(x)dx$ simultaneamente convergem ou divergem.
    \end{theorem}

\end{frame}

\section{Testes de Comparação}

\end{document}

