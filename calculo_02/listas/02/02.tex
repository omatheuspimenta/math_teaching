\documentclass[oneside,a4paper,12pt]{article}
\usepackage[english,brazilian]{babel}
\usepackage{multicol}
\usepackage{textcomp}
\usepackage[alf]{abntex2cite}
\usepackage[utf8]{inputenc}
\usepackage[T1]{fontenc}
\usepackage{amsmath,amssymb,exscale}
\usepackage[top=20mm, bottom=20mm, left=20mm, right=20mm]{geometry}%margens cima, baixo, esquerda direita
\usepackage{framed}
\usepackage{booktabs} %Pacote para deixar tabelas mais bonitas.
\usepackage{color} %Pacote de Cores
\usepackage{hyperref} %Pacotes para Hiperlinks
\usepackage{graphicx} %Pacote de imagens
\graphicspath{{./Figuras/}}%Direciona as imagens para uma pasta chamada "Figuras" (uso isso para organizar. Uma vez que todas as imagens vao ficar em uma pasta isolada)    
\definecolor{shadecolor}{rgb}{0.8,0.8,0.8}

\newcommand{\sen}{{\rm sen}}

%FAZ EDICOES AQUI (somente no conteudo que esta entre entre as ultimas  chaves de cada linha!!!)
\newcommand{\universidade}{Universidade Estadual de Londrina}
\newcommand{\centro}{Centro de Ciências Exatas}
\newcommand{\departamento}{Departamento de Matemática}
\newcommand{\curso}{Ciência da Computação}
\newcommand{\professores}{Matheus Pimenta}
\newcommand{\disciplina}{Cálculo Diferencial e Integral II}
%\newcommand{\tema}{Lista 01}
%\newcommand{\turma}{MA31G}
%\newcommand{\data}{Março de 2019}%{\today}
%\newcommand{\tempodeaula}{30 minutos}
%\newcommand{\prerequisitos}{Matrizes, Transformações Lineares e Bases}
%ATE AQUI !!!	

\begin{document}
	\pagestyle{empty}
	
	\begin{center}
		\includegraphics[width=\linewidth/2]{logo.jpg}%LOGOTIPO DA INSTITUICAO
	 	\vspace{2pt} 	
		
		\universidade
		\par
		\centro
		\par
		\departamento
		\par
	%	Curso de \curso
		\par
		\vspace{12pt}
		\LARGE \textbf{Lista 02}
		
	\end{center}
	
	\vspace{12pt}
	
	\begin{tabular}{ |l|p{12cm}| }
		
		\hline
		\multicolumn{2}{|c|}{\textbf{Dados de Identificação}} \\
		\hline
		Professor:         &    \professores           \\
		\hline
		Disciplina:        &    \disciplina          \\
		\hline
	%	Tema:              &    \tema                \\
	%	\hline
	%	Pré-requisito	:  &    \prerequisitos         \\
	%	\hline
		Aluno:             &                   \\
	%	\hline
	%	Data:              &    \data                \\
	%	\hline
	%	Duração da aula:   &    \tempodeaula         \\
		\hline
		
	\end{tabular}
	\vspace{6pt}
	
	{\bf Observação:} Não precisa ser entregue. Apenas para desenvolvimento do pensamento matemático.
	
	\begin{snugshade}
	\end{snugshade}

\begin{enumerate}

	\item Determine o limite (se existir) da sequência $X_n = \displaystyle \frac{\sen(n\pi)}{n}$
	\item Determine o limite (se existir) da sequência $X_n = \displaystyle \frac{\log(n)}{n}$ \\ \emph{DICA: Use que} $\forall n \in \mathbb{N}, \sqrt{n}<e^{\sqrt{n}}$
	\item Apresente dois exemplos de sequências de termos $a_n$, satisfazendo simultaneamente as condições abaixo:
        \begin{enumerate}
            \item $\displaystyle \lim_{n\to\infty}a_n=0$
            \item $\displaystyle \sum_{n=1}^{\infty}a_n$ diverge.
        \end{enumerate}
    \item Apresente um exemplo de uma sequência limitada que não é convergente.
    \item Determine o limite (se existir) das sequências abaixo:
        \begin{enumerate}
            \item $\displaystyle\frac{\cos(x)}{n}$
            \item $\displaystyle \frac{1}{2^n}$
            \item $a_n = \displaystyle \left(\frac{n+1}{n-1}\right)^n$
        \end{enumerate}
    \item Encontre a fórmula para o $n-$ésimo termo das sequências abaixo:
        \begin{enumerate}
            \item $1,-1,1,-1,\dots$
            \item $-1,1,-1,1,\dots$
            \item $1,4,9,16,25,\dots$
            \item $\displaystyle 1,-\frac{1}{4},\frac{1}{9},-\frac{1}{16},\dots$
        \end{enumerate}

	\item Analise a série $\displaystyle \sum \frac{1+n}{n}$
	\item Analise a série $\displaystyle \sum_{n=1}^{\infty}\frac{\ln(n)}{n^{\frac{3}{2}}}$
	\item Analise a série $\displaystyle \sum \frac{(2n)!}{n!n!}$
	\item Mostre que a série Telescópica é convergente.
	\item Mostre que a série Harmônica é divergente.
	\item Mostre que a série Harmônica alternada é convergente.
	\item Analise a série $\displaystyle \sum \frac{1}{n\ln(n)}$
	\item Analise a série $\displaystyle \sum_{n=2}^{\infty}\frac{n-1}{n!}$
	\item Analise a série $\displaystyle \sum_{n=1}^{\infty}\ln\left(\frac{n+1}{n}\right)$
	\item Mostre que a sequência $X_n = \displaystyle \frac{(n!)^{\frac{1}{n}}}{n}$ converge para $\displaystyle \frac{1}{e}$ \\ \emph{DICA: Use que} $\forall n \in \mathbb{N}, en^ne^{-n}<n!<en^{n+1}e^{-n}$
	\item Para quais valores de $a>0$ a série $\displaystyle \sum_{n=1}^{\infty}\frac{a^nn!}{n^n}$ converge?
	\item Estude a série $\displaystyle \sum_{n=1}^{\infty}\frac{1}{n^p}$. Para quais valores de $p$ a série é convergente? E para quais valores de $p$ a série é divergente?
	\item Analise a série $\displaystyle \sum \frac{(-1)^n}{n}$
	\item Analise a série $\displaystyle \sum_{n=1}^{\infty}\frac{\cos(3\pi^2n)}{n^2}$
	\item Analise a série $\displaystyle \sum_{n=1}^{\infty}\frac{n!}{n^n}$
	\item Encontre a série de Taylor e os polinômios de Taylor gerados por $f(x)=e^x$ em $x=0$.
	\item {\bf DESAFIO:} Mostre que se $a>0$, então $X_n = \displaystyle \sqrt[n]{a}=a^{\frac{1}{n}}$ é tal que o limite de $X_n$ é $1$.
	\item {\bf DESAFIO:} Analise a série $\displaystyle \sum_{n=1}^{\infty}\frac{4n+3}{n^3+\sqrt{n}}$\\ \emph{DICA: Use estimativas} $\displaystyle \frac{4n+3}{n^3\sqrt{n}} < \frac{4n}{n^3}$



\end{enumerate}


	
\end{document}
