\documentclass[oneside,a4paper,12pt]{article}
\usepackage[english,brazilian]{babel}
\usepackage{multicol}
\usepackage{textcomp}
\usepackage[alf]{abntex2cite}
\usepackage[utf8]{inputenc}
\usepackage[T1]{fontenc}
\usepackage{amsmath,amssymb,exscale}
\usepackage[top=20mm, bottom=20mm, left=20mm, right=20mm]{geometry}%margens cima, baixo, esquerda direita
\usepackage{framed}
\usepackage{booktabs} %Pacote para deixar tabelas mais bonitas.
\usepackage{color} %Pacote de Cores
\usepackage{hyperref} %Pacotes para Hiperlinks
\usepackage{graphicx} %Pacote de imagens
\graphicspath{{./Figuras/}}%Direciona as imagens para uma pasta chamada "Figuras" (uso isso para organizar. Uma vez que todas as imagens vao ficar em uma pasta isolada)    
\definecolor{shadecolor}{rgb}{0.8,0.8,0.8}

%FAZ EDICOES AQUI (somente no conteudo que esta entre entre as ultimas  chaves de cada linha!!!)
\newcommand{\universidade}{Universidade Estadual de Londrina}
\newcommand{\centro}{Centro de Ciências Exatas}
\newcommand{\departamento}{Departamento de Matemática}
\newcommand{\curso}{Ciência da Computação}
\newcommand{\professores}{Matheus Pimenta}
\newcommand{\disciplina}{Cálculo Diferencial e Integral II - 1MAT180}
%\newcommand{\tema}{Lista 01}
%\newcommand{\turma}{MA31G}
%\newcommand{\data}{Março de 2019}%{\today}
%\newcommand{\tempodeaula}{30 minutos}
%\newcommand{\prerequisitos}{Matrizes, Transformações Lineares e Bases}
%ATE AQUI !!!	

\begin{document}
	\pagestyle{empty}
	
	\begin{center}
		\includegraphics[width=\linewidth/2]{logo.jpg}%LOGOTIPO DA INSTITUICAO
	 	\vspace{2pt} 	
		
		\universidade
		\par
		\centro
		\par
		\departamento
		\par
	%	Curso de \curso
		\par
		\vspace{12pt}
		\LARGE \textbf{Lista 01 - REVISÃO}
		
	\end{center}
	
	\vspace{12pt}
	
	\begin{tabular}{ |l|p{12cm}| }
		
		\hline
		\multicolumn{2}{|c|}{\textbf{Dados de Identificação}} \\
		\hline
		Professor:         &    \professores           \\
		\hline
		Disciplina:        &    \disciplina          \\
		\hline
	%	Tema:              &    \tema                \\
	%	\hline
	%	Pré-requisito	:  &    \prerequisitos         \\
	%	\hline
		Aluno:             &                   \\
	%	\hline
	%	Data:              &    \data                \\
	%	\hline
	%	Duração da aula:   &    \tempodeaula         \\
		\hline
		
	\end{tabular}
	\vspace{6pt}
	
	{\bf Observação:} Confirme as respostas, você pode chegar em uma outra forma de apresentação das respostas.
	
	\begin{snugshade}
	\end{snugshade}

\begin{enumerate}

	\item Resolva:
        \begin{enumerate}
        \item Determine o coeficiente angular da reta tangente ao gráfico da função $f(x) = -3x^2+2x$, no ponto $P(2,f(2))$. \\\textbf{R: -10}
        \item Determine o coeficiente angular da reta tangente ao gráfico da função $f(x) = \sqrt{x}$, no ponto $P(1,1)$. \\\textbf{R: }$\displaystyle \frac{1}{2}$
        \item Determine a equação da reta tangente ao gráfico da função $f(x) = 3x^2 -5x + 1$ no ponto $P(2,3)$ \\\textbf{R: $y=7x-11$}
        \item Um ponto em movimento obedece a euqação horária $S=t^2+3t$. Determine a velocidade do móvel no instante $t=4s$, nas unidades $S$ em metros e $t$ em segundos. \\\textbf{R: $v(t_0) = S'(t_0) = 11m/s$}
        \item Um móvel se desloca segundo a função horária $S = t^3+t^2+t$. Determine a aceleração do móvel no instante $t=1s$. As unidades são as mesmas do item anterior. \\\textbf{R: $a(t_0) = 8m/s^2$}
        \end{enumerate}
    \item Utilizando as propriedades operatórias e as regras de derivação, calcule as derivadas das função abaixo:
        \begin{enumerate}
         \item $f(x) = 5x^3 - 2x^2 + x - 4$ \\\textbf{R:} $f'(x) = 15x^2 - 4x + 1$
         \item $f(x) = x^4 - \displaystyle \frac{2}{x^3} - \frac{8}{x} + 2$ \\\textbf{R:} $f'(x) = 4x^3 + \displaystyle \frac{6}{x^4} + \frac{8}{x^2}$
         \item $f(x) = (5x - 2)^6(3x-1)^3$ \\\textbf{R:} $f'(x) = (5x-2)^5(3x-1)^2(135x-48)$
         \item $f(x) = \displaystyle \sqrt[3]{(3x^2+6x-2)^2}$ \\\textbf{R:} $f'(x) = \displaystyle \frac{4(x+1)}{\sqrt[3]{3x^2+6x-2}}$
         \item $f(x) = \displaystyle \frac{a + \sqrt{x}}{a - \sqrt{x}}$ \\\textbf{R: } $f'(x) = \displaystyle \frac{a}{\sqrt{x}(a - \sqrt{x})^2}$
         \item $f(r) = \sqrt{\frac{1+r}{1-r}}$ \\\textbf{R: } $f'(r) = \displaystyle \frac{1}{(1-r)^2\sqrt{\frac{1+r}{1-r}}}$
         \item $f(x) = x^3\sqrt[4]{x^3}$ \\\textbf{R: } $f'(x) = \displaystyle \frac{15}{4}x^2\sqrt[4]{x^3}$
         \item $f(x) = \displaystyle \frac{1 + \cos(x)}{1 - \cos(x)}$ \\\textbf{R:} $f'(x) = \displaystyle \frac{-2\sin(x)}{(1-\cos(x))^2}$
         \item $f(x) = \displaystyle \frac{2-\sin(x)}{2 + \cos(x)}$ \\\textbf{R: } $f'(x) = \displaystyle \frac{2\sin(x) - 2\cos(x) - 1}{(2 + \cos(x))^2}$
         \item $f(x) = \displaystyle \frac{e^x}{\ln(x)}$ \\\textbf{R: } $f'(x) = \displaystyle \frac{xe^x\ln(x)-e^x}{x(\ln(x))^2}$
         \item $f(x) = \log_e\left(\displaystyle \frac{a+x}{a-x}\right)$ \\\textbf{R: } $f'(x) = \displaystyle \frac{2a}{a^2-x^2}$
         \item $f(x) = (x^3-2x)^{\ln(x)}$ \\\textbf{R: }$f'(x) = (x^3-2x)^{\ln(x)}\left[ \left( \displaystyle \frac{3x^2 -2}{x^3-2x}\right) \ln(x) + \displaystyle \frac{1}{x}\ln(x^3-2x) \right]$
         \item $f(x) = (\sin(x))^{\cos(x)}$ \\\textbf{R: } $f'(x) = (\sin(x))^{\cos(x)}\left( - \sin(x) \ln(\sin(x)) + \displaystyle \frac{\cos^2(x)}{\sin(x)} \right)$
         \item $f(x) = e^{\sin^3(x^w)}$ \\\textbf{R: } $f'(x) = 6x e^{\sin^3(x^2)}\sin^2(x^2)\cos(x^2)$ 
         \item $f(x) = \sqrt{4 + \text{cossec}^2(3x)}$ \\\textbf{R: }$f'(x) = \displaystyle\frac{-3\text{cossec}^2(3x) \text{cotg}(3x)}{\sqrt{4 + \text{cossec}^2(3x)}}$
         \item $f(x) = \ln\left(\displaystyle\sqrt{\frac{1+\sin(x)}{1-\sin(x)}}\right)$ \\\textbf{R: } $f'(x) = \sec(x)$
        \end{enumerate}
    \item Determine a derivada de segunda ordem das seguintes funções:
        \begin{enumerate}
         \item $y = \ln(x + \sqrt{a^2 + x^2})$ \\\textbf{R: } $y'' = \displaystyle \frac{-x}{\sqrt{(a^2+x^2)^3}}$
         \item $y = \ln(\sqrt[3]{1+x^2})$ \\\textbf{R: } $y'' = \displaystyle \frac{2(1-x^2)}{3(1+x^2)^2}$
         \item $y = e^{x^2}$ \\\textbf{R: } $y'' = e^{x^2}(4x^2+2)$
         \item $y = (1+x^2)\text{arctg}(x)$ \\\textbf{R: } $y'' = 2\text{arctg}(x)+\displaystyle\frac{2x}{1+x^2}$
         \item $y = (\text{arcsin}(x))^2$ \\\textbf{R: } $y'' = \displaystyle \frac{2}{1-x^2} + \displaystyle \frac{2x \text{arcsin}(x)}{(1-x^2)^{\frac{3}{2}}}$
        \end{enumerate}
    \item Expresse $\displaystyle \frac{\partial y}{\partial x}$ em termos de $x$ e $y$, onde $y=y(x)$, é uma função derivável dada implicitamente pela equação:
        \begin{enumerate}
         \item $e^y + \ln(y) = x$ \\\textbf{R: } $\displaystyle \frac{\partial y}{\partial x} = \displaystyle \frac{1}{e^y + \displaystyle \frac{1}{y}}$
         \item $xy+x-2y=1$ \\\textbf{R: } $\displaystyle \frac{\partial y}{\partial x} = - \displaystyle \frac{y+1}{x-2} $
         \item $2y + \sin(y) = x$ \\\textbf{R: } $\displaystyle \frac{\partial y}{\partial x} = \displaystyle \frac{1}{2 + \cos(y)} $
         \item $5y + \cos(y) = xy$ \\\textbf{R: } $\displaystyle \frac{\partial y}{\partial x} = \frac{y}{5 - \sin(y) -x} $
        \end{enumerate}
    \item Determine a derivada de ordem $123$ da função $y = \sin(x)$. \\\textbf{R: } $y^{(123)}=-\cos(x)$
    
    \item Demonstre que a função $y = \displaystyle \frac{1}{2}x^2e^x$, satisfaz a equação diferencial $y'' - 2y' + y = e^x$.
    
    \item Um retângulo de dimensões $x$ e $y$ tem perímetro $2a$ ($a$ é constante dada). Determinar $x$ e $y$ para que a sua área seja máxima. \\\textbf{R: } $x=y=\displaystyle \frac{a}{2}$
    
    \item A prefeitura de um município pretende construir um parque retangular, com área de $3600m^2$ e pretende protegê-lo com uma cerca. Que dimensões devem ter o parque para que o comprimento da cerca seja mínimo? \\\textbf{R: }$60m$
    
    \item Estima-se que daqui a $t$ anos, a circulação de um jornal será $C(t) = 100t^2 + 400t + 5000$.
        \begin{enumerate}
        \item Encontre uma expressão para a taxa de variação da circulação com o tempo daqui a $t$ anos. \\\textbf{R: } $C'(t) = 200t+400$
        \item Qual será a taxa de variação da circulação com o tempo daqui a $5$ anos? Nessa ocasião a circulação está aumentando ou diminuindo? \\\textbf{R: } $C'(5) = 1400$, aumentando
        \item Qual será a variação da circulação durante o sexto ano? \\\textbf{R: } $1500$ exemplares
        \end{enumerate}

    \item Utilizando a regra de L'Hôpital calcule os limites abaixo:
        \begin{enumerate}
         \item $\displaystyle \lim_{x\to 1} \displaystyle \frac{\ln(x)}{x-1}$ \\\textbf{R: }1
         \item $\displaystyle \lim_{x\to \infty} e^x\ln(x)$ \\\textbf{R: }0
         \item $\displaystyle \lim_{x\to \infty} \displaystyle \frac{e^x}{x^2}$ \\\textbf{R: } $+\infty$
         \item $\displaystyle \lim_{x\to 0} \displaystyle \frac{2x}{e^x - 1}$ \\\textbf{R: }2
         \item $\displaystyle \lim_{x\to 0} \displaystyle \frac{\tan(x) -x}{x^3}$ \\\textbf{R: } $\displaystyle \frac{1}{3}$
         \item $\displaystyle \lim_{x\to 0} \displaystyle \frac{\sin(x) - x}{\tan(x)-x}$ \\\textbf{R: } $\displaystyle -\frac{1}{2}$
        \end{enumerate}
        
    \item Estude as funções abaixo.\\ Dica: verifique os pontos de descontinuidade, interseção do gráfico com os eixos, comportamento no infinito, crescimento ou decrescimento, a concavidade, pontos de inflexão, os gráficos e os extremantes.
        \begin{enumerate}
         \item $f(x) = 2x^3 - 6x$
         \item $f(x) = (x-2)^2$
         \item $f(x) = 4x^3 - x^2 - 24x - 1$
         \item $f(x) = \displaystyle \frac{x + 1}{x -3}$
         \item $f(x) = 3x^4 + 4x^3 + 6x^2 - 4$
         \item $f(x) = xe^x$
        \end{enumerate}



\end{enumerate}


	
\end{document}
