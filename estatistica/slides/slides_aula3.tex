\documentclass[hyperref={pdfpagelabels=false}]{beamer}
\usepackage{lmodern}
\usetheme{CambridgeUS}

\usepackage[english,brazilian]{babel}
\usepackage{multicol}
\usepackage{textcomp}
\usepackage[alf]{abntex2cite}
\usepackage[utf8]{inputenc}
\usepackage[T1]{fontenc}

\usepackage{amsmath,amssymb,exscale}

\title{Probabilidade e Estatística}  
\author[Matheus Pimenta]{Matheus Pimenta} 
\institute[UTFPR-CP]{\normalsize Universidade Tecnológica Federal do Paraná \\
	Câmpus Cornélio Procópio
} 
\date{Agosto de 2020 \\ ADNP 2020} 
\begin{document}
	
\begin{frame}
\titlepage
\end{frame} 


%\begin{frame}
%\frametitle{Table of contents}
%\tableofcontents
%\end{frame} 


\section{Variáveis Aleatórias} 

\begin{frame}
\frametitle{Exemplo}

São lançadas três moedas.Seja $X$ o número de ocorrências de ``cara''. Qual a distribuição de probabilidade de $X$?

\pause
{\it Solução:}

$\footnotesize{ \Omega = \{ (c,c,c), (c,c,k), (c,k,c), (c,k,k), (k,c,c), (k,c,k), (k,k,c), (k,k,k) \}}$

\pause

Se $X$ é o número de ``cara'' que pode-se ter então $X$ tem os seguintes valores: $0,1,2$ e $3$. Associando os valores a $\Omega$, segue:

\begin{table}[!h]
	\centering
	\begin{tabular}{|c|c|}
		\hline
		$X$		&	Evento Associado \\ \hline \pause
		$0$		&	$A_1 = \{(k,k,k)\}$	\\ \hline \pause
		$1$		&	$A_2 = \{(c,k,k),(k,c,k),(k,k,c)\}$	\\ \hline \pause
		$2$		&	$A_3 = \{(c,c,k),(c,k,c),(k,c,c)\}$	\\ \hline \pause
		$3$		&	$A_4 = \{(c,c,c)\}$	\\ \hline
	\end{tabular}
	\caption{Resolução do Exemplo 01.}
\end{table}
\pause

\end{frame}

\begin{frame}
\frametitle{Exemplo}
Associando a cada evento uma probabilidade temos que:
$P(X=0)=P(A_1) = \displaystyle \frac{1}{8}$
\vspace{4pt}

\pause
$P(X=1)=P(A_2) = \displaystyle \frac{3}{8} = P(X=2)=P(A_3)$
\vspace{4pt}

\pause
$P(X=3)=P(A_4) = \displaystyle \frac{1}{8}$

\end{frame}


\begin{frame}
	\frametitle{Exemplo}
	
Dessa maneira:
	
	\begin{table}[!h]
		\centering
		\begin{tabular}{|c|c|}
			\hline
			$X$		&	Evento Associado \\ \hline \pause
			$0$		&	$\displaystyle \frac{1}{8}$	\\ \hline \pause
			$1$		&	$\displaystyle \frac{3}{8}$	\\ \hline \pause
			$2$		&	$\displaystyle \frac{3}{8}$	\\ \hline \pause
			$3$		&	$\displaystyle \frac{1}{8}$	\\ \hline
					&	$1$							\\ \hline
		\end{tabular}
		\caption{Resolução do Exemplo 01.}
	\end{table}
	\pause
	
	{\bf Através do Gráfico e Diagrama}
	
\end{frame}




\begin{frame}
\frametitle{Definição}
	
{\bf Definição de Variável Aleatória:}

É a função que associa a todo evento pertencente a uma partição do espaço amostral um único valor real.

No caso discreto, a variável deve assumir valores em um conjunto finito ou em um conjunto infinito, porém enumerável.

No caso finito, será indicado por:
$$X: x_1,x_2,\dots,x_n$$
	
\end{frame}


\begin{frame}
\frametitle{Definição}

{\bf Definição Função de Probabilidade:}

É a função que associa a cada valor assumido pela variável aleatória a probabilidade do evento correspondente, isto é:

$$P(X=x_i) = P(A_i),i=1,2,\dots,n$$
\end{frame}


\begin{frame}
\frametitle{Definição}

{\bf Definição Distribuição de Probabilidades da Variável Aleatória $X$:}

	É o conjunto $\{ x_i, p(x_i), i=1,2,\dots,n \}$.
	
	\pause
	
	{\bf Observação:} Para que faça sentido uma distribuição de probabilidades de uma variável aleatória $X$, é necessário que:
	$$\sum_{i=1}^{n}p(x_i) = 1$$
\end{frame}

\begin{frame}
\frametitle{Esperança Matemática}

Quando trabalhamos com distribuições de probabilidades de uma variável aleatória discreta, os parâmetros da distribuição são características numéricas de grande importância.

\pause
O primeiro parâmetro é a \emph{esperança matemática} (ou simplesmente média) de uma variável aleatória.

\pause
{\bf Definição Esperança Matemática:}

	$$E(X) = \displaystyle \sum_{i=1}^{n}x_i \cdot p(x_i)$$
	
	A \emph{esperança matemática} é um número real. É também uma média aritmética ponderada. 
	
	Notação: $E(X)$, $\mu(x)$, $\mu_x$, $\mu$.
	
\end{frame}

\begin{frame}
\frametitle{Exemplo}

{\bf Exemplo 01:} Uma seguradora paga $R\$ 30.000,00$ em caso de acidente de carro e cobra uma taxa de $R\$ 1.000,00$. Sabe-se que a probabilidade de que um carro sofra acidente é de $3\%$. Quanto espera a seguradora ganhar por carro segurado?

\pause
{\it Solução:}
\pause

Suponha que em $100$ carros, $97$ dão lucro de $R\$ 1.000,00$ e $3$ dão prejuízo de $R\$ 29.000,00$ $(R\$ 30.000,00 - R\$ 1.000,00)$.
\pause

Lucro total: \pause $97 \cdot 1.000 - 3 \cdot 29.000 = R\$ 10.000,00$

\pause
Lucro médio por carro: \pause $R\$ 10.000,00 / 100 = R\$ 100,00$

\pause
Se chamarmos de $X$ o lucro por carro, e o lucro médio por carro de $E(X)$, teremos:
\pause

$$E(X) = \displaystyle \frac{0,97 \cdot 1.000}{0,03 \cdot 29.000} = 100$$

\begin{flushright}
	$\blacksquare$
\end{flushright}
\end{frame}


\begin{frame}
\frametitle{Propriedades da Esperança Matemática}

\begin{enumerate}
	\item $E(k) = k$, onde $k$ é uma constante; \pause
	\item $E(k\cdot X) = k \cdot E(X)$; \pause
	\item $E(X \pm Y) = E(X) \pm E(Y)$; \pause
	\item $E\left\{ \displaystyle \sum_{i=1}^{n}X_i \right\} = \displaystyle \sum_{i=1}^{n}\{ E(X_i) \}$; \pause
	\item $E(aX \pm b) = aE(X) \pm b$, onde $a$ e $b$ são constantes; \pause
	\item $E(X - \mu_x) = 0$. \pause
\end{enumerate}

\end{frame}

\begin{frame}
\frametitle{Variância}

A medida que dá o grau de dispersão (ou concentração) de probabilidade em torno da média é a \emph{variância}.

\pause
{\bf Definição Variância:}

	$$VAR(X) = \sum_{i = 1}^{n}(x_i - \mu_x)^2 \cdot p(x_i)$$
	
	Notação: $VAR(X)$, $V(X)$, $\sigma^2(X)$, $\sigma_X^2$,$\sigma^2$.

\pause

{\bf Observação:} Quanto menor a variância, menor o grau de dispersão de probabilidades em torno da média e vice-versa; quanto maior a variância, maior e grau de dispersão da probabilidade em torno da média.

\end{frame}

\begin{frame}
\frametitle{Desvio Padrão}

{\bf Definição Desvio Padrão:}

É a raiz quadrada da variância de $X$, isto é:
$$\sigma_x = \sqrt{VAR(X)}$$

\pause
{\bf Propriedades da Variância}

\begin{enumerate}
	\item $VAR(k) = 0$, onde $k$ é constante; \pause
	\item $VAR(k\cdot X) = k^2 \cdot VAR(X)$; \pause
	\item $VAR(X \pm Y) = VAR(X) + VAR(Y) \pm 2cov(X,Y)$ \pause
	
	\footnotesize{
	{\bf Definição Covariância entre $X$ e $Y$:}
	
		$$cov(X,Y) =  E\{[X - E(X)]\cdot[Y - E(Y)]\}$$
		A covariância mede o grau de dependência entre as duas variáveis $X$ e $Y$.
	}
	
	\item $VAR\left(\displaystyle \sum_{i=1}^{n}X_i\right) = \displaystyle \sum_{i=1}^{n}VAR(X_i) + 2\displaystyle \sum_{i<j}^{n}cov(X_i , Y_j)$ \pause
	\item $VAR(aX \pm b) = a^2VAR(X)$, $a$ e $b$ constantes.
\end{enumerate}


\end{frame}

\begin{frame}
\frametitle{Função de Distribuição}
{\bf Definição Função de Distribuição:}

$$F(x) = P(X \leq x) = \sum_{x_i \leq x}p(x_i)$$

\pause
{\bf Propriedades de $F(x)$}
\pause
\begin{enumerate}
	\item $0 \leq F(x) \leq 1$; \pause
	\item $(F - \infty) = 0$; \pause
	\item $F(+\infty) = 1$; \pause
	
	As demais propriedades requer conhecimentos prévios de cálculo.
\end{enumerate}

\end{frame}

\begin{frame}
	\frametitle{Exemplo}

Suponha que uma variável aleatória $X$ tenha a seguinte distribuição de probabilidades: \pause
\begin{table}[!h]
	\centering
	\begin{tabular}{|c|c|}
		\hline
		$X$		&	$P(X)$ \\ \hline 
		$1$		&	$0,1$	\\ \hline 
		$2$		&	$0,2$	\\ \hline 
		$3$		&	$0,4$	\\ \hline
		$4$		&	$0,2$	\\ \hline
		$5$		&	$0,1$	\\ \hline
	\end{tabular}
	\caption{Resolução do Exemplo 01.}
\end{table}
\end{frame}

\begin{frame}
	\frametitle{Exemplo}

Dessa maneira:
\pause
\begin{align*}
F(1) &= P(X \leq 1) = P(X=1) = 0,1 \\ 
F(2) &= P(X \leq 2) = P(X=1) + P(X=2) = 0,3 \\ 
F(3) &= P(X \leq 3) = P(X=1) + P(X=2) + P(X=3)\\
& = F(2) + P(X=3) = 0,3 + 0,4 = 0,7 \\ 
&\vdots& \\ 
F(5) &= P(X \leq 4) + P(X = 5) = F(4) + P(X=5) = 0,9 + 0,1 = 1
\end{align*}
\pause
Pode-se utilizar valores contínuos também.

\end{frame}

\begin{frame}{Exemplo}
	Dessa maneira pode-se resumir através de:
	$$F(x) = \begin{cases}
	0	& \text{,se } X < 1 \\
	0,1 & \text{,se } 1 \leq X < 2 \\
	0,3 & \text{,se } 2 \leq X < 3 \\
	0,7 & \text{,se } 3 \leq X < 4 \\
	0,9 & \text{,se } 4 \leq X < 5 \\
	1   & \text{,se } X \geq 5
	\end{cases}$$
\pause

O domínio de $F(x)$ será $D = \{\mathbb{R}\}$ e o contradomínio será o conjunto $\{ 0,1;0,3;0,7;0,9;1 \}$.

\end{frame}



\end{document}

