\documentclass[hyperref={pdfpagelabels=false}]{beamer}
\usepackage{lmodern}
\usetheme{CambridgeUS}

\usepackage[english,brazilian]{babel}
\usepackage{multicol}
\usepackage{textcomp}
\usepackage[alf]{abntex2cite}
\usepackage[utf8]{inputenc}
\usepackage[T1]{fontenc}

\usepackage{amsmath,amssymb,exscale}

\title{Probabilidade e Estatística}  
\author[Matheus Pimenta]{Matheus Pimenta} 
\institute[UTFPR-CP]{\normalsize Universidade Tecnológica Federal do Paraná \\
	Câmpus Cornélio Procópio
} 
\date{ADNP 2020} 
\begin{document}
	
\begin{frame}
\titlepage
\end{frame} 


%\begin{frame}
%\frametitle{Table of contents}
%\tableofcontents
%\end{frame} 


\section{Intervalos de Confiança} 

\begin{frame}
\frametitle{Estimação}
\textbf{Definição:} As conjunto de técnicas e procedimentos que permitem dar ao pesquisador um grau de confiabilidade, de confiança, nas afirmações que faz para a população baseadas nos resultados das amostras.

\pause
O problema fundamental da inferência estatística, portanto, é medir o \emph{grau de incerteza ou risco} dessas generalizações.

\end{frame}

\begin{frame}{Estimação de Parâmetros}
É um dos objetivos básicos da experimentação. São dois tipos de estimação: por pontos e por intervalos.

\pause

{\bf Estimação por Pontos:} a partir das observações, calcula-se uma estimativa, usando o estimador ou ``estatística''.
\end{frame}

\begin{frame}{Qualidades de um Bom Estimador}
Quanto maior o grau de concentração da distribuição amostral do estimador em torno do verdadeiro valor do parâmetro populacional, tanto melhor será o estimador.

\pause
As principais qualidades de um estimador são:
\pause
\begin{enumerate}
	\item [a)] consistência;\pause
	\item [b)] ausência de vício; \pause
	\item [c)] eficiência; \pause
	\item [d)] suficiência.
\end{enumerate}

\pause
As definições formais requerem conhecimentos de cálculo.
\end{frame}

\begin{frame}{Estimação por Intervalo}
{\bf Definição:} A estimação por pontos de um parâmetro não possui uma medida do possível erro cometido na estimação. Para solucionar isto, uma alternativa é estabelecer \emph{limites}, que com certa probabilidade incluam  o verdadeiro valor do parâmetro da população.

\pause
Estes limites, são definidos como \emph{limites de confiança} e determinam um intervalo de confiança, no qual deverá estar o verdadeiro valor do parâmetro.

\pause
Assim, a estimação por intervalos consiste na fixação de dois valores, tais que $(1 - \alpha)$ seja a probabilidade de que o intervalo, por eles determinado, contenha o verdadeiro valor do parâmetro.
\end{frame}

\begin{frame}{Estimação por Intervalo}
	$\alpha:$ é o nível de incerteza ou grau de desconfiança;
	
	\pause
	$1 - \alpha:$ é o nível de confiabilidade.
	
	\pause
	Logo, $\alpha$ nos da o nível de incerteza desta inferência, chamamos de grau de significância.
\end{frame}

\begin{frame}{Intervalos de Confiança (IC) para a média $\mu$ de uma população Normal com variância $\sigma^2$ conhecida}
	Considerando uma população normal com média desconhecida que desejamos estimar e com $\sigma^2$ conhecida, $X:N(?,\sigma^2)$.
	\pause
	
	O passo a passo para obter intervalos de confiança são:
	\pause
	\begin{enumerate}
		\item Retiramos uma amostra casual simples com $n$ elementos;
		\item Calculamos a média da amostra $\bar{x}$;
		\item Calculamos o desvio padrão da média amostral: $\sigma_{\bar{x}} = \displaystyle \sqrt{\frac{\sigma^2}{n}} = \frac{\sigma}{\sqrt{n}}$;
		\item Fixamos um nível de significância $\alpha$, e com ele determinamos $z_\alpha$, tal que $P(|z|>z_\alpha) = \alpha$, ou seja: $P(z>z_\alpha) = \displaystyle \frac{\alpha}{2}$ e $P(z<-z_\alpha) = \displaystyle \frac{\alpha}{2}$. Logo devemos ter: $P(|z| < z_\alpha) = 1-\alpha$. 
		
		Com isso, desenvolvendo a fórmula anterior chegamos a:
		$$P(\bar{x}-z_\alpha\cdot \sigma_{\bar{x}} < \mu < \bar{x}+z_\alpha\cdot \sigma_{\bar{x}})$$
		
		Que é a fórmula do $IC$ para a média de populações normais com variância conhecidas.
	\end{enumerate}
\end{frame}

\begin{frame}{Intervalos de Confiança (IC) para a média $\mu$ de uma população Normal com variância $\sigma^2$ conhecida}
	Simplificando a notação temos com os limites anteriores: $\mu_1 = \bar{x}-z_\alpha\cdot \sigma_{\bar{x}} $ e $\mu_2 = \bar{x}+z_\alpha\cdot \sigma_{\bar{x}}$, com isto segue que:
	$$IC(\mu,(1-\alpha)\%) = (\mu_1,\mu_2)$$
	\pause
	
	Em outras palavras, tomando $\alpha = 5\%$, podemos esperar que $95$ dos $IC$ contenham o verdadeiro valor de $\mu$ e $5$ não contenham o valor de $\mu$, em $100$ amostras de mesmo tamanho $n$, onde obteremos $100$ estimativas para $\bar{x}$, com as quais construiremos $100$ $IC$ para $\mu$.
	\pause
	
	Isto é, em uma amostra qualquer, a probabilidade de que o $IC$ determinado contenha o valor da média é de $95\%$, ou seja, uma confiança de $95\%$ de que o $IC$ determinado contenha o verdadeiro valor de $\mu$. O risco que corremos de que não contenha o verdadeiro valor é de $5\%$.
\end{frame}

\begin{frame}{Tabela}
	\begin{center}
		Tabela para valores de $Z_{\alpha}$.
		\begin{table}[h]
			\tiny
			\begin{tabular}{|c|cccccccccc|}
				\hline
				Nível de Confiança	& $99,73$\% & $99$\% & $98$\% & $96$\% & $95,45$\% & $95$\% & $90$\% & $80$\% & $68,27$\% & $50$\% \\ \hline
				$Z_{\alpha}$	& $3,00$ & $2,58$ & $2,33$ & $2,05$ & $2,00$ & $1,96$ & $1,645$ & $1,28$ & $1,00$ & $0,6745$ \\ \hline
			\end{tabular}
		\end{table}
	\end{center}
\end{frame}

\begin{frame}{Exemplo 01}
	De uma população normal $X$, com $\sigma^2 = 9$, tiramos uma amostra de $25$ observações, obtendo $\displaystyle \sum_{i=1}^{25}x_i = 152$. Determinar um $IC$ de limites de $90\%$ para $\mu$.
	
\end{frame}

\begin{frame}{Exemplo 01 - Solução}
	\textit{Solução: }
	
	$\alpha = 10\%$ e $\bar{x} = \displaystyle \frac{1}{n}\sum_{i = 1}^{n}x_i = \frac{152}{25} = 6,08$
	
	\pause
	$\sigma_{\bar{x}} = \displaystyle \sqrt{\frac{\sigma^2}{25}} = \sqrt{\frac{9}{25}} = \frac{3}{5} \implies \sigma_{\bar{x}} = 0,6$
	
	\pause
	Utilizando a tabela do inicio das notas de aula, segue que:
	
	\pause
	$z_\alpha = z_{45\%} = z_{0,45} = 1,64$
	
	\pause
	Com isto, nosso intervalo de confiança é dado por:
	$$P(6,08 - 1,64 \cdot 0,6 < \mu < 6,08 + 1,64 \cdot 0,6) = 0,9$$
	$$P(5,096 < \mu < 7,064) = 0,90$$
	\pause
	Ou ainda,
	$$IC(\mu,90\%) = (5,096; 7,064)$$
	
\end{frame}


\begin{frame}{Exemplo 01 - Solução}
	Portanto, temos $90\%$ de confiança que o verdadeiro valor  $\mu$ populacional se encontra entre $5,096$ e $7,064$, ou então corremos um risco de $10\%$ de que o verdadeiro valor da média $\mu$ populacional seja menor que $5,096$ ou maior que $7,064$.

\begin{flushright}
	$\blacksquare$
\end{flushright}
\end{frame}

\begin{frame}{Intervalos de Confiança para a Média de Populações Normais com Variâncias Desconhecidas}
	
	Quando queremos estimar a média de uma população normal com variância desconhecida, consideramos dois procedimentos:
	\pause
	\begin{itemize}
		\item se $n \leq 30$, então usa-se a distribuição $t$ de Student, que veremos a diante;\pause
		\item se $n > 30$, então usa-se a distribuição normal com o estimador $s^2$ de $\sigma^2$.
	\end{itemize}
	
	\pause
	Nesta seção nosso interesse é no segundo caso. Vejamos um exemplo.
	
\end{frame}


\begin{frame}{Exemplo 02}
	
	De uma população normal com parâmetros desconhecidos, tiramos uma amostra de tamanho $100$, obtendo-se $\bar{x} = 112$ e $s = 11$. Fazer um $IC$ para $\mu$ ao nível de $10\%$.
		
\end{frame}

\begin{frame}{Exemplo 02 - Solução}
\textit{Solução:} 

Como a amostra é superior a $30$, utilizamos: \pause

$\sigma_{\bar{x}} \approx \displaystyle \frac{s}{\sqrt{n}} = \frac{11}{10} = 1,1$ \pause

$z_\alpha = z_{45\%} = z_{0,45} = 1,64$ \pause

Logo, \pause

$$P(112 - 1,64 \cdot 1,1 < \mu < 112 + 1,64 \cdot 1,1) = 0,90$$
$$P(110,20<\mu<113,80) = 0,90$$ \pause
Ou \pause
$$IC(\mu,90\%) = (110,20;113,80)$$ \pause

O que concluímos que apesar de usar o desvio padrão da amostra, temos um grau de certeza de $90\%$ de que o verdadeiro valor da média populacional está entre $110,20$ e $113,80$.	$\blacksquare$

\end{frame}

\end{document}

