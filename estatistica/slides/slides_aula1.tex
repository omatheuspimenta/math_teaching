\documentclass[hyperref={pdfpagelabels=false}]{beamer}
\usepackage{lmodern}
\usetheme{CambridgeUS}

\usepackage[english,brazilian]{babel}
\usepackage{multicol}
\usepackage{textcomp}
\usepackage[alf]{abntex2cite}
\usepackage[utf8]{inputenc}
\usepackage[T1]{fontenc}

\usepackage{amsmath,amssymb,exscale}

\title{Probabilidade e Estatística}  
\author[Matheus Pimenta]{Matheus Pimenta} 
\institute[UTFPR-CP]{\normalsize Universidade Tecnológica Federal do Paraná \\
	Câmpus Cornélio Procópio
} 
\date{Agosto de 2020 \\ ADNP 2020} 
\begin{document}
	
\begin{frame}
\titlepage
\end{frame} 


%\begin{frame}
%\frametitle{Table of contents}
%\tableofcontents
%\end{frame} 


\section{Conceitos Básicos - Probabilidade} 

\begin{frame}{Conteúdo Programático}
	\begin{itemize}
		\item Espaço Amostral
		\pause
		\item Representação do Espaço Amostral
		\pause
		\begin{itemize}
			\item Tabelas
			\pause
			\item Diagramas
		\end{itemize}
		\pause
		\item Operações com Eventos Aleatórios
		\pause
		\item Propriedades das Operações
		\pause
		\item Função de Probabilidade
		\pause
		\item Alguns Teoremas
	\end{itemize}
\end{frame}

\begin{frame}
\frametitle{Espaço Amostral}

Os fenômenos podem ser classificados como: {\it determinísticos} ou {\it aleatórios}.

\pause

{\bf Definição fenômenos determinísticos:}

São aqueles que o resultado são sempre iguais, qualquer que seja o número de ocorrências verificadas.

Por exemplo, a temperatura que a água entra em ebulição, a temperatura de solidificação de um certo composto.

\pause

{\bf Definição fenômenos aleatórios:}

São do tipo que não podem ser previsíveis, mesmo que haja um grande número de repetições do mesmo fenômeno. 

Alguns exemplos são: condições climáticas no próximo mês, resultado de lançamento de um dado, número de veículos que passam em um cruzamento em um certo período do dia.

\end{frame}

\begin{frame}
\frametitle{Espaço Amostral - Exemplos}
	
Os {\it experimentos aleatórios} são os fenômenos que mesmo com as mesmas condições iniciais, os resultados finais de cada tentativa do experimento são diferentes e não previsíveis.

\pause 

Alguns exemplos:
{\begin{enumerate}
		\item lançamento de uma moeda honesta;
		\pause
		\item lançamento de um dado;
		\pause
		\item retirada de uma carta de um baralho de $52$ cartas;
\end{enumerate}}

\pause
Quando os resultados não podem ser previsíveis é definidos como {\it eventos aleatórios}.

\pause
No exemplo $1$, os eventos aleatórios associados são: {\it cara} ou {\it coroa}, já no exemplo $2$ os eventos aleatórios associados são as faces do dados que poderão ocorrer $1, 2, 3, 4, 5$ ou $6$.
	
\end{frame}


\begin{frame}
\frametitle{Espaço Amostral}

{\bf Definição Espaço Amostral:}

É o conjunto dos resultados do experimento. Os {\it pontos amostrais} são os elementos do espaço amostral. 

\pause

A notação do espaço amostral é: $\Omega$.

\pause
Nossos exemplos anteriores, os espaços amostrais são:
\begin{enumerate}
	\item $\Omega = \{ cara, coroa \}$
	\pause
	\item $\Omega = \{ 1,2,3,4,5,6 \}$
	\pause
	\item $\Omega = \{ A_O,\dots,K_O, A_P,\dots,K_P, A_E,\dots, K_E, A_C,\dots, K_C \}$
\end{enumerate}

\end{frame}


\begin{frame}
\frametitle{Espaço Amostral}

Pode-se ter a união de vários eventos aleatórios como interesse.

\pause
\begin{enumerate}
	\item saída de faces iguais;
	\pause
	\item saída de faces cuja a soma seja igual a 10;
	\pause
	\item saída de faces cuja a soma seja inferior a 2;
	\pause
	\item saída de faces cuja a soma seja inferior a 15;
	\pause
	\item saída de faces onde uma  face é o dobro da outra;
\end{enumerate}

\pause
Para determinar qual é o espaço amostral, podemos utilizar uma tabela ou um diagrama de árvore.
\end{frame}

\begin{frame}
\frametitle{Tabela e Diagrama de Árvore}

{\bf Tabela e Diagrama de Árvore}
\end{frame}

\begin{frame}
\frametitle{Espaço Amostral}

Assim, os eventos pedidos são:
\begin{enumerate}
	\pause
	\item $\Omega = \{ (1,1), (2,2), (3,3), (4,4), (5,5), (6,6) \}$
	\pause
	\item $\Omega = \{ (4,6), (5,5), (6,4) \}$
	\pause
	\item $\Omega = \varnothing$ (evento impossível)
	\pause
	\item $\Omega = \Omega$ (evento certo)
	\pause
	\item $\Omega = \{ (1,2), (2,1), (2,4), (3,6), (4,2), (6,3) \}$
\end{enumerate}

\end{frame}

\begin{frame}
\frametitle{Operações com Eventos Aleatórios}

Considere o espaço amostral finito $\Omega = \{ e_1, e_2, \dots, e_n \}$. Sejam $A$ e $B$ dois eventos da classe de eventos $F(\Omega)$. As seguintes operações  estão definidas:
\pause
\begin{itemize}
	\item União:
	
	{\bf Definição:} $A \cup B = \{ e_i \in \Omega; e_i \in A \lor e_i \in B \}$, $i = 1,\dots,n$.
	
	O {\it evento reunião} é formado pelos pontos amostrais que pertencem a pelo menos um dos eventos.
	
	\begin{center}
		{\bf \tiny DIAGRAMA DE VEEN}
	\end{center}
	\pause
	\item Interseção:
	
	{\bf Definição:} $A \cap B = \{ e_i \in \Omega; e_i \in A \land e_i \in B \}$, $i = 1,\dots,n$
	
	O {\it evento interseção} é formado pelos pontos amostrais que pertencem simultaneamente aos eventos $A$ e $B$. Se $A \cap B = \varnothing$ dizemos que $A$ e $B$ são {\it eventos mutualmente exclusivos}.
	
	\begin{center}
		{\bf \tiny DIAGRAMA DE VEEN}
	\end{center}	
	\pause
	\item Complementação:
	
	{\bf Definição:} $\Omega - A = \bar{A} = \{ e_i \in \Omega; e_i \notin A \}$	
	
	\begin{center}
		{\bf \tiny DIAGRAMA DE VEEN}
	\end{center}
	
\end{itemize}


\end{frame}




\begin{frame}
\frametitle{Propriedades das Operações}

Sejam $A,B$ e $C$ eventos associados a um espaço amostral $\Omega$. As propriedades a seguir são válidas:
\pause

\begin{enumerate}
	\footnotesize{
	\item Comutativa:
	
	$A \cup B = B \cup A$
	
	$A \cap B = B \cap A$
	\pause
	\item Idempotentes:
	
	$A \cup A = A$
	
	$A \cap A = A$
	\pause
	\item Associativas
	
	$A \cap (B \cap C) = (A \cap B) \cap C$
	
	$A \cup (B \cup C) = (A \cup B) \cup C$
	\pause
	\item Distributivas:
	
	$A \cup (B \cap C) = (A \cup B)\cap (A \cup C)$
	
	$A \cap (B \cup C) = (A \cap B)\cup (A \cap B)$
	\pause
	\item Identidades:
	
	$A \cap \Omega = A$
	
	$A \cup \Omega = \Omega$
	
	$A \cap \varnothing = \varnothing$
	
	$A \cup \varnothing = A$
	}
\end{enumerate}


\end{frame}

\begin{frame}
\frametitle{Função de Probabilidade}

{\bf Definição:}

	É a função $P$ que associa a cada evento de $F$ um número real pertencente ao intervalo $[0,1]$, que satisfaz os seguintes axiomas:\pause
	\begin{itemize}
		\item $P(\Omega) = 1$;
		\pause
		\item $P(A \cup B) = P(A) + P(B)$, se $A$ e $B$ forem mutualmente exclusivos;
		\pause
		\item $P\left(\displaystyle\cup_{i=1}^{n} A_i \right) = \displaystyle \sum_{i = 1}^{n}P(A_i)$, se $A_1,A_2,\dots,A_n$ forem, dois a dois, eventos mutualmente exclusivos.
		\pause
	\end{itemize}
	
	
	Pela definição acima, temos que $0\leq P(A) \leq 1$, para todo evento $A$, tal que $A \subset \Omega$.
\end{frame}

\begin{frame}
\frametitle{Teoremas}
{\bf Teorema:}

	Se os eventos $A_1,A_2,\dots,A_n$ formam uma partição do espaço amostral, então:
	$$\displaystyle \sum_{i = 1}^{n}P(A_i) = 1$$
\pause

{\bf Teorema:}

	Se $\varnothing$ é o evento impossível, então:
	$$P(\varnothing) = 0$$

\pause

{\bf Teorema: do evento complementar}

	Para todo evento $A \subset \Omega$, é valido:
	$$P(A) + P(\bar{A}) = 1$$

\end{frame}

\begin{frame}
	\frametitle{Teoremas}

{\bf Teorema: da soma}

Sejam $A \subset \Omega$ e $B \subset \Omega$. Então:
$$P(A \cup B) = P(A) + P(B) - P(A \cap B)$$
\pause

{\bf Teorema:}

	Para $A \subset \Omega$ e $B \subset \Omega$, temos:
	$$P(A \cup B) \leq P(A) + P(B)$$
\pause

{\bf Teorema:}

	Dado o espaço amostral $\Omega$ e os eventos $A_1, A_2, \dots, A_n$, então:
	\footnotesize{
	$$P\left(\displaystyle\cup_{i=1}^{n} A_i \right) = \displaystyle \sum_{i = 1}^{n}P(A_i) - \displaystyle \sum_{i \neq j}^{n}P(A_i \cap A_j) + \displaystyle \sum_{i \neq j \neq k}^{n}P(A_i \cap A_j \cap A_k) - \dots + (-1)^{n-1}P(A_1\cap\dots\cap A_n)$$}
\end{frame}

\begin{frame}
	\frametitle{Teoremas}

{\bf Teorema:}

	Dados os eventos $A_1,A_2,\dots,A_n$, então:
	$$P\left( \displaystyle \cup_{i=1}^{n} A_i \right) \leq \displaystyle \sum_{i = 1}^{n}P(A_i)$$

\end{frame}

\end{document}

