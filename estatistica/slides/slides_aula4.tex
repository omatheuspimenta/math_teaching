\documentclass[hyperref={pdfpagelabels=false}]{beamer}
\usepackage{lmodern}
\usetheme{CambridgeUS}

\usepackage[english,brazilian]{babel}
\usepackage{multicol}
\usepackage{textcomp}
\usepackage[alf]{abntex2cite}
\usepackage[utf8]{inputenc}
\usepackage[T1]{fontenc}

\usepackage{amsmath,amssymb,exscale}

\title{Probabilidade e Estatística}  
\author[Matheus Pimenta]{Matheus Pimenta} 
\institute[UTFPR-CP]{\normalsize Universidade Tecnológica Federal do Paraná \\
	Câmpus Cornélio Procópio
} 
\date{ADNP 2020} 
\begin{document}
	
\begin{frame}
\titlepage
\end{frame} 


%\begin{frame}
%\frametitle{Table of contents}
%\tableofcontents
%\end{frame} 


\section{Variáveis Aleatórias Contínuas} 

\begin{frame}
\frametitle{Variáveis Aleatórias Contínuas}

Seja o seguinte exemplo, considerando as probabilidades da variável aleatória discreta $X$:

\begin{table}[!h]
	\centering
	\begin{tabular}{|c|c|}
		\hline
		$X$	&	$P(X)$	\\\hline
		$1$	&	$0,1$	\\\hline
		$2$	&	$0,2$	\\\hline
		$3$	&	$0,4$	\\\hline
		$4$	&	$0,2$	\\\hline
		$5$	&	$0,1$	\\\hline	
	\end{tabular}
\end{table}

Faremos o \emph{histograma} da distribuição de probabilidades de $X$.


\end{frame}

\begin{frame}
\frametitle{Histograma}
O {\it histograma} é um gráfico da distribuição de $X$. É construído com retângulos de bases unitárias e alturas iguais às probabilidades de $X = x_0$.

\pause
Para calcularmos, por exemplo, $P(1\leq X \leq 3)$, somamos as áreas dos retângulos $1,2$ e $3$.

\pause
Ao ligar os pontos médios de todos os retângulos teremos uma curva, se considerarmos $X$ uma \emph{uma variável aleatória contínua}, essa curva representará uma função contínua $f(X)$, representada no gráfico.

\end{frame}


\begin{frame}
	\frametitle{Definição}
	
{\bf Variável Aleatória Contínua:}

Uma variável aleatória $X$ é contínua em $\mathbb{R}$ se existir uma função $f(x)$, tal que: \pause
\begin{itemize}
	\item $f(x) \geq 0$, isto é, $f(x)$ é não negativa; \pause
	\item $\displaystyle \int_{- \infty}^{+ \infty}f(x)dx = 1$ \pause
\end{itemize}
	
	A função $f(x)$ é chamada \emph{função densidade de probabilidade (f. d. p.)}. \pause
	
	$$P(a \leq X\leq b) = \displaystyle \int_{a}^{b}f(x)dx$$
	
\end{frame}

\begin{frame}
\frametitle{Definição}
	
{\bf Esperança de uma variável aleatória contínua:} \pause

$$E(X) = \displaystyle \int_{- \infty}^{+ \infty}xf(x)dx$$ \pause

Também é vista como o ``centro da distribuição de probabilidade''. \pause

Veremos o significado a seguir. \pause

{\bf Variância de uma variável aleatória contínua:} \pause
$$VAR(X) = \displaystyle \int_{- \infty}^{+ \infty}(x - E(X))^2 dx$$ \pause
ou
$$VAR(X) = E(X^2) - (E(X))^2$$
	
\end{frame}

\section{Principais Distribuições de Probabilidades - Contínuas}

\begin{frame}
\frametitle{Distribuição Uniforme}

{\bf Definição:}

Uma variável aleatória contínua $X$ tem distribuição uniforme de probabilidades no intervalo $[a,b]$ se a sua função de densidade de probabilidade é dada por: \pause
$$f(x) = \begin{cases}
k \text{ se } a \leq x \leq b \\
0 \text{ se } x < a \text{ ou } x > b
\end{cases}$$
\pause
{\bf GRÁFICO}
\pause

{\bf Exercício:} Determinar o valor de $k$.

\end{frame}

\begin{frame}
\frametitle{Distribuição Exponencial}

{\bf Definição:}

Uma variável aleatória contínua $X$ tem distribuição exponencial de probabilidades se a sua função de densidade de probabilidade é dada por:
	\pause
	
$$f(x) = \begin{cases}
\lambda e^{-\lambda x} &\text{ se } x \geq 0 \\
0 &\text{ se } x < 0
\end{cases}$$
\pause

{\bf GRÁFICO}

\end{frame}

\begin{frame}
\frametitle{Distribuição Normal}

Uma variável aleatória contínua $X$ tem distribuição normal de probabilidades se a sua função de densidade de probabilidade é dada por:

\pause
$$
f(x) = \displaystyle \frac{1}{\sigma \sqrt{2\pi}}e^{- \displaystyle \frac{1}{2}\left( \displaystyle \frac{x - \mu}{\sigma} \right)^2 } \text{ , para } -\infty < x < \infty
$$
\pause
{\bf GRÁFICO}
	
\end{frame}

\begin{frame}
\frametitle{Distribuição Normal}

As principais características da distribuição Normal são: \pause
\begin{enumerate}
	\item O ponto de máximo de $f(x)$ é o ponto $X = \mu$; \pause
	\item Os pontos de inflexão da função são: $X = \mu + \sigma$ e $X = \mu - \sigma$; \pause
	\item A curva é simétrica com relação a $\mu$; \pause
	\item $E(X) = \mu$ e $VAR(X) = \sigma^2$. \pause
\end{enumerate}

Utilizaremos a seguinte notação:
$$X: N(\mu,\sigma^2)$$ \pause
$X$ tem distribuição normal com média $\mu$ e variância $\sigma^2$.
\end{frame}


\begin{frame}
\frametitle{Distribuição Normal}

Seja $X: N(\mu,\sigma^2)$, definimos: \pause
$$Z = \displaystyle \frac{X - \mu}{\sigma}$$

\pause
$Z$ é chamada de \emph{variável normal reduzida, normal padronizada} ou \emph{variável normalizada}.

\pause
$Z$ possui $E(X) = 0$ e $VAR(X) = 1$.

\pause
A vantagem de utilizar a variável $Z$ é a utilização da tabela.
\end{frame}

\begin{frame}
\frametitle{Uso da Tabela}

Exemplo: Seja $X:N(100,25)$.
\begin{enumerate}
	\item $P(100 \leq X \leq 106)$ \pause
	\item $P(89 \leq X \leq 107)$ \pause
	\item $P(112 \leq X \leq 116)$ \pause
	\item $P(X \geq 108)$
\end{enumerate}

\end{frame}

\begin{frame}
	\frametitle{Uso do GeoGebra}
	
	Exemplo: Seja $X:N(100,25)$.
	\begin{enumerate}
		\item $P(100 \leq X \leq 106)$ \pause
		\item $P(89 \leq X \leq 107)$ \pause
		\item $P(112 \leq X \leq 116)$ \pause
		\item $P(X \geq 108)$
	\end{enumerate}
	
\end{frame}

\begin{frame}
\frametitle{Exemplo}

Um fabricante de baterias sabe, por experiência passada, que as baterias de sua fabricação têm vida média de $600$ dias e desvio-padrão de $100$ dias, sendo que a duração tem aproximadamente distribuição normal. Oferece uma garantia de $312$ dias, isto é, troca as baterias que apresentem falhas nesse período. Fabrica $10.000$ baterias mensalmente. Quantas deverá trocar pelo uso da garantia, mensalmente?

\end{frame}

\begin{frame}
\frametitle{Exemplo - Solução}
{\it Resolução:}

\pause
$X:$ é a duração da bateria e assim \pause $\begin{cases}
\mu = 600 \text{ dias} \\
\sigma = 100 \text{ dias}
\end{cases}
$

\pause
Logo,

\pause
$Z = \displaystyle \frac{X - 600}{100}$

\pause
Queremos determinar a $P(X < 312)$. 

\pause
Utilizando o GeoGebra (e a Tabela) determinaremos qual é essa probabilidade.

\pause
Assim, $P(X < 312) = 0,001988 \approx 0,002$. 

\pause
Para determinarmos quantas baterias serão substituídas mensalmente fazemos:

\pause
$10000 \times 0,001988 = 19,88 = 20$ baterias.

\pause
\begin{flushright}
	$\blacksquare$
\end{flushright}
\end{frame}

\end{document}

