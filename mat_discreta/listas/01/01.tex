\documentclass[oneside,a4paper,12pt]{article}
\usepackage[english,brazilian]{babel}
\usepackage{multicol}
\usepackage{textcomp}
\usepackage[alf]{abntex2cite}
\usepackage[utf8]{inputenc}
\usepackage[T1]{fontenc}
\usepackage{amsmath,amssymb,exscale}
\usepackage[top=20mm, bottom=20mm, left=20mm, right=20mm]{geometry}%margens cima, baixo, esquerda direita
\usepackage{framed}
\usepackage{booktabs} %Pacote para deixar tabelas mais bonitas.
\usepackage{color} %Pacote de Cores
\usepackage{hyperref} %Pacotes para Hiperlinks
\usepackage{graphicx} %Pacote de imagens
\graphicspath{{./Figuras/}}%Direciona as imagens para uma pasta chamada "Figuras" (uso isso para organizar. Uma vez que todas as imagens vao ficar em uma pasta isolada)    
\definecolor{shadecolor}{rgb}{0.8,0.8,0.8}

%FAZ EDICOES AQUI (somente no conteudo que esta entre entre as ultimas  chaves de cada linha!!!)
\newcommand{\universidade}{Universidade Tecnológica Federal do Paraná}
\newcommand{\centro}{Câmpus Cornélio Procópio}
\newcommand{\departamento}{Departamento Acadêmico de Matemática}
\newcommand{\curso}{Engenharia da Computação}
\newcommand{\professores}{Matheus Pimenta}
\newcommand{\disciplina}{Matemática Discreta - EC34G}
%\newcommand{\tema}{Lista 01}
%\newcommand{\turma}{MA31G}
%\newcommand{\data}{Março de 2019}%{\today}
%\newcommand{\tempodeaula}{30 minutos}
%\newcommand{\prerequisitos}{Matrizes, Transformações Lineares e Bases}
%ATE AQUI !!!	

\begin{document}
	\pagestyle{empty}
	
	\begin{center}
		\includegraphics[width=\linewidth/8]{logo.jpg}%LOGOTIPO DA INSTITUICAO
	 	\vspace{2pt} 	
		
		\universidade
		\par
		\centro
		\par
		\departamento
		\par
	%	Curso de \curso
		\par
		\vspace{12pt}
		\LARGE \textbf{Lista 01}
		
	\end{center}
	
	\vspace{12pt}
	
	\begin{tabular}{ |l|p{12cm}| }
		
		\hline
		\multicolumn{2}{|c|}{\textbf{Dados de Identificação}} \\
		\hline
		Professor:         &    \professores           \\
		\hline
		Disciplina:        &    \disciplina          \\
		\hline
	%	Tema:              &    \tema                \\
	%	\hline
	%	Pré-requisito	:  &    \prerequisitos         \\
	%	\hline
		Aluno:             &                   \\
	%	\hline
	%	Data:              &    \data                \\
	%	\hline
	%	Duração da aula:   &    \tempodeaula         \\
		\hline
		
	\end{tabular}
	\vspace{6pt}
	
	
	\begin{snugshade}
	\end{snugshade}

\begin{enumerate}



	\item Indique o antecedente e o consequente em cada uma das sentenças:
	
	\textbf{DICA: }Reescreva na forma de {\it se-então}.
	
	\begin{enumerate}
		\item Se a chuva continuar, o rio vai transbordar.
		\item Uma condição suficiente para a falha de uma rede é que a chave geral pare de funcionar.
		\item Os abacates só estão maduros quando estão escuros e macios.
		\item Uma boa dieta é uma condição necessária para um gato saudável.
	\end{enumerate}


	\item Construa as tabelas-verdade para as seguintes {\it wffs}.
	
	\textbf{DICA: } $A \leftrightarrow B$ só é verdade quando $A$ e $B$ possuem o mesmo valor-verdade.
	
	\begin{enumerate}
		\item $(A \rightarrow B) \leftrightarrow (B \rightarrow A)$
		\item $(A \lor \neg A) \rightarrow (B \lor \neg B)$
		\item $\neg [(A \land \neg B) \rightarrow \neg C]$
		\item $(A \rightarrow B) \leftrightarrow (\neg B \rightarrow \neg A)$
	\end{enumerate}

	\item Apresente a tabela-verdade para a seguinte proposição:
	$$p \leftrightarrow q \Leftrightarrow (p \rightarrow q) \land (q \rightarrow p)$$
	Que é a tabela-verdade para a {\it bicondição}

	\item Apresente a tabela-verdade para a seguinte proposição:
	$$p \rightarrow q \Leftrightarrow \neg q \rightarrow \neg p$$
	Que é a tabela-verdade para a {\it contraposição}

	\item Apresente a tabela-verdade para a seguinte proposição:
	$$p \rightarrow q \Leftrightarrow p \land \neg q \rightarrow Falso$$
	Que é a tabela-verdade para a {\it redução ao absurdo}

	\item Suponha o conjunto universo $\{1,2,3,4,5,6,7,8,9\}$. Apresente a negação de cada proposição e se possível um contra-exemplo de cada.
	\begin{enumerate}
		\item $(\forall x)(x+5<12)$
		\item $(\forall x)(x \text{ é primo})$
		\item $(\forall x)(x^2 > 1)$
		\item $(\forall x)(x \text{ é par})$
	\end{enumerate}
	
	\item Determine o valor-verdade (VERDADEIRO ou FALSO) para cada uma das seguintes proposições sabendo que o conjunto universo a ser considerado é $\{2,3,4,5,6,7,8,9\}$:
	\begin{multicols}{2}
	\begin{enumerate}
		\item $ (\forall x)(\forall y)(x+5 < y + 12)$
		\item $ (\forall x)(\exists y)(xy \text{ não é primo})$
		\item $ (\exists y)(\forall x)(xy \text{ não é primo})$
		\item $ (\exists x)(\exists y)(x^2>y)$
		\item $ (\forall x)(\exists y)(x^2>y)$
		\item $ (\exists x)(\forall y)(x^2>y)$
		\item $ (\forall x)(\forall y)(\forall z)(x+y>z)$
		\item $ (\exists x)(\forall y)(\forall z)(x+y>z)$
		\item $ (\forall x)(\exists y)(\forall z)(x+y>z)$
		\item $ (\forall x)(\forall y)(\exists z)(x+y>z)$
		\item $ (\forall x)(\exists y)(\exists z)(x+y>z)$
		\item $ (\exists x)(\forall y)(\exists z)(x+y>z)$
		\item $ (\exists x)(\exists y)(\forall z)(x+y>z)$
	\end{enumerate}
	\end{multicols}

	\item Negue todas as proposições do exercício anterior.

	\item Mostre que se elevarmos um número ímpar ao quadrado, seu resultado também será um número ímpar.

	\item Mostre que se elevarmos agora um número par ao quadrado, seu resultado será um número par.

	\item Se somarmos dois números $m + n$ e o resultado for um número par, então $m$ e $n$ são pares. Prove essa afirmação.

	\item Mostre que os números primos são infinitos.
	
	\item Mostre que:
	$$1+2+3+4+\dots+n = \frac{(n+1)n}{2}$$

	\item Mostre que:
	$$2^{n} \geq 1 + n ,\forall n \in \mathbb{N}$$
	
	
	\item Mostre que:
	$$1^3 + 2^3 + \dots + n^3 > \frac{n^4}{4}, \forall n \in \mathbb{N}^{*}$$

	\item Mostre que $\forall n \in \mathbb{N}$:
	$$2+5+8+11+14+17+\dots+(2+3n) = \frac{(n+1)(4+3n)}{2}$$
	
	\item Mostre que:
	$$ 2+4+6+8+\dots+2n = (n+1)n$$

\end{enumerate}


	
\end{document}
